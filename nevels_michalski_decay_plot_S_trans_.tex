% This file was created by matlab2tikz.
%
\begin{tikzpicture}

\begin{axis}[%
width=5.086in,
height=5.442in,
at={(0.853in,0.734in)},
scale only axis,
xmode=log,
xmin=0.01,
xmax=10000,
xminorticks=true,
xlabel style={font=\bfseries},
xlabel={$\frac{x}{\lambda}$},
ymode=log,
ymin=0.0001,
ymax=10000,
yminorticks=true,
ylabel style={font=\bfseries},
ylabel={Creeping Wave},
axis background/.style={fill=white},
title style={font=\bfseries},
title={Creeping Wave Decay}
]
\addplot [color=black,solid,line width=1.4pt,forget plot]
  table[row sep=crcr]{%
0.01	2046.7812808434\\
0.0103961584633854	2025.75090403866\\
0.0107923169267707	2005.35450431569\\
0.0111884753901561	1985.56292135023\\
0.0115846338535414	1966.3486338484\\
0.0119807923169268	1947.68565667246\\
0.0123769507803121	1929.54944486962\\
0.0127731092436975	1911.91680412099\\
0.0131692677070828	1894.7658071622\\
0.0135654261704682	1878.07571575932\\
0.0139615846338535	1861.82690785378\\
0.0143577430972389	1846.00080951741\\
0.0147539015606242	1830.57983138438\\
0.0151500600240096	1815.54730925078\\
0.015546218487395	1800.88744855447\\
0.0159423769507803	1786.58527246826\\
0.0163385354141657	1772.62657335872\\
0.016734693877551	1758.99786738008\\
0.0171308523409364	1745.68635198948\\
0.0175270108043217	1732.67986618467\\
0.0179231692677071	1719.9668532793\\
0.0183193277310924	1707.53632604421\\
0.0187154861944778	1695.37783405492\\
0.0191116446578631	1683.48143309701\\
0.0195078031212485	1671.83765649123\\
0.0199039615846339	1660.43748821012\\
0.0203001200480192	1649.27233766655\\
0.0206962785114046	1638.33401606328\\
0.0210924369747899	1627.61471420006\\
0.0214885954381753	1617.10698164199\\
0.0218847539015606	1606.8037071599\\
0.022280912364946	1596.69810035898\\
0.0226770708283313	1586.78367441838\\
0.0230732292917167	1577.05422986929\\
0.023469387755102	1567.5038393442\\
0.0238655462184874	1558.12683323469\\
0.0242617046818727	1548.91778619907\\
0.0246578631452581	1539.87150446563\\
0.0250540216086435	1530.98301388051\\
0.0254501800720288	1522.24754865284\\
0.0258463385354142	1513.66054075303\\
0.0262424969987995	1505.217609923\\
0.0266386554621849	1496.91455425974\\
0.0270348139255702	1488.74734133649\\
0.0274309723889556	1480.71209982798\\
0.0278271308523409	1472.80511160829\\
0.0282232893157263	1465.02280429244\\
0.0286194477791116	1457.36174419405\\
0.029015606242497	1449.8186296738\\
0.0294117647058824	1442.39028485488\\
0.0298079231692677	1435.07365368295\\
0.0302040816326531	1427.86579430997\\
0.0306002400960384	1420.76387378234\\
0.0309963985594238	1413.76516301514\\
0.0313925570228091	1406.86703203541\\
0.0317887154861945	1400.06694547848\\
0.0321848739495798	1393.36245832253\\
0.0325810324129652	1386.75121184721\\
0.0329771908763505	1380.23092980334\\
0.0333733493397359	1373.79941478136\\
0.0337695078031212	1367.45454476711\\
0.0341656662665066	1361.19426987396\\
0.034561824729892	1355.01660924139\\
0.0349579831932773	1348.91964809043\\
0.0353541416566627	1342.90153492706\\
0.035750300120048	1336.96047888521\\
0.0361464585834334	1331.0947472016\\
0.0365426170468187	1325.30266281495\\
0.0369387755102041	1319.58260208279\\
0.0373349339735894	1313.93299260917\\
0.0377310924369748	1308.35231117738\\
0.0381272509003601	1302.83908178187\\
0.0385234093637455	1297.39187375389\\
0.0389195678271309	1292.00929997591\\
0.0393157262905162	1286.69001517999\\
0.0397118847539016	1281.43271432558\\
0.0401080432172869	1276.23613105263\\
0.0405042016806723	1271.09903620591\\
0.0409003601440576	1266.02023642685\\
0.041296518607443	1260.99857280939\\
0.0416926770708283	1256.0329196165\\
0.0420888355342137	1251.12218305421\\
0.042484993997599	1246.26530010021\\
0.0428811524609844	1241.46123738426\\
0.0432773109243697	1236.70899011775\\
0.0436734693877551	1232.00758106991\\
0.0440696278511405	1227.35605958839\\
0.0444657863145258	1222.75350066192\\
0.0448619447779112	1218.19900402309\\
0.0452581032412965	1213.69169328903\\
0.0456542617046819	1209.23071513846\\
0.0460504201680672	1204.81523852304\\
0.0464465786314526	1200.44445391156\\
0.0468427370948379	1196.11757256526\\
0.0472388955582233	1191.83382584289\\
0.0476350540216086	1187.59246453402\\
0.048031212484994	1183.39275821931\\
0.0484273709483794	1179.23399465646\\
0.0488235294117647	1175.11547919061\\
0.0492196878751501	1171.03653418812\\
0.0496158463385354	1166.99649849261\\
0.0500120048019208	1162.99472690222\\
0.0504081632653061	1159.03058966718\\
0.0508043217286915	1155.10347200667\\
0.0512004801920768	1151.21277364428\\
0.0515966386554622	1147.35790836103\\
0.0519927971188475	1143.5383035653\\
0.0523889555822329	1139.75339987888\\
0.0527851140456183	1136.00265073847\\
0.0531812725090036	1132.28552201185\\
0.053577430972389	1128.60149162826\\
0.0539735894357743	1124.95004922219\\
0.0543697478991597	1121.33069579011\\
0.054765906362545	1117.7429433596\\
0.0551620648259304	1114.1863146703\\
0.0555582232893157	1110.66034286617\\
0.0559543817527011	1107.16457119872\\
0.0563505402160864	1103.69855274053\\
0.0567466986794718	1100.26185010885\\
0.0571428571428571	1096.85403519875\\
0.0575390156062425	1093.47468892545\\
0.0579351740696279	1090.12340097545\\
0.0583313325330132	1086.79976956618\\
0.0587274909963986	1083.50340121367\\
0.0591236494597839	1080.23391050807\\
0.0595198079231693	1076.99091989671\\
0.0599159663865546	1073.77405947421\\
0.06031212484994	1070.58296677963\\
0.0607082833133253	1067.41728660022\\
0.0611044417767107	1064.27667078148\\
0.061500600240096	1061.16077804348\\
0.0618967587034814	1058.06927380302\\
0.0622929171668668	1055.0018300014\\
0.0626890756302521	1051.95812493782\\
0.0630852340936374	1048.93784310787\\
0.0634813925570228	1045.94067504717\\
0.0638775510204082	1042.96631717987\\
0.0642737094837935	1040.01447167184\\
0.0646698679471789	1037.08484628837\\
0.0650660264105642	1034.17715425624\\
0.0654621848739496	1031.29111412999\\
0.0658583433373349	1028.4264496622\\
0.0662545018007203	1025.58288967772\\
0.0666506602641056	1022.76016795165\\
0.067046818727491	1019.95802309092\\
0.0674429771908764	1017.17619841939\\
0.0678391356542617	1014.41444186634\\
0.0682352941176471	1011.67250585819\\
0.0686314525810324	1008.95014721338\\
0.0690276110444178	1006.2471270403\\
0.0694237695078031	1003.56321063814\\
0.0698199279711885	1000.89816740056\\
0.0702160864345738	998.251770722142\\
0.0706122448979592	995.62379790741\\
0.0710084033613445	993.014030082497\\
0.0714045618247299	990.422252109213\\
0.0718007202881152	987.84825250153\\
0.0721968787515006	985.29182334437\\
0.0725930372148859	982.752760214633\\
0.0729891956782713	980.230862104369\\
0.0733853541416567	977.725931346064\\
0.073781512605042	975.237773539924\\
0.0741776710684274	972.766197483133\\
0.0745738295318127	970.311015100997\\
0.0749699879951981	967.872041379928\\
0.0753661464585834	965.449094302198\\
0.0757623049219688	963.041994782417\\
0.0761584633853541	960.650566605676\\
0.0765546218487395	958.274636367306\\
0.0769507803121249	955.914033414196\\
0.0773469387755102	953.568589787632\\
0.0777430972388956	951.2381401676\\
0.0781392557022809	948.92252181852\\
0.0785354141656663	946.621574536355\\
0.0789315726290516	944.335140597056\\
0.079327731092437	942.063064706317\\
0.0797238895558223	939.805193950579\\
0.0801200480192077	937.561377749261\\
0.080516206482593	935.331467808173\\
0.0809123649459784	933.115318074085\\
0.0813085234093637	930.91278469041\\
0.0817046818727491	928.723725953969\\
0.0821008403361344	926.548002272811\\
0.0824969987995198	924.385476125059\\
0.0828931572629052	922.236012018739\\
0.0832893157262905	920.099476452581\\
0.0836854741896759	917.975737877755\\
0.0840816326530612	915.864666660514\\
0.0844777911164466	913.766135045722\\
0.0848739495798319	911.680017121247\\
0.0852701080432173	909.606188783178\\
0.0856662665066026	907.544527701866\\
0.086062424969988	905.494913288741\\
0.0864585834333733	903.457226663911\\
0.0868547418967587	901.431350624496\\
0.0872509003601441	899.41716961369\\
0.0876470588235294	897.414569690534\\
0.0880432172869148	895.423438500367\\
0.0884393757503001	893.443665245956\\
0.0888355342136855	891.475140659269\\
0.0892316926770708	889.517756973875\\
0.0896278511404562	887.57140789798\\
0.0900240096038415	885.635988588036\\
0.0904201680672269	883.711395622957\\
0.0908163265306122	881.797526978886\\
0.0912124849939976	879.894282004529\\
0.0916086434573829	878.001561397015\\
0.0920048019207683	876.119267178291\\
0.0924009603841537	874.247302672024\\
0.092797118847539	872.385572480998\\
0.0931932773109244	870.533982465004\\
0.0935894357743097	868.692439719197\\
0.0939855942376951	866.860852552914\\
0.0943817527010804	865.039130468943\\
0.0947779111644658	863.227184143228\\
0.0951740696278511	861.424925405003\\
0.0955702280912365	859.632267217343\\
0.0959663865546218	857.849123658112\\
0.0963625450180072	856.075409901321\\
0.0967587034813926	854.311042198858\\
0.0971548619447779	852.555937862602\\
0.0975510204081633	850.810015246903\\
0.0979471788715486	849.073193731419\\
0.098343337334934	847.345393704302\\
0.0987394957983193	845.626536545726\\
0.0991356542617047	843.916544611748\\
0.09953181272509	842.215341218489\\
0.0999279711884754	840.522850626645\\
0.100324129651861	838.838998026285\\
0.100720288115246	837.163709521976\\
0.101116446578631	835.496912118185\\
0.101512605042017	833.838533704974\\
0.101908763505402	832.188503043978\\
0.102304921968788	830.54674975466\\
0.102701080432173	828.913204300824\\
0.103097238895558	827.2877979774\\
0.103493397358944	825.670462897481\\
0.103889555822329	824.061131979609\\
0.104285714285714	822.459738935307\\
0.1046818727491	820.866218256847\\
0.105078031212485	819.280505205256\\
0.10547418967587	817.702535798544\\
0.105870348139256	816.132246800159\\
0.106266506602641	814.569575707658\\
0.106662665066026	813.014460741589\\
0.107058823529412	811.466840834588\\
0.107454981992797	809.926655620667\\
0.107851140456182	808.393845424714\\
0.108247298919568	806.868351252182\\
0.108643457382953	805.350114778963\\
0.109039615846339	803.839078341457\\
0.109435774309724	802.335184926818\\
0.109831932773109	800.838378163378\\
0.110228091236495	799.348602311245\\
0.11062424969988	797.865802253077\\
0.111020408163265	796.389923485012\\
0.111416566626651	794.92091210777\\
0.111812725090036	793.458714817911\\
0.112208883553421	792.003278899249\\
0.112605042016807	790.554552214422\\
0.113001200480192	789.112483196607\\
0.113397358943577	787.677020841381\\
0.113793517406963	786.248114698735\\
0.114189675870348	784.825714865215\\
0.114585834333734	783.409771976207\\
0.114981992797119	782.000237198361\\
0.115378151260504	780.597062222138\\
0.11577430972389	779.200199254491\\
0.116170468187275	777.809601011669\\
0.11656662665066	776.425220712153\\
0.116962785114046	775.047012069701\\
0.117358943577431	773.674929286523\\
0.117755102040816	772.308927046563\\
0.118151260504202	770.948960508905\\
0.118547418967587	769.594985301276\\
0.118943577430972	768.246957513679\\
0.119339735894358	766.904833692113\\
0.119735894357743	765.56857083241\\
0.120132052821128	764.238126374173\\
0.120528211284514	762.91345819481\\
0.120924369747899	761.594524603674\\
0.121320528211285	760.281284336295\\
0.12171668667467	758.973696548709\\
0.122112845138055	757.67172081188\\
0.122509003601441	756.375317106213\\
0.122905162064826	755.084445816158\\
0.123301320528211	753.799067724897\\
0.123697478991597	752.519144009124\\
0.124093637454982	751.24463623391\\
0.124489795918367	749.975506347635\\
0.124885954381753	748.711716677028\\
0.125282112845138	747.453229922261\\
0.125678271308523	746.200009152139\\
0.126074429771909	744.952017799356\\
0.126470588235294	743.709219655833\\
0.126866746698679	742.471578868125\\
0.127262905162065	741.239059932902\\
0.12765906362545	740.011627692506\\
0.128055222088836	738.789247330567\\
0.128451380552221	737.571884367698\\
0.128847539015606	736.359504657253\\
0.129243697478992	735.152074381147\\
0.129639855942377	733.949560045748\\
0.130036014405762	732.75192847783\\
0.130432172869148	731.559146820581\\
0.130828331332533	730.371182529687\\
0.131224489795918	729.188003369461\\
0.131620648259304	728.00957740904\\
0.132016806722689	726.83587301864\\
0.132412965186074	725.66685886586\\
0.13280912364946	724.50250391205\\
0.133205282112845	723.342777408733\\
0.133601440576231	722.187648894075\\
0.133997599039616	721.037088189414\\
0.134393757503001	719.891065395839\\
0.134789915966387	718.749550890816\\
0.135186074429772	717.612515324875\\
0.135582232893157	716.479929618329\\
0.135978391356543	715.351764958061\\
0.136374549819928	714.227992794343\\
0.136770708283313	713.108584837708\\
0.137166866746699	711.993513055871\\
0.137563025210084	710.882749670685\\
0.137959183673469	709.776267155154\\
0.138355342136855	708.674038230479\\
0.13875150060024	707.576035863154\\
0.139147659063625	706.482233262096\\
0.139543817527011	705.392603875826\\
0.139939975990396	704.307121389685\\
0.140336134453782	703.225759723088\\
0.140732292917167	702.14849302682\\
0.141128451380552	701.075295680374\\
0.141524609843938	700.00614228932\\
0.141920768307323	698.941007682715\\
0.142316926770708	697.879866910549\\
0.142713085234094	696.822695241228\\
0.143109243697479	695.769468159089\\
0.143505402160864	694.720161361956\\
0.14390156062425	693.674750758724\\
0.144297719087635	692.633212466979\\
0.14469387755102	691.595522810651\\
0.145090036014406	690.561658317702\\
0.145486194477791	689.531595717841\\
0.145882352941176	688.505311940275\\
0.146278511404562	687.482784111488\\
0.146674669867947	686.463989553052\\
0.147070828331333	685.448905779469\\
0.147466986794718	684.437510496039\\
0.147863145258103	683.429781596759\\
0.148259303721489	682.425697162252\\
0.148655462184874	681.425235457724\\
0.149051620648259	680.428374930941\\
0.149447779111645	679.435094210248\\
0.14984393757503	678.445372102599\\
0.150240096038415	677.459187591625\\
0.150636254501801	676.47651983572\\
0.151032412965186	675.497348166159\\
0.151428571428571	674.521652085234\\
0.151824729891957	673.549411264423\\
0.152220888355342	672.580605542575\\
0.152617046818727	671.615214924125\\
0.153013205282113	670.653219577328\\
0.153409363745498	669.694599832522\\
0.153805522208884	668.739336180406\\
0.154201680672269	667.787409270347\\
0.154597839135654	666.838799908708\\
0.15499399759904	665.893489057192\\
0.155390156062425	664.951457831217\\
0.15578631452581	664.012687498302\\
0.156182472989196	663.077159476482\\
0.156578631452581	662.144855332742\\
0.156974789915966	661.215756781462\\
0.157370948379352	660.289845682898\\
0.157767106842737	659.367104041665\\
0.158163265306122	658.447514005254\\
0.158559423769508	657.531057862561\\
0.158955582232893	656.617718042432\\
0.159351740696279	655.707477112233\\
0.159747899159664	654.800317776432\\
0.160144057623049	653.896222875205\\
0.160540216086435	652.995175383054\\
0.16093637454982	652.097158407443\\
0.161332533013205	651.202155187458\\
0.161728691476591	650.31014909247\\
0.162124849939976	649.42112362083\\
0.162521008403361	648.535062398571\\
0.162917166866747	647.651949178126\\
0.163313325330132	646.771767837065\\
0.163709483793517	645.894502376847\\
0.164105642256903	645.020136921588\\
0.164501800720288	644.14865571684\\
0.164897959183673	643.28004312839\\
0.165294117647059	642.414283641071\\
0.165690276110444	641.551361857586\\
0.16608643457383	640.691262497352\\
0.166482593037215	639.833970395352\\
0.1668787515006	638.979470501003\\
0.167274909963986	638.127747877038\\
0.167671068427371	637.278787698402\\
0.168067226890756	636.43257525116\\
0.168463385354142	635.589095931416\\
0.168859543817527	634.748335244254\\
0.169255702280912	633.910278802679\\
0.169651860744298	633.074912326578\\
0.170048019207683	632.242221641694\\
0.170444177671068	631.412192678608\\
0.170840336134454	630.584811471738\\
0.171236494597839	629.760064158343\\
0.171632653061224	628.937936977547\\
0.17202881152461	628.118416269367\\
0.172424969987995	627.301488473757\\
0.172821128451381	626.48714012966\\
0.173217286914766	625.675357874076\\
0.173613445378151	624.866128441137\\
0.174009603841537	624.059438661188\\
0.174405762304922	623.255275459891\\
0.174801920768307	622.45362585733\\
0.175198079231693	621.654476967126\\
0.175594237695078	620.857815995568\\
0.175990396158463	620.06363024075\\
0.176386554621849	619.271907091719\\
0.176782713085234	618.48263402763\\
0.177178871548619	617.695798616917\\
0.177575030012005	616.911388516464\\
0.17797118847539	616.129391470799\\
0.178367346938776	615.34979531128\\
0.178763505402161	614.572587955307\\
0.179159663865546	613.797757405529\\
0.179555822328932	613.02529174907\\
0.179951980792317	612.255179156758\\
0.180348139255702	611.487407882364\\
0.180744297719088	610.721966261853\\
0.181140456182473	609.958842712635\\
0.181536614645858	609.198025732834\\
0.181932773109244	608.439503900559\\
0.182328931572629	607.683265873184\\
0.182725090036014	606.929300386639\\
0.1831212484994	606.177596254703\\
0.183517406962785	605.428142368312\\
0.18391356542617	604.68092769487\\
0.184309723889556	603.935941277567\\
0.184705882352941	603.193172234707\\
0.185102040816327	602.452609759042\\
0.185498199279712	601.714243117114\\
0.185894357743097	600.978061648605\\
0.186290516206483	600.244054765688\\
0.186686674669868	599.512211952397\\
0.187082833133253	598.782522763991\\
0.187478991596639	598.054976826331\\
0.187875150060024	597.329563835266\\
0.188271308523409	596.606273556022\\
0.188667466986795	595.885095822597\\
0.18906362545018	595.166020537164\\
0.189459783913565	594.44903766948\\
0.189855942376951	593.734137256304\\
0.190252100840336	593.021309400815\\
0.190648259303722	592.310544272043\\
0.191044417767107	591.601832104303\\
0.191440576230492	590.895163196628\\
0.191836734693878	590.190527912227\\
0.192232893157263	589.487916677926\\
0.192629051620648	588.78731998363\\
0.193025210084034	588.088728381786\\
0.193421368547419	587.39213248685\\
0.193817527010804	586.697522974764\\
0.19421368547419	586.004890582432\\
0.194609843937575	585.314226107209\\
0.19500600240096	584.625520406388\\
0.195402160864346	583.9387643967\\
0.195798319327731	583.25394905381\\
0.196194477791116	582.571065411827\\
0.196590636254502	581.890104562812\\
0.196986794717887	581.211057656299\\
0.197382953181273	580.53391589881\\
0.197779111644658	579.858670553386\\
0.198175270108043	579.185312939118\\
0.198571428571429	578.513834430677\\
0.198967587034814	577.844226457862\\
0.199363745498199	577.17648050514\\
0.199759903961585	576.510588111197\\
0.20015606242497	575.846540868495\\
0.200552220888355	575.184330422826\\
0.200948379351741	574.523948472878\\
0.201344537815126	573.865386769805\\
0.201740696278511	573.208637116793\\
0.202136854741897	572.553691368641\\
0.202533013205282	571.900541431341\\
0.202929171668667	571.249179261661\\
0.203325330132053	570.599596866734\\
0.203721488595438	569.951786303655\\
0.204117647058824	569.305739679071\\
0.204513805522209	568.661449148786\\
0.204909963985594	568.018906917369\\
0.20530612244898	567.378105237754\\
0.205702280912365	566.739036410864\\
0.20609843937575	566.101692785217\\
0.206494597839136	565.466066756554\\
0.206890756302521	564.832150767457\\
0.207286914765906	564.199937306981\\
0.207683073229292	563.569418910285\\
0.208079231692677	562.940588158263\\
0.208475390156062	562.313437677187\\
0.208871548619448	561.687960138347\\
0.209267707082833	561.064148257695\\
0.209663865546218	560.441994795497\\
0.210060024009604	559.821492555983\\
0.210456182472989	559.202634387004\\
0.210852340936375	558.585413179687\\
0.21124849939976	557.969821868105\\
0.211644657863145	557.355853428934\\
0.212040816326531	556.743500881127\\
0.212436974789916	556.132757285583\\
0.212833133253301	555.523615744824\\
0.213229291716687	554.916069402672\\
0.213625450180072	554.31011144393\\
0.214021608643457	553.705735094066\\
0.214417767106843	553.102933618899\\
0.214813925570228	552.501700324294\\
0.215210084033613	551.902028555848\\
0.215606242496999	551.30391169859\\
0.216002400960384	550.70734317668\\
0.21639855942377	550.112316453108\\
0.216794717887155	549.5188250294\\
0.21719087635054	548.926862445326\\
0.217587034813926	548.336422278607\\
0.217983193277311	547.747498144632\\
0.218379351740696	547.160083696167\\
0.218775510204082	546.574172623078\\
0.219171668667467	545.989758652051\\
0.219567827130852	545.406835546312\\
0.219963985594238	544.825397105356\\
0.220360144057623	544.245437164671\\
0.220756302521008	543.666949595474\\
0.221152460984394	543.089928304441\\
0.221548619447779	542.514367233441\\
0.221944777911164	541.940260359279\\
0.22234093637455	541.36760169343\\
0.222737094837935	540.796385281786\\
0.223133253301321	540.226605204404\\
0.223529411764706	539.658255575243\\
0.223925570228091	539.091330541925\\
0.224321728691477	538.525824285481\\
0.224717887154862	537.961731020106\\
0.225114045618247	537.399044992916\\
0.225510204081633	536.837760483707\\
0.225906362545018	536.277871804715\\
0.226302521008403	535.719373300379\\
0.226698679471789	535.162259347107\\
0.227094837935174	534.606524353041\\
0.227490996398559	534.05216275783\\
0.227887154861945	533.499169032398\\
0.22828331332533	532.947537678718\\
0.228679471788716	532.397263229588\\
0.229075630252101	531.848340248408\\
0.229471788715486	531.30076332896\\
0.229867947178872	530.75452709519\\
0.230264105642257	530.209626200987\\
0.230660264105642	529.666055329974\\
0.231056422569028	529.12380919529\\
0.231452581032413	528.582882539383\\
0.231848739495798	528.043270133797\\
0.232244897959184	527.504966778969\\
0.232641056422569	526.967967304017\\
0.233037214885954	526.432266566543\\
0.23343337334934	525.897859452427\\
0.233829531812725	525.364740875628\\
0.23422569027611	524.832905777984\\
0.234621848739496	524.302349129019\\
0.235018007202881	523.773065925741\\
0.235414165666266	523.245051192458\\
0.235810324129652	522.718299980576\\
0.236206482593037	522.192807368417\\
0.236602641056423	521.668568461027\\
0.236998799519808	521.145578389989\\
0.237394957983193	520.623832313238\\
0.237791116446579	520.103325414881\\
0.238187274909964	519.584052905007\\
0.238583433373349	519.066010019514\\
0.238979591836735	518.549192019926\\
0.23937575030012	518.033594193218\\
0.239771908763505	517.519211851637\\
0.240168067226891	517.00604033253\\
0.240564225690276	516.494074998171\\
0.240960384153661	515.983311235587\\
0.241356542617047	515.473744456391\\
0.241752701080432	514.965370096611\\
0.242148859543818	514.458183616525\\
0.242545018007203	513.952180500493\\
0.242941176470588	513.447356256793\\
0.243337334933974	512.943706417459\\
0.243733493397359	512.441226538119\\
0.244129651860744	511.939912197833\\
0.24452581032413	511.439758998938\\
0.244921968787515	510.940762566885\\
0.2453181272509	510.442918550086\\
0.245714285714286	509.946222619758\\
0.246110444177671	509.450670469771\\
0.246506602641056	508.956257816492\\
0.246902761104442	508.462980398637\\
0.247298919567827	507.970833977118\\
0.247695078031212	507.479814334896\\
0.248091236494598	506.989917276834\\
0.248487394957983	506.501138629551\\
0.248883553421369	506.013474241273\\
0.249279711884754	505.526919981694\\
0.249675870348139	505.04147174183\\
0.250072028811525	504.557125433877\\
0.25046818727491	504.073876991075\\
0.250864345738295	503.591722367563\\
0.251260504201681	503.110657538245\\
0.251656662665066	502.630678498649\\
0.252052821128451	502.151781264795\\
0.252448979591837	501.673961873059\\
0.252845138055222	501.197216380037\\
0.253241296518607	500.721540862415\\
0.253637454981993	500.246931416835\\
0.254033613445378	499.773384159767\\
0.254429771908764	499.300895227377\\
0.254825930372149	498.829460775402\\
0.255222088835534	498.359076979018\\
0.25561824729892	497.889740032715\\
0.256014405762305	497.421446150176\\
0.25641056422569	496.954191564144\\
0.256806722689076	496.487972526307\\
0.257202881152461	496.022785307171\\
0.257599039615846	495.558626195938\\
0.257995198079232	495.09549150039\\
0.258391356542617	494.633377546763\\
0.258787515006002	494.172280679633\\
0.259183673469388	493.712197261797\\
0.259579831932773	493.253123674156\\
0.259975990396158	492.795056315599\\
0.260372148859544	492.337991602889\\
0.260768307322929	491.881925970547\\
0.261164465786315	491.426855870742\\
0.2615606242497	490.972777773174\\
0.261956782713085	490.519688164968\\
0.262352941176471	490.06758355056\\
0.262749099639856	489.616460451586\\
0.263145258103241	489.166315406779\\
0.263541416566627	488.717144971854\\
0.263937575030012	488.268945719407\\
0.264333733493397	487.821714238803\\
0.264729891956783	487.375447136075\\
0.265126050420168	486.930141033818\\
0.265522208883553	486.485792571084\\
0.265918367346939	486.042398403279\\
0.266314525810324	485.599955202063\\
0.266710684273709	485.158459655244\\
0.267106842737095	484.717908466684\\
0.26750300120048	484.278298356192\\
0.267899159663866	483.839626059429\\
0.268295318127251	483.401888327809\\
0.268691476590636	482.9650819284\\
0.269087635054022	482.529203643829\\
0.269483793517407	482.094250272182\\
0.269879951980792	481.660218626913\\
0.270276110444178	481.227105536749\\
0.270672268907563	480.79490784559\\
0.271068427370948	480.363622412424\\
0.271464585834334	479.933246111228\\
0.271860744297719	479.503775830878\\
0.272256902761104	479.075208475061\\
0.27265306122449	478.647540962178\\
0.273049219687875	478.22077022526\\
0.273445378151261	477.794893211875\\
0.273841536614646	477.369906884042\\
0.274237695078031	476.945808218141\\
0.274633853541417	476.522594204827\\
0.275030012004802	476.100261848944\\
0.275426170468187	475.678808169436\\
0.275822328931573	475.258230199267\\
0.276218487394958	474.83852498533\\
0.276614645858343	474.419689588368\\
0.277010804321729	474.001721082889\\
0.277406962785114	473.584616557082\\
0.277803121248499	473.168373112736\\
0.278199279711885	472.752987865158\\
0.27859543817527	472.338457943091\\
0.278991596638655	471.924780488638\\
0.279387755102041	471.511952657175\\
0.279783913565426	471.099971617279\\
0.280180072028812	470.688834550643\\
0.280576230492197	470.278538652002\\
0.280972388955582	469.869081129056\\
0.281368547418968	469.46045920239\\
0.281764705882353	469.052670105398\\
0.282160864345738	468.645711084211\\
0.282557022809124	468.239579397617\\
0.282953181272509	467.834272316988\\
0.283349339735894	467.429787126206\\
0.28374549819928	467.026121121592\\
0.284141656662665	466.623271611827\\
0.28453781512605	466.221235917884\\
0.284933973589436	465.820011372954\\
0.285330132052821	465.419595322375\\
0.285726290516206	465.019985123563\\
0.286122448979592	464.621178145936\\
0.286518607442977	464.223171770849\\
0.286914765906363	463.825963391523\\
0.287310924369748	463.429550412975\\
0.287707082833133	463.033930251952\\
0.288103241296519	462.639100336858\\
0.288499399759904	462.245058107692\\
0.288895558223289	461.851801015978\\
0.289291716686675	461.459326524698\\
0.28968787515006	461.067632108228\\
0.290084033613445	460.676715252269\\
0.290480192076831	460.286573453786\\
0.290876350540216	459.897204220938\\
0.291272509003601	459.508605073021\\
0.291668667466987	459.120773540396\\
0.292064825930372	458.733707164431\\
0.292460984393758	458.347403497436\\
0.292857142857143	457.961860102601\\
0.293253301320528	457.577074553935\\
0.293649459783914	457.193044436201\\
0.294045618247299	456.809767344859\\
0.294441776710684	456.427240886001\\
0.29483793517407	456.045462676295\\
0.295234093637455	455.664430342922\\
0.29563025210084	455.284141523517\\
0.296026410564226	454.90459386611\\
0.296422569027611	454.525785029067\\
0.296818727490996	454.147712681035\\
0.297214885954382	453.770374500878\\
0.297611044417767	453.393768177624\\
0.298007202881152	453.017891410406\\
0.298403361344538	452.642741908407\\
0.298799519807923	452.268317390802\\
0.299195678271309	451.894615586703\\
0.299591836734694	451.521634235102\\
0.299987995198079	451.149371084817\\
0.300384153661465	450.777823894439\\
0.30078031212485	450.406990432273\\
0.301176470588235	450.036868476288\\
0.301572629051621	449.667455814063\\
0.301968787515006	449.298750242729\\
0.302364945978391	448.930749568924\\
0.302761104441777	448.563451608733\\
0.303157262905162	448.19685418764\\
0.303553421368547	447.830955140473\\
0.303949579831933	447.465752311358\\
0.304345738295318	447.10124355366\\
0.304741896758703	446.73742672994\\
0.305138055222089	446.374299711897\\
0.305534213685474	446.011860380324\\
0.30593037214886	445.650106625055\\
0.306326530612245	445.289036344918\\
0.30672268907563	444.928647447681\\
0.307118847539016	444.568937850008\\
0.307515006002401	444.209905477409\\
0.307911164465786	443.85154826419\\
0.308307322929172	443.493864153409\\
0.308703481392557	443.136851096824\\
0.309099639855942	442.780507054848\\
0.309495798319328	442.424829996504\\
0.309891956782713	442.069817899375\\
0.310288115246098	441.715468749557\\
0.310684273709484	441.361780541619\\
0.311080432172869	441.00875127855\\
0.311476590636254	440.65637897172\\
0.31187274909964	440.304661640829\\
0.312268907563025	439.953597313867\\
0.312665066026411	439.603184027067\\
0.313061224489796	439.253419824862\\
0.313457382953181	438.904302759838\\
0.313853541416567	438.555830892696\\
0.314249699879952	438.208002292203\\
0.314645858343337	437.860815035154\\
0.315042016806723	437.514267206325\\
0.315438175270108	437.168356898432\\
0.315834333733493	436.823082212089\\
0.316230492196879	436.478441255767\\
0.316626650660264	436.134432145751\\
0.31702280912365	435.791053006097\\
0.317418967587035	435.448301968596\\
0.31781512605042	435.106177172727\\
0.318211284513806	434.76467676562\\
0.318607442977191	434.423798902017\\
0.319003601440576	434.083541744228\\
0.319399759903962	433.743903462092\\
0.319795918367347	433.404882232941\\
0.320192076830732	433.066476241555\\
0.320588235294118	432.72868368013\\
0.320984393757503	432.391502748232\\
0.321380552220888	432.054931652764\\
0.321776710684274	431.718968607922\\
0.322172869147659	431.383611835165\\
0.322569027611044	431.048859563168\\
0.32296518607443	430.714710027792\\
0.323361344537815	430.381161472043\\
0.3237575030012	430.048212146032\\
0.324153661464586	429.715860306948\\
0.324549819927971	429.384104219008\\
0.324945978391357	429.052942153432\\
0.325342136854742	428.722372388402\\
0.325738295318127	428.392393209023\\
0.326134453781513	428.063002907294\\
0.326530612244898	427.734199782069\\
0.326926770708283	427.405982139019\\
0.327322929171669	427.078348290603\\
0.327719087635054	426.751296556027\\
0.328115246098439	426.424825261216\\
0.328511404561825	426.098932738772\\
0.32890756302521	425.773617327946\\
0.329303721488595	425.448877374602\\
0.329699879951981	425.124711231182\\
0.330096038415366	424.801117256673\\
0.330492196878751	424.478093816575\\
0.330888355342137	424.155639282868\\
0.331284513805522	423.833752033976\\
0.331680672268908	423.512430454738\\
0.332076830732293	423.191672936372\\
0.332472989195678	422.871477876447\\
0.332869147659064	422.551843678846\\
0.333265306122449	422.232768753737\\
0.333661464585834	421.914251517544\\
0.33405762304922	421.596290392906\\
0.334453781512605	421.278883808658\\
0.33484993997599	420.96203019979\\
0.335246098439376	420.645728007422\\
0.335642256902761	420.329975678769\\
0.336038415366146	420.014771667114\\
0.336434573829532	419.700114431775\\
0.336830732292917	419.386002438078\\
0.337226890756303	419.072434157323\\
0.337623049219688	418.759408066756\\
0.338019207683073	418.44692264954\\
0.338415366146459	418.134976394726\\
0.338811524609844	417.823567797221\\
0.339207683073229	417.512695357762\\
0.339603841536615	417.202357582884\\
0.34	416.892552984894\\
0.340396158463385	416.583280081842\\
0.340792316926771	416.27453739749\\
0.341188475390156	415.966323461288\\
0.341584633853541	415.658636808341\\
0.341980792316927	415.351475979387\\
0.342376950780312	415.044839520762\\
0.342773109243697	414.738725984379\\
0.343169267707083	414.433133927699\\
0.343565426170468	414.1280619137\\
0.343961584633854	413.823508510854\\
0.344357743097239	413.519472293099\\
0.344753901560624	413.215951839813\\
0.34515006002401	412.912945735784\\
0.345546218487395	412.610452571189\\
0.34594237695078	412.308470941562\\
0.346338535414166	412.006999447773\\
0.346734693877551	411.706036695998\\
0.347130852340936	411.405581297697\\
0.347527010804322	411.105631869584\\
0.347923169267707	410.806187033604\\
0.348319327731092	410.50724541691\\
0.348715486194478	410.208805651832\\
0.349111644657863	409.910866375858\\
0.349507803121248	409.613426231604\\
0.349903961584634	409.316483866793\\
0.350300120048019	409.020037934228\\
0.350696278511405	408.724087091769\\
0.35109243697479	408.428630002309\\
0.351488595438175	408.133665333747\\
0.351884753901561	407.839191758968\\
0.352280912364946	407.545207955816\\
0.352677070828331	407.251712607071\\
0.353073229291717	406.958704400426\\
0.353469387755102	406.666182028464\\
0.353865546218487	406.374144188632\\
0.354261704681873	406.082589583223\\
0.354657863145258	405.791516919347\\
0.355054021608643	405.50092490891\\
0.355450180072029	405.210812268595\\
0.355846338535414	404.921177719833\\
0.3562424969988	404.632019988786\\
0.356638655462185	404.34333780632\\
0.35703481392557	404.055129907988\\
0.357430972388956	403.767395034002\\
0.357827130852341	403.480131929216\\
0.358223289315726	403.193339343101\\
0.358619447779112	402.907016029724\\
0.359015606242497	402.621160747727\\
0.359411764705882	402.335772260304\\
0.359807923169268	402.050849335182\\
0.360204081632653	401.766390744598\\
0.360600240096038	401.482395265278\\
0.360996398559424	401.198861678416\\
0.361392557022809	400.915788769654\\
0.361788715486195	400.63317532906\\
0.36218487394958	400.351020151107\\
0.362581032412965	400.069322034655\\
0.362977190876351	399.788079782928\\
0.363373349339736	399.507292203494\\
0.363769507803121	399.226958108247\\
0.364165666266507	398.947076313383\\
0.364561824729892	398.667645639384\\
0.364957983193277	398.388664910996\\
0.365354141656663	398.110132957209\\
0.365750300120048	397.832048611239\\
0.366146458583433	397.554410710506\\
0.366542617046819	397.277218096618\\
0.366938775510204	397.00046961535\\
0.367334933973589	396.724164116622\\
0.367731092436975	396.448300454484\\
0.36812725090036	396.172877487096\\
0.368523409363746	395.897894076709\\
0.368919567827131	395.623349089645\\
0.369315726290516	395.349241396277\\
0.369711884753902	395.075569871018\\
0.370108043217287	394.802333392292\\
0.370504201680672	394.529530842525\\
0.370900360144058	394.257161108119\\
0.371296518607443	393.985223079439\\
0.371692677070828	393.713715650795\\
0.372088835534214	393.442637720421\\
0.372484993997599	393.171988190456\\
0.372881152460984	392.901765966934\\
0.37327731092437	392.631969959758\\
0.373673469387755	392.362599082686\\
0.37406962785114	392.093652253314\\
0.374465786314526	391.825128393057\\
0.374861944777911	391.557026427133\\
0.375258103241296	391.289345284546\\
0.375654261704682	391.022083898066\\
0.376050420168067	390.755241204219\\
0.376446578631453	390.48881614326\\
0.376842737094838	390.222807659166\\
0.377238895558223	389.957214699613\\
0.377635054021609	389.692036215961\\
0.378031212484994	389.427271163239\\
0.378427370948379	389.162918500127\\
0.378823529411765	388.89897718894\\
0.37921968787515	388.635446195612\\
0.379615846338535	388.37232448968\\
0.380012004801921	388.109611044268\\
0.380408163265306	387.847304836068\\
0.380804321728692	387.58540484533\\
0.381200480192077	387.323910055842\\
0.381596638655462	387.062819454913\\
0.381992797118848	386.802132033364\\
0.382388955582233	386.541846785504\\
0.382785114045618	386.281962709119\\
0.383181272509004	386.022478805459\\
0.383577430972389	385.763394079217\\
0.383973589435774	385.504707538519\\
0.38436974789916	385.246418194903\\
0.384765906362545	384.988525063311\\
0.38516206482593	384.731027162068\\
0.385558223289316	384.473923512872\\
0.385954381752701	384.217213140774\\
0.386350540216086	383.960895074168\\
0.386746698679472	383.704968344774\\
0.387142857142857	383.449431987624\\
0.387539015606243	383.194285041045\\
0.387935174069628	382.93952654665\\
0.388331332533013	382.685155549318\\
0.388727490996399	382.431171097185\\
0.389123649459784	382.177572241625\\
0.389519807923169	381.924358037237\\
0.389915966386555	381.671527541835\\
0.39031212484994	381.419079816428\\
0.390708283313325	381.167013925211\\
0.391104441776711	380.915328935547\\
0.391500600240096	380.664023917959\\
0.391896758703481	380.413097946108\\
0.392292917166867	380.162550096789\\
0.392689075630252	379.912379449909\\
0.393085234093637	379.66258508848\\
0.393481392557023	379.413166098599\\
0.393877551020408	379.164121569443\\
0.394273709483794	378.915450593249\\
0.394669867947179	378.667152265302\\
0.395066026410564	378.419225683925\\
0.39546218487395	378.171669950463\\
0.395858343337335	377.92448416927\\
0.39625450180072	377.677667447699\\
0.396650660264106	377.431218896086\\
0.397046818727491	377.185137627737\\
0.397442977190876	376.93942275892\\
0.397839135654262	376.694073408848\\
0.398235294117647	376.449088699665\\
0.398631452581032	376.204467756438\\
0.399027611044418	375.960209707144\\
0.399423769507803	375.716313682653\\
0.399819927971188	375.472778816722\\
0.400216086434574	375.229604245977\\
0.400612244897959	374.986789109904\\
0.401008403361345	374.744332550839\\
0.40140456182473	374.50223371395\\
0.401800720288115	374.260491747229\\
0.402196878751501	374.019105801481\\
0.402593037214886	373.778075030308\\
0.402989195678271	373.537398590102\\
0.403385354141657	373.297075640029\\
0.403781512605042	373.05710534202\\
0.404177671068427	372.81748686076\\
0.404573829531813	372.578219363672\\
0.404969987995198	372.339302020911\\
0.405366146458583	372.100734005349\\
0.405762304921969	371.862514492565\\
0.406158463385354	371.624642660833\\
0.40655462184874	371.387117691111\\
0.406950780312125	371.149938767031\\
0.40734693877551	370.913105074885\\
0.407743097238896	370.676615803616\\
0.408139255702281	370.440470144808\\
0.408535414165666	370.204667292672\\
0.408931572629052	369.969206444036\\
0.409327731092437	369.734086798336\\
0.409723889555822	369.499307557604\\
0.410120048019208	369.264867926456\\
0.410516206482593	369.030767112083\\
0.410912364945978	368.797004324241\\
0.411308523409364	368.563578775236\\
0.411704681872749	368.33048967992\\
0.412100840336134	368.097736255674\\
0.41249699879952	367.865317722403\\
0.412893157262905	367.633233302522\\
0.413289315726291	367.401482220948\\
0.413685474189676	367.170063705086\\
0.414081632653061	366.938976984825\\
0.414477791116447	366.70822129252\\
0.414873949579832	366.47779586299\\
0.415270108043217	366.247699933502\\
0.415666266506603	366.017932743763\\
0.416062424969988	365.788493535909\\
0.416458583433373	365.559381554498\\
0.416854741896759	365.330596046496\\
0.417250900360144	365.102136261271\\
0.417647058823529	364.874001450581\\
0.418043217286915	364.646190868563\\
0.4184393757503	364.418703771728\\
0.418835534213685	364.191539418946\\
0.419231692677071	363.964697071439\\
0.419627851140456	363.738175992773\\
0.420024009603842	363.511975448845\\
0.420420168067227	363.286094707875\\
0.420816326530612	363.060533040399\\
0.421212484993998	362.835289719255\\
0.421608643457383	362.610364019578\\
0.422004801920768	362.385755218789\\
0.422400960384154	362.161462596583\\
0.422797118847539	361.937485434926\\
0.423193277310924	361.713823018041\\
0.42358943577431	361.4904746324\\
0.423985594237695	361.267439566715\\
0.42438175270108	361.04471711193\\
0.424777911164466	360.82230656121\\
0.425174069627851	360.600207209937\\
0.425570228091237	360.378418355693\\
0.425966386554622	360.156939298257\\
0.426362545018007	359.935769339598\\
0.426758703481393	359.714907783859\\
0.427154861944778	359.494353937354\\
0.427551020408163	359.274107108559\\
0.427947178871549	359.054166608102\\
0.428343337334934	358.834531748752\\
0.428739495798319	358.615201845417\\
0.429135654261705	358.396176215128\\
0.42953181272509	358.177454177038\\
0.429927971188475	357.959035052406\\
0.430324129651861	357.740918164595\\
0.430720288115246	357.52310283906\\
0.431116446578631	357.305588403342\\
0.431512605042017	357.088374187056\\
0.431908763505402	356.871459521888\\
0.432304921968788	356.654843741584\\
0.432701080432173	356.43852618194\\
0.433097238895558	356.222506180798\\
0.433493397358944	356.006783078036\\
0.433889555822329	355.791356215558\\
0.434285714285714	355.57622493729\\
0.4346818727491	355.36138858917\\
0.435078031212485	355.146846519139\\
0.43547418967587	354.932598077133\\
0.435870348139256	354.718642615081\\
0.436266506602641	354.504979486889\\
0.436662665066026	354.291608048437\\
0.437058823529412	354.078527657569\\
0.437454981992797	353.865737674089\\
0.437851140456183	353.65323745975\\
0.438247298919568	353.441026378246\\
0.438643457382953	353.229103795208\\
0.439039615846339	353.017469078192\\
0.439435774309724	352.806121596676\\
0.439831932773109	352.595060722049\\
0.440228091236495	352.384285827604\\
0.44062424969988	352.173796288535\\
0.441020408163265	351.963591481921\\
0.441416566626651	351.753670786729\\
0.441812725090036	351.544033583798\\
0.442208883553421	351.334679255838\\
0.442605042016807	351.125607187417\\
0.443001200480192	350.91681676496\\
0.443397358943577	350.708307376736\\
0.443793517406963	350.500078412857\\
0.444189675870348	350.292129265266\\
0.444585834333734	350.08445932773\\
0.444981992797119	349.877067995836\\
0.445378151260504	349.669954666983\\
0.44577430972389	349.463118740374\\
0.446170468187275	349.256559617011\\
0.44656662665066	349.050276699684\\
0.446962785114046	348.844269392971\\
0.447358943577431	348.638537103222\\
0.447755102040816	348.433079238563\\
0.448151260504202	348.22789520888\\
0.448547418967587	348.022984425817\\
0.448943577430972	347.818346302769\\
0.449339735894358	347.613980254872\\
0.449735894357743	347.409885699002\\
0.450132052821129	347.206062053763\\
0.450528211284514	347.002508739486\\
0.450924369747899	346.799225178215\\
0.451320528211284	346.596210793708\\
0.45171668667467	346.393465011426\\
0.452112845138055	346.190987258527\\
0.452509003601441	345.988776963863\\
0.452905162064826	345.786833557968\\
0.453301320528211	345.585156473057\\
0.453697478991597	345.383745143016\\
0.454093637454982	345.182599003397\\
0.454489795918367	344.981717491413\\
0.454885954381753	344.781100045929\\
0.455282112845138	344.580746107459\\
0.455678271308523	344.380655118156\\
0.456074429771909	344.18082652181\\
0.456470588235294	343.981259763838\\
0.456866746698679	343.781954291282\\
0.457262905162065	343.582909552799\\
0.45765906362545	343.384124998656\\
0.458055222088836	343.185600080726\\
0.458451380552221	342.987334252481\\
0.458847539015606	342.789326968985\\
0.459243697478992	342.591577686888\\
0.459639855942377	342.394085864421\\
0.460036014405762	342.196850961391\\
0.460432172869148	341.999872439174\\
0.460828331332533	341.803149760707\\
0.461224489795918	341.606682390488\\
0.461620648259304	341.410469794563\\
0.462016806722689	341.214511440528\\
0.462412965186074	341.018806797515\\
0.46280912364946	340.823355336193\\
0.463205282112845	340.628156528759\\
0.46360144057623	340.433209848935\\
0.463997599039616	340.238514771957\\
0.464393757503001	340.044070774576\\
0.464789915966387	339.849877335048\\
0.465186074429772	339.655933933131\\
0.465582232893157	339.462240050077\\
0.465978391356543	339.268795168628\\
0.466374549819928	339.075598773011\\
0.466770708283313	338.882650348932\\
0.467166866746699	338.689949383571\\
0.467563025210084	338.497495365574\\
0.467959183673469	338.305287785051\\
0.468355342136855	338.11332613357\\
0.46875150060024	337.921609904149\\
0.469147659063625	337.730138591257\\
0.469543817527011	337.538911690799\\
0.469939975990396	337.34792870012\\
0.470336134453781	337.157189117995\\
0.470732292917167	336.966692444624\\
0.471128451380552	336.776438181628\\
0.471524609843938	336.586425832044\\
0.471920768307323	336.396654900319\\
0.472316926770708	336.207124892305\\
0.472713085234094	336.017835315254\\
0.473109243697479	335.828785677813\\
0.473505402160864	335.639975490019\\
0.47390156062425	335.451404263294\\
0.474297719087635	335.26307151044\\
0.47469387755102	335.074976745635\\
0.475090036014406	334.887119484423\\
0.475486194477791	334.699499243718\\
0.475882352941176	334.512115541792\\
0.476278511404562	334.324967898271\\
0.476674669867947	334.138055834134\\
0.477070828331333	333.951378871702\\
0.477466986794718	333.76493653464\\
0.477863145258103	333.578728347947\\
0.478259303721489	333.392753837954\\
0.478655462184874	333.207012532318\\
0.479051620648259	333.021503960016\\
0.479447779111645	332.836227651345\\
0.47984393757503	332.651183137912\\
0.480240096038415	332.466369952631\\
0.480636254501801	332.281787629719\\
0.481032412965186	332.097435704693\\
0.481428571428571	331.913313714361\\
0.481824729891957	331.729421196822\\
0.482220888355342	331.545757691457\\
0.482617046818727	331.362322738929\\
0.483013205282113	331.179115881175\\
0.483409363745498	330.996136661403\\
0.483805522208884	330.813384624086\\
0.484201680672269	330.630859314961\\
0.484597839135654	330.448560281019\\
0.48499399759904	330.266487070508\\
0.485390156062425	330.08463923292\\
0.48578631452581	329.903016318993\\
0.486182472989196	329.721617880706\\
0.486578631452581	329.54044347127\\
0.486974789915966	329.359492645128\\
0.487370948379352	329.178764957951\\
0.487767106842737	328.99825996663\\
0.488163265306122	328.817977229275\\
0.488559423769508	328.637916305209\\
0.488955582232893	328.458076754965\\
0.489351740696279	328.278458140281\\
0.489747899159664	328.099060024095\\
0.490144057623049	327.919881970542\\
0.490540216086435	327.74092354495\\
0.49093637454982	327.562184313834\\
0.491332533013205	327.383663844896\\
0.491728691476591	327.205361707014\\
0.492124849939976	327.027277470244\\
0.492521008403361	326.849410705815\\
0.492917166866747	326.671760986121\\
0.493313325330132	326.494327884721\\
0.493709483793517	326.317110976334\\
0.494105642256903	326.140109836835\\
0.494501800720288	325.963324043247\\
0.494897959183673	325.786753173745\\
0.495294117647059	325.610396807646\\
0.495690276110444	325.434254525404\\
0.49608643457383	325.258325908614\\
0.496482593037215	325.082610539997\\
0.4968787515006	324.907108003405\\
0.497274909963986	324.731817883814\\
0.497671068427371	324.55673976732\\
0.498067226890756	324.381873241132\\
0.498463385354142	324.207217893576\\
0.498859543817527	324.032773314084\\
0.499255702280912	323.858539093192\\
0.499651860744298	323.684514822539\\
0.500048019207683	323.510700094859\\
0.500444177671068	323.337094503981\\
0.500840336134454	323.163697644822\\
0.501236494597839	322.990509113385\\
0.501632653061224	322.817528506758\\
0.50202881152461	322.644755423103\\
0.502424969987995	322.47218946166\\
0.50282112845138	322.299830222739\\
0.503217286914766	322.127677307716\\
0.503613445378151	321.955730319033\\
0.504009603841537	321.783988860191\\
0.504405762304922	321.612452535748\\
0.504801920768307	321.441120951313\\
0.505198079231693	321.269993713548\\
0.505594237695078	321.099070430157\\
0.505990396158463	320.928350709887\\
0.506386554621849	320.757834162527\\
0.506782713085234	320.587520398896\\
0.507178871548619	320.417409030849\\
0.507575030012005	320.247499671266\\
0.50797118847539	320.077791934054\\
0.508367346938776	319.908285434139\\
0.508763505402161	319.738979787467\\
0.509159663865546	319.569874610995\\
0.509555822328932	319.400969522695\\
0.509951980792317	319.232264141545\\
0.510348139255702	319.063758087524\\
0.510744297719088	318.895450981615\\
0.511140456182473	318.727342445798\\
0.511536614645858	318.559432103047\\
0.511932773109244	318.391719577324\\
0.512328931572629	318.224204493582\\
0.512725090036014	318.056886477756\\
0.5131212484994	317.889765156761\\
0.513517406962785	317.722840158491\\
0.513913565426171	317.556111111814\\
0.514309723889556	317.389577646568\\
0.514705882352941	317.223239393557\\
0.515102040816326	317.057095984554\\
0.515498199279712	316.891147052289\\
0.515894357743097	316.725392230451\\
0.516290516206483	316.559831153685\\
0.516686674669868	316.394463457587\\
0.517082833133253	316.229288778699\\
0.517478991596639	316.064306754513\\
0.517875150060024	315.899517023459\\
0.518271308523409	315.734919224907\\
0.518667466986795	315.570512999165\\
0.51906362545018	315.40629798747\\
0.519459783913566	315.242273831992\\
0.519855942376951	315.078440175827\\
0.520252100840336	314.914796662992\\
0.520648259303721	314.751342938427\\
0.521044417767107	314.58807864799\\
0.521440576230492	314.42500343845\\
0.521836734693877	314.262116957491\\
0.522232893157263	314.099418853703\\
0.522629051620648	313.936908776583\\
0.523025210084034	313.774586376529\\
0.523421368547419	313.61245130484\\
0.523817527010804	313.45050321371\\
0.52421368547419	313.288741756227\\
0.524609843937575	313.127166586369\\
0.52500600240096	312.965777359003\\
0.525402160864346	312.804573729881\\
0.525798319327731	312.643555355635\\
0.526194477791117	312.482721893777\\
0.526590636254502	312.322073002695\\
0.526986794717887	312.16160834165\\
0.527382953181272	312.001327570774\\
0.527779111644658	311.841230351067\\
0.528175270108043	311.681316344392\\
0.528571428571429	311.521585213474\\
0.528967587034814	311.3620366219\\
0.529363745498199	311.202670234109\\
0.529759903961585	311.043485715398\\
0.53015606242497	310.88448273191\\
0.530552220888355	310.725660950639\\
0.530948379351741	310.567020039425\\
0.531344537815126	310.408559666947\\
0.531740696278511	310.250279502727\\
0.532136854741897	310.092179217123\\
0.532533013205282	309.934258481326\\
0.532929171668667	309.776516967361\\
0.533325330132053	309.61895434808\\
0.533721488595438	309.461570297162\\
0.534117647058824	309.304364489109\\
0.534513805522209	309.147336599245\\
0.534909963985594	308.990486303711\\
0.53530612244898	308.833813279465\\
0.535702280912365	308.677317204277\\
0.53609843937575	308.520997756728\\
0.536494597839136	308.364854616206\\
0.536890756302521	308.208887462905\\
0.537286914765906	308.053095977821\\
0.537683073229292	307.897479842751\\
0.538079231692677	307.742038740288\\
0.538475390156062	307.586772353822\\
0.538871548619448	307.431680367533\\
0.539267707082833	307.276762466393\\
0.539663865546218	307.122018336159\\
0.540060024009604	306.967447663375\\
0.540456182472989	306.813050135367\\
0.540852340936375	306.658825440239\\
0.54124849939976	306.504773266874\\
0.541644657863145	306.35089330493\\
0.542040816326531	306.197185244836\\
0.542436974789916	306.043648777793\\
0.542833133253301	305.890283595766\\
0.543229291716687	305.737089391489\\
0.543625450180072	305.584065858456\\
0.544021608643457	305.431212690923\\
0.544417767106843	305.278529583901\\
0.544813925570228	305.12601623316\\
0.545210084033613	304.973672335221\\
0.545606242496999	304.821497587355\\
0.546002400960384	304.669491687582\\
0.54639855942377	304.517654334668\\
0.546794717887155	304.365985228124\\
0.54719087635054	304.214484068198\\
0.547587034813926	304.063150555882\\
0.547983193277311	303.9119843929\\
0.548379351740696	303.760985281713\\
0.548775510204082	303.610152925512\\
0.549171668667467	303.459487028221\\
0.549567827130852	303.308987294486\\
0.549963985594238	303.158653429684\\
0.550360144057623	303.008485139909\\
0.550756302521008	302.858482131979\\
0.551152460984394	302.70864411343\\
0.551548619447779	302.558970792512\\
0.551944777911164	302.409461878192\\
0.55234093637455	302.260117080144\\
0.552737094837935	302.110936108756\\
0.553133253301321	301.961918675119\\
0.553529411764706	301.813064491033\\
0.553925570228091	301.664373268996\\
0.554321728691477	301.51584472221\\
0.554717887154862	301.367478564574\\
0.555114045618247	301.219274510683\\
0.555510204081633	301.071232275826\\
0.555906362545018	300.923351575985\\
0.556302521008403	300.775632127829\\
0.556698679471789	300.628073648717\\
0.557094837935174	300.480675856692\\
0.557490996398559	300.333438470482\\
0.557887154861945	300.186361209493\\
0.55828331332533	300.039443793813\\
0.558679471788716	299.892685944205\\
0.559075630252101	299.746087382109\\
0.559471788715486	299.599647829635\\
0.559867947178872	299.453367009566\\
0.560264105642257	299.307244645352\\
0.560660264105642	299.16128046111\\
0.561056422569028	299.015474181622\\
0.561452581032413	298.869825532332\\
0.561848739495798	298.724334239346\\
0.562244897959184	298.579000029424\\
0.562641056422569	298.433822629988\\
0.563037214885954	298.28880176911\\
0.56343337334934	298.143937175516\\
0.563829531812725	297.999228578583\\
0.56422569027611	297.854675708336\\
0.564621848739496	297.710278295445\\
0.565018007202881	297.566036071225\\
0.565414165666267	297.421948767635\\
0.565810324129652	297.278016117273\\
0.566206482593037	297.134237853375\\
0.566602641056423	296.990613709814\\
0.566998799519808	296.847143421099\\
0.567394957983193	296.70382672237\\
0.567791116446579	296.560663349398\\
0.568187274909964	296.417653038584\\
0.568583433373349	296.274795526954\\
0.568979591836735	296.13209055216\\
0.56937575030012	295.989537852478\\
0.569771908763505	295.847137166804\\
0.570168067226891	295.704888234654\\
0.570564225690276	295.56279079616\\
0.570960384153661	295.420844592071\\
0.571356542617047	295.279049363751\\
0.571752701080432	295.137404853173\\
0.572148859543818	294.995910802921\\
0.572545018007203	294.854566956188\\
0.572941176470588	294.713373056773\\
0.573337334933974	294.57232884908\\
0.573733493397359	294.431434078114\\
0.574129651860744	294.290688489481\\
0.57452581032413	294.15009182939\\
0.574921968787515	294.009643844641\\
0.5753181272509	293.869344282635\\
0.575714285714286	293.729192891363\\
0.576110444177671	293.58918941941\\
0.576506602641056	293.449333615951\\
0.576902761104442	293.309625230747\\
0.577298919567827	293.170064014149\\
0.577695078031213	293.03064971709\\
0.578091236494598	292.891382091087\\
0.578487394957983	292.75226088824\\
0.578883553421369	292.613285861226\\
0.579279711884754	292.474456763301\\
0.579675870348139	292.335773348298\\
0.580072028811525	292.197235370623\\
0.58046818727491	292.058842585256\\
0.580864345738295	291.920594747746\\
0.581260504201681	291.782491614213\\
0.581656662665066	291.644532941344\\
0.582052821128451	291.506718486393\\
0.582448979591837	291.369048007176\\
0.582845138055222	291.231521262075\\
0.583241296518607	291.094138010028\\
0.583637454981993	290.956898010537\\
0.584033613445378	290.81980102366\\
0.584429771908764	290.682846810008\\
0.584825930372149	290.546035130751\\
0.585222088835534	290.409365747609\\
0.58561824729892	290.272838422852\\
0.586014405762305	290.136452919303\\
0.58641056422569	290.000209000329\\
0.586806722689076	289.864106429844\\
0.587202881152461	289.72814497231\\
0.587599039615846	289.592324392726\\
0.587995198079232	289.456644456637\\
0.588391356542617	289.321104930126\\
0.588787515006002	289.185705579813\\
0.589183673469388	289.050446172858\\
0.589579831932773	288.915326476952\\
0.589975990396158	288.780346260323\\
0.590372148859544	288.645505291727\\
0.590768307322929	288.510803340454\\
0.591164465786315	288.37624017632\\
0.5915606242497	288.24181556967\\
0.591956782713085	288.107529291373\\
0.592352941176471	287.973381112824\\
0.592749099639856	287.839370805939\\
0.593145258103241	287.705498143156\\
0.593541416566627	287.571762897432\\
0.593937575030012	287.438164842242\\
0.594333733493397	287.304703751578\\
0.594729891956783	287.171379399946\\
0.595126050420168	287.038191562367\\
0.595522208883553	286.905140014372\\
0.595918367346939	286.772224532005\\
0.596314525810324	286.639444891817\\
0.59671068427371	286.506800870867\\
0.597106842737095	286.374292246722\\
0.59750300120048	286.241918797449\\
0.597899159663866	286.109680301624\\
0.598295318127251	285.97757653832\\
0.598691476590636	285.845607287112\\
0.599087635054022	285.713772328075\\
0.599483793517407	285.58207144178\\
0.599879951980792	285.450504409293\\
0.600276110444178	285.319071012176\\
0.600672268907563	285.187771032484\\
0.601068427370948	285.056604252763\\
0.601464585834334	284.925570456049\\
0.601860744297719	284.794669425868\\
0.602256902761104	284.663900946233\\
0.60265306122449	284.533264801642\\
0.603049219687875	284.402760777078\\
0.603445378151261	284.272388658007\\
0.603841536614646	284.142148230379\\
0.604237695078031	284.012039280622\\
0.604633853541417	283.882061595643\\
0.605030012004802	283.752214962828\\
0.605426170468187	283.622499170039\\
0.605822328931573	283.492914005612\\
0.606218487394958	283.363459258358\\
0.606614645858343	283.23413471756\\
0.607010804321729	283.10494017297\\
0.607406962785114	282.975875414812\\
0.607803121248499	282.846940233778\\
0.608199279711885	282.718134421024\\
0.60859543817527	282.589457768175\\
0.608991596638655	282.460910067319\\
0.609387755102041	282.332491111007\\
0.609783913565426	282.204200692249\\
0.610180072028812	282.076038604521\\
0.610576230492197	281.948004641752\\
0.610972388955582	281.820098598331\\
0.611368547418968	281.692320269106\\
0.611764705882353	281.564669449376\\
0.612160864345738	281.437145934896\\
0.612557022809124	281.309749521872\\
0.612953181272509	281.182480006962\\
0.613349339735894	281.055337187275\\
0.61374549819928	280.928320860368\\
0.614141656662665	280.801430824244\\
0.61453781512605	280.674666877354\\
0.614933973589436	280.548028818592\\
0.615330132052821	280.421516447298\\
0.615726290516207	280.295129563252\\
0.616122448979592	280.168867966677\\
0.616518607442977	280.042731458235\\
0.616914765906363	279.916719839026\\
0.617310924369748	279.790832910588\\
0.617707082833133	279.665070474897\\
0.618103241296519	279.539432334361\\
0.618499399759904	279.413918291822\\
0.618895558223289	279.288528150558\\
0.619291716686675	279.163261714274\\
0.61968787515006	279.038118787107\\
0.620084033613445	278.913099173623\\
0.620480192076831	278.788202678816\\
0.620876350540216	278.663429108106\\
0.621272509003601	278.538778267337\\
0.621668667466987	278.41424996278\\
0.622064825930372	278.289844001126\\
0.622460984393758	278.165560189491\\
0.622857142857143	278.041398335407\\
0.623253301320528	277.917358246831\\
0.623649459783914	277.793439732133\\
0.624045618247299	277.669642600104\\
0.624441776710684	277.545966659949\\
0.62483793517407	277.422411721287\\
0.625234093637455	277.298977594153\\
0.62563025210084	277.175664088992\\
0.626026410564226	277.052471016663\\
0.626422569027611	276.929398188433\\
0.626818727490996	276.806445415978\\
0.627214885954382	276.683612511383\\
0.627611044417767	276.56089928714\\
0.628007202881153	276.438305556146\\
0.628403361344538	276.315831131702\\
0.628799519807923	276.193475827513\\
0.629195678271309	276.071239457686\\
0.629591836734694	275.949121836731\\
0.629987995198079	275.827122779556\\
0.630384153661465	275.705242101467\\
0.63078031212485	275.583479618172\\
0.631176470588235	275.461835145771\\
0.631572629051621	275.340308500763\\
0.631968787515006	275.218899500041\\
0.632364945978391	275.097607960891\\
0.632761104441777	274.97643370099\\
0.633157262905162	274.85537653841\\
0.633553421368547	274.73443629161\\
0.633949579831933	274.613612779439\\
0.634345738295318	274.492905821136\\
0.634741896758704	274.372315236326\\
0.635138055222089	274.251840845017\\
0.635534213685474	274.131482467608\\
0.63593037214886	274.011239924876\\
0.636326530612245	273.891113037985\\
0.63672268907563	273.77110162848\\
0.637118847539016	273.651205518284\\
0.637515006002401	273.531424529704\\
0.637911164465786	273.411758485423\\
0.638307322929172	273.292207208502\\
0.638703481392557	273.17277052238\\
0.639099639855942	273.053448250869\\
0.639495798319328	272.934240218159\\
0.639891956782713	272.815146248811\\
0.640288115246098	272.69616616776\\
0.640684273709484	272.577299800311\\
0.641080432172869	272.458546972142\\
0.641476590636255	272.339907509299\\
0.64187274909964	272.221381238195\\
0.642268907563025	272.102967985615\\
0.642665066026411	271.984667578706\\
0.643061224489796	271.866479844984\\
0.643457382953181	271.748404612327\\
0.643853541416567	271.630441708978\\
0.644249699879952	271.512590963542\\
0.644645858343337	271.394852204986\\
0.645042016806723	271.277225262639\\
0.645438175270108	271.159709966186\\
0.645834333733493	271.042306145676\\
0.646230492196879	270.92501363151\\
0.646626650660264	270.807832254451\\
0.64702280912365	270.690761845614\\
0.647418967587035	270.573802236472\\
0.64781512605042	270.456953258848\\
0.648211284513806	270.340214744923\\
0.648607442977191	270.223586527226\\
0.649003601440576	270.107068438638\\
0.649399759903962	269.990660312392\\
0.649795918367347	269.874361982069\\
0.650192076830732	269.758173281596\\
0.650588235294118	269.642094045252\\
0.650984393757503	269.52612410766\\
0.651380552220888	269.410263303786\\
0.651776710684274	269.294511468946\\
0.652172869147659	269.178868438795\\
0.652569027611044	269.063334049332\\
0.65296518607443	268.947908136901\\
0.653361344537815	268.832590538181\\
0.653757503001201	268.717381090197\\
0.654153661464586	268.602279630309\\
0.654549819927971	268.487285996216\\
0.654945978391357	268.372400025957\\
0.655342136854742	268.257621557903\\
0.655738295318127	268.142950430765\\
0.656134453781513	268.028386483585\\
0.656530612244898	267.913929555741\\
0.656926770708283	267.799579486942\\
0.657322929171669	267.685336117232\\
0.657719087635054	267.571199286983\\
0.658115246098439	267.457168836899\\
0.658511404561825	267.343244608012\\
0.65890756302521	267.229426441683\\
0.659303721488595	267.115714179603\\
0.659699879951981	267.002107663786\\
0.660096038415366	266.888606736573\\
0.660492196878751	266.775211240633\\
0.660888355342137	266.661921018954\\
0.661284513805522	266.548735914852\\
0.661680672268908	266.435655771963\\
0.662076830732293	266.322680434245\\
0.662472989195678	266.209809745978\\
0.662869147659064	266.097043551759\\
0.663265306122449	265.984381696508\\
0.663661464585834	265.871824025461\\
0.66405762304922	265.75937038417\\
0.664453781512605	265.647020618507\\
0.66484993997599	265.534774574658\\
0.665246098439376	265.422632099124\\
0.665642256902761	265.31059303872\\
0.666038415366147	265.198657240574\\
0.666434573829532	265.086824552127\\
0.666830732292917	264.975094821133\\
0.667226890756303	264.863467895654\\
0.667623049219688	264.751943624064\\
0.668019207683073	264.640521855045\\
0.668415366146459	264.52920243759\\
0.668811524609844	264.417985220996\\
0.669207683073229	264.306870054869\\
0.669603841536615	264.195856789122\\
0.67	264.08494527397\\
0.670396158463385	263.974135359935\\
0.670792316926771	263.863426897842\\
0.671188475390156	263.752819738819\\
0.671584633853541	263.642313734296\\
0.671980792316927	263.531908736003\\
0.672376950780312	263.421604595974\\
0.672773109243697	263.311401166538\\
0.673169267707083	263.201298300326\\
0.673565426170468	263.091295850268\\
0.673961584633854	262.981393669588\\
0.674357743097239	262.87159161181\\
0.674753901560624	262.761889530752\\
0.67515006002401	262.652287280527\\
0.675546218487395	262.542784715544\\
0.67594237695078	262.433381690504\\
0.676338535414166	262.324078060401\\
0.676734693877551	262.214873680522\\
0.677130852340936	262.105768406445\\
0.677527010804322	261.996762094038\\
0.677923169267707	261.887854599459\\
0.678319327731092	261.779045779156\\
0.678715486194478	261.670335489865\\
0.679111644657863	261.561723588607\\
0.679507803121248	261.453209932695\\
0.679903961584634	261.344794379724\\
0.680300120048019	261.236476787575\\
0.680696278511405	261.128257014415\\
0.68109243697479	261.020134918695\\
0.681488595438175	260.912110359148\\
0.681884753901561	260.80418319479\\
0.682280912364946	260.696353284918\\
0.682677070828331	260.588620489112\\
0.683073229291717	260.480984667231\\
0.683469387755102	260.373445679413\\
0.683865546218487	260.266003386078\\
0.684261704681873	260.158657647919\\
0.684657863145258	260.051408325911\\
0.685054021608643	259.944255281303\\
0.685450180072029	259.837198375624\\
0.685846338535414	259.730237470673\\
0.6862424969988	259.623372428527\\
0.686638655462185	259.516603111536\\
0.68703481392557	259.409929382325\\
0.687430972388956	259.30335110379\\
0.687827130852341	259.196868139097\\
0.688223289315726	259.090480351688\\
0.688619447779112	258.984187605272\\
0.689015606242497	258.877989763827\\
0.689411764705882	258.771886691604\\
0.689807923169268	258.66587825312\\
0.690204081632653	258.559964313159\\
0.690600240096038	258.454144736775\\
0.690996398559424	258.348419389285\\
0.691392557022809	258.242788136275\\
0.691788715486194	258.137250843593\\
0.69218487394958	258.031807377354\\
0.692581032412965	257.926457603936\\
0.692977190876351	257.82120138998\\
0.693373349339736	257.716038602389\\
0.693769507803121	257.610969108329\\
0.694165666266507	257.505992775224\\
0.694561824729892	257.401109470764\\
0.694957983193277	257.296319062894\\
0.695354141656663	257.191621419819\\
0.695750300120048	257.087016410005\\
0.696146458583433	256.982503902173\\
0.696542617046819	256.878083765303\\
0.696938775510204	256.77375586863\\
0.697334933973589	256.669520081648\\
0.697731092436975	256.565376274103\\
0.69812725090036	256.461324315996\\
0.698523409363745	256.357364077585\\
0.698919567827131	256.253495429378\\
0.699315726290516	256.149718242137\\
0.699711884753902	256.046032386877\\
0.700108043217287	255.942437734864\\
0.700504201680672	255.838934157613\\
0.700900360144058	255.735521526893\\
0.701296518607443	255.632199714718\\
0.701692677070828	255.528968593356\\
0.702088835534214	255.42582803532\\
0.702484993997599	255.322777913372\\
0.702881152460984	255.219818100519\\
0.70327731092437	255.116948470019\\
0.703673469387755	255.014168895372\\
0.70406962785114	254.911479250325\\
0.704465786314526	254.808879408869\\
0.704861944777911	254.70636924524\\
0.705258103241297	254.603948633916\\
0.705654261704682	254.50161744962\\
0.706050420168067	254.399375567316\\
0.706446578631453	254.29722286221\\
0.706842737094838	254.195159209748\\
0.707238895558223	254.093184485619\\
0.707635054021609	253.99129856575\\
0.708031212484994	253.889501326307\\
0.708427370948379	253.787792643696\\
0.708823529411765	253.686172394562\\
0.70921968787515	253.584640455784\\
0.709615846338535	253.483196704481\\
0.710012004801921	253.381841018009\\
0.710408163265306	253.280573273957\\
0.710804321728691	253.179393350151\\
0.711200480192077	253.078301124651\\
0.711596638655462	252.977296475753\\
0.711992797118848	252.876379281983\\
0.712388955582233	252.775549422103\\
0.712785114045618	252.674806775106\\
0.713181272509004	252.574151220216\\
0.713577430972389	252.473582636892\\
0.713973589435774	252.373100904818\\
0.71436974789916	252.272705903913\\
0.714765906362545	252.172397514323\\
0.71516206482593	252.072175616424\\
0.715558223289316	251.972040090819\\
0.715954381752701	251.87199081834\\
0.716350540216086	251.772027680048\\
0.716746698679472	251.672150557227\\
0.717142857142857	251.572359331389\\
0.717539015606242	251.472653884273\\
0.717935174069628	251.373034097842\\
0.718331332533013	251.273499854283\\
0.718727490996399	251.174051036007\\
0.719123649459784	251.074687525649\\
0.719519807923169	250.975409206068\\
0.719915966386555	250.876215960343\\
0.72031212484994	250.777107671776\\
0.720708283313325	250.678084223892\\
0.721104441776711	250.579145500434\\
0.721500600240096	250.480291385366\\
0.721896758703481	250.381521762873\\
0.722292917166867	250.282836517359\\
0.722689075630252	250.184235533444\\
0.723085234093637	250.085718695968\\
0.723481392557023	249.987285889991\\
0.723877551020408	249.888937000786\\
0.724273709483794	249.790671913844\\
0.724669867947179	249.692490514874\\
0.725066026410564	249.594392689796\\
0.72546218487395	249.49637832475\\
0.725858343337335	249.398447306088\\
0.72625450180072	249.300599520375\\
0.726650660264106	249.202834854391\\
0.727046818727491	249.105153195129\\
0.727442977190876	249.007554429794\\
0.727839135654262	248.910038445801\\
0.728235294117647	248.812605130781\\
0.728631452581032	248.71525437257\\
0.729027611044418	248.61798605922\\
0.729423769507803	248.520800078989\\
0.729819927971188	248.423696320346\\
0.730216086434574	248.326674671968\\
0.730612244897959	248.229735022742\\
0.731008403361345	248.132877261761\\
0.73140456182473	248.036101278326\\
0.731800720288115	247.939406961945\\
0.732196878751501	247.842794202333\\
0.732593037214886	247.74626288941\\
0.732989195678271	247.649812913301\\
0.733385354141657	247.553444164338\\
0.733781512605042	247.457156533055\\
0.734177671068427	247.360949910192\\
0.734573829531813	247.26482418669\\
0.734969987995198	247.168779253695\\
0.735366146458583	247.072815002554\\
0.735762304921969	246.976931324819\\
0.736158463385354	246.88112811224\\
0.73655462184874	246.785405256769\\
0.736950780312125	246.68976265056\\
0.73734693877551	246.594200185965\\
0.737743097238896	246.498717755538\\
0.738139255702281	246.40331525203\\
0.738535414165666	246.307992568392\\
0.738931572629052	246.212749597773\\
0.739327731092437	246.117586233518\\
0.739723889555822	246.022502369172\\
0.740120048019208	245.927497898475\\
0.740516206482593	245.832572715365\\
0.740912364945978	245.737726713973\\
0.741308523409364	245.642959788628\\
0.741704681872749	245.548271833854\\
0.742100840336134	245.453662744367\\
0.74249699879952	245.35913241508\\
0.742893157262905	245.264680741097\\
0.743289315726291	245.170307617718\\
0.743685474189676	245.076012940432\\
0.744081632653061	244.981796604923\\
0.744477791116447	244.887658507067\\
0.744873949579832	244.793598542929\\
0.745270108043217	244.699616608767\\
0.745666266506603	244.605712601028\\
0.746062424969988	244.511886416349\\
0.746458583433373	244.418137951559\\
0.746854741896759	244.324467103673\\
0.747250900360144	244.230873769895\\
0.747647058823529	244.137357847619\\
0.748043217286915	244.043919234426\\
0.7484393757503	243.950557828083\\
0.748835534213685	243.857273526545\\
0.749231692677071	243.764066227955\\
0.749627851140456	243.670935830639\\
0.750024009603842	243.57788223311\\
0.750420168067227	243.484905334066\\
0.750816326530612	243.392005032391\\
0.751212484993998	243.299181227151\\
0.751608643457383	243.206433817597\\
0.752004801920768	243.113762703164\\
0.752400960384154	243.021167783469\\
0.752797118847539	242.928648958311\\
0.753193277310924	242.836206127674\\
0.75358943577431	242.743839191721\\
0.753985594237695	242.651548050797\\
0.75438175270108	242.559332605427\\
0.754777911164466	242.467192756319\\
0.755174069627851	242.375128404359\\
0.755570228091237	242.283139450613\\
0.755966386554622	242.191225796327\\
0.756362545018007	242.099387342924\\
0.756758703481393	242.007623992008\\
0.757154861944778	241.915935645358\\
0.757551020408163	241.824322204933\\
0.757947178871549	241.732783572868\\
0.758343337334934	241.641319651475\\
0.758739495798319	241.549930343242\\
0.759135654261705	241.458615550834\\
0.75953181272509	241.36737517709\\
0.759927971188475	241.276209125025\\
0.760324129651861	241.185117297828\\
0.760720288115246	241.094099598864\\
0.761116446578631	241.003155931671\\
0.761512605042017	240.912286199959\\
0.761908763505402	240.821490307614\\
0.762304921968788	240.730768158692\\
0.762701080432173	240.640119657422\\
0.763097238895558	240.549544708207\\
0.763493397358944	240.459043215619\\
0.763889555822329	240.368615084402\\
0.764285714285714	240.278260219471\\
0.7646818727491	240.187978525911\\
0.765078031212485	240.097769908977\\
0.76547418967587	240.007634274095\\
0.765870348139256	239.917571526857\\
0.766266506602641	239.827581573026\\
0.766662665066026	239.737664318535\\
0.767058823529412	239.647819669481\\
0.767454981992797	239.558047532132\\
0.767851140456182	239.468347812921\\
0.768247298919568	239.37872041845\\
0.768643457382953	239.289165255486\\
0.769039615846339	239.199682230963\\
0.769435774309724	239.11027125198\\
0.769831932773109	239.020932225801\\
0.770228091236495	238.931665059856\\
0.77062424969988	238.84246966174\\
0.771020408163265	238.753345939211\\
0.771416566626651	238.664293800192\\
0.771812725090036	238.575313152768\\
0.772208883553421	238.486403905189\\
0.772605042016807	238.397565965866\\
0.773001200480192	238.308799243374\\
0.773397358943577	238.220103646448\\
0.773793517406963	238.131479083987\\
0.774189675870348	238.04292546505\\
0.774585834333734	237.954442698857\\
0.774981992797119	237.866030694788\\
0.775378151260504	237.777689362385\\
0.77577430972389	237.689418611347\\
0.776170468187275	237.601218351537\\
0.77656662665066	237.513088492971\\
0.776962785114046	237.42502894583\\
0.777358943577431	237.337039620448\\
0.777755102040816	237.249120427322\\
0.778151260504202	237.161271277102\\
0.778547418967587	237.073492080598\\
0.778943577430972	236.985782748777\\
0.779339735894358	236.898143192762\\
0.779735894357743	236.810573323833\\
0.780132052821128	236.723073053424\\
0.780528211284514	236.635642293126\\
0.780924369747899	236.548280954687\\
0.781320528211285	236.460988950006\\
0.78171668667467	236.37376619114\\
0.782112845138055	236.286612590297\\
0.782509003601441	236.199528059842\\
0.782905162064826	236.112512512291\\
0.783301320528211	236.025565860316\\
0.783697478991597	235.938688016738\\
0.784093637454982	235.851878894533\\
0.784489795918367	235.76513840683\\
0.784885954381753	235.678466466906\\
0.785282112845138	235.591862988194\\
0.785678271308523	235.505327884275\\
0.786074429771909	235.418861068881\\
0.786470588235294	235.332462455896\\
0.786866746698679	235.246131959354\\
0.787262905162065	235.159869493437\\
0.78765906362545	235.073674972477\\
0.788055222088835	234.987548310957\\
0.788451380552221	234.901489423506\\
0.788847539015606	234.815498224904\\
0.789243697478992	234.729574630077\\
0.789639855942377	234.6437185541\\
0.790036014405762	234.557929912195\\
0.790432172869148	234.472208619731\\
0.790828331332533	234.386554592224\\
0.791224489795918	234.300967745337\\
0.791620648259304	234.215447994879\\
0.792016806722689	234.129995256804\\
0.792412965186074	234.044609447212\\
0.79280912364946	233.959290482348\\
0.793205282112845	233.874038278602\\
0.793601440576231	233.788852752509\\
0.793997599039616	233.703733820748\\
0.794393757503001	233.618681400142\\
0.794789915966387	233.533695407656\\
0.795186074429772	233.4487757604\\
0.795582232893157	233.363922375627\\
0.795978391356543	233.279135170733\\
0.796374549819928	233.194414063254\\
0.796770708283313	233.10975897087\\
0.797166866746699	233.025169811402\\
0.797563025210084	232.940646502813\\
0.797959183673469	232.856188963207\\
0.798355342136855	232.771797110828\\
0.79875150060024	232.68747086406\\
0.799147659063625	232.60321014143\\
0.799543817527011	232.519014861602\\
0.799939975990396	232.43488494338\\
0.800336134453781	232.350820305708\\
0.800732292917167	232.266820867669\\
0.801128451380552	232.182886548484\\
0.801524609843938	232.099017267512\\
0.801920768307323	232.015212944251\\
0.802316926770708	231.931473498337\\
0.802713085234094	231.847798849541\\
0.803109243697479	231.764188917775\\
0.803505402160864	231.680643623084\\
0.80390156062425	231.597162885651\\
0.804297719087635	231.513746625796\\
0.80469387755102	231.430394763975\\
0.805090036014406	231.347107220778\\
0.805486194477791	231.26388391693\\
0.805882352941176	231.180724773295\\
0.806278511404562	231.097629710867\\
0.806674669867947	231.014598650778\\
0.807070828331332	230.931631514291\\
0.807466986794718	230.848728222805\\
0.807863145258103	230.765888697853\\
0.808259303721489	230.683112861099\\
0.808655462184874	230.600400634342\\
0.809051620648259	230.517751939513\\
0.809447779111645	230.435166698676\\
0.80984393757503	230.352644834026\\
0.810240096038415	230.27018626789\\
0.810636254501801	230.187790922728\\
0.811032412965186	230.105458721129\\
0.811428571428571	230.023189585816\\
0.811824729891957	229.940983439639\\
0.812220888355342	229.858840205582\\
0.812617046818727	229.776759806757\\
0.813013205282113	229.694742166407\\
0.813409363745498	229.612787207902\\
0.813805522208884	229.530894854746\\
0.814201680672269	229.449065030568\\
0.814597839135654	229.367297659127\\
0.81499399759904	229.28559266431\\
0.815390156062425	229.203949970134\\
0.81578631452581	229.122369500742\\
0.816182472989196	229.040851180406\\
0.816578631452581	228.959394933524\\
0.816974789915966	228.878000684621\\
0.817370948379352	228.796668358352\\
0.817767106842737	228.715397879493\\
0.818163265306122	228.634189172952\\
0.818559423769508	228.55304216376\\
0.818955582232893	228.471956777073\\
0.819351740696278	228.390932938174\\
0.819747899159664	228.309970572471\\
0.820144057623049	228.229069605498\\
0.820540216086435	228.148229962911\\
0.82093637454982	228.067451570492\\
0.821332533013205	227.986734354148\\
0.821728691476591	227.906078239907\\
0.822124849939976	227.825483153924\\
0.822521008403361	227.744949022475\\
0.822917166866747	227.664475771961\\
0.823313325330132	227.584063328903\\
0.823709483793517	227.503711619947\\
0.824105642256903	227.423420571861\\
0.824501800720288	227.343190111534\\
0.824897959183673	227.263020165978\\
0.825294117647059	227.182910662324\\
0.825690276110444	227.102861527828\\
0.82608643457383	227.022872689865\\
0.826482593037215	226.94294407593\\
0.8268787515006	226.863075613639\\
0.827274909963986	226.78326723073\\
0.827671068427371	226.703518855058\\
0.828067226890756	226.6238304146\\
0.828463385354142	226.544201837451\\
0.828859543817527	226.464633051827\\
0.829255702280912	226.385123986061\\
0.829651860744298	226.305674568606\\
0.830048019207683	226.226284728032\\
0.830444177671068	226.14695439303\\
0.830840336134454	226.067683492406\\
0.831236494597839	225.988471955085\\
0.831632653061224	225.909319710109\\
0.83202881152461	225.830226686638\\
0.832424969987995	225.751192813949\\
0.832821128451381	225.672218021434\\
0.833217286914766	225.593302238604\\
0.833613445378151	225.514445395084\\
0.834009603841537	225.435647420616\\
0.834405762304922	225.356908245058\\
0.834801920768307	225.278227798383\\
0.835198079231693	225.199606010679\\
0.835594237695078	225.12104281215\\
0.835990396158463	225.042538133113\\
0.836386554621849	224.964091904002\\
0.836782713085234	224.885704055364\\
0.837178871548619	224.807374517858\\
0.837575030012005	224.729103222261\\
0.83797118847539	224.650890099461\\
0.838367346938776	224.572735080458\\
0.838763505402161	224.494638096369\\
0.839159663865546	224.41659907842\\
0.839555822328932	224.338617957952\\
0.839951980792317	224.260694666417\\
0.840348139255702	224.182829135381\\
0.840744297719088	224.10502129652\\
0.841140456182473	224.027271081622\\
0.841536614645858	223.949578422587\\
0.841932773109244	223.871943251426\\
0.842328931572629	223.794365500262\\
0.842725090036014	223.716845101326\\
0.8431212484994	223.639381986963\\
0.843517406962785	223.561976089625\\
0.84391356542617	223.484627341877\\
0.844309723889556	223.407335676392\\
0.844705882352941	223.330101025952\\
0.845102040816327	223.252923323451\\
0.845498199279712	223.17580250189\\
0.845894357743097	223.098738494379\\
0.846290516206483	223.021731234138\\
0.846686674669868	222.944780654494\\
0.847082833133253	222.867886688883\\
0.847478991596639	222.79104927085\\
0.847875150060024	222.714268334045\\
0.848271308523409	222.637543812228\\
0.848667466986795	222.560875639266\\
0.84906362545018	222.484263749133\\
0.849459783913565	222.407708075908\\
0.849855942376951	222.33120855378\\
0.850252100840336	222.254765117043\\
0.850648259303722	222.178377700096\\
0.851044417767107	222.102046237446\\
0.851440576230492	222.025770663703\\
0.851836734693878	221.949550913587\\
0.852232893157263	221.873386921919\\
0.852629051620648	221.797278623628\\
0.853025210084034	221.721225953747\\
0.853421368547419	221.645228847413\\
0.853817527010804	221.569287239869\\
0.85421368547419	221.49340106646\\
0.854609843937575	221.417570262637\\
0.85500600240096	221.341794763956\\
0.855402160864346	221.266074506073\\
0.855798319327731	221.190409424751\\
0.856194477791116	221.114799455853\\
0.856590636254502	221.039244535348\\
0.856986794717887	220.963744599306\\
0.857382953181273	220.8882995839\\
0.857779111644658	220.812909425406\\
0.858175270108043	220.737574060201\\
0.858571428571429	220.662293424765\\
0.858967587034814	220.587067455679\\
0.859363745498199	220.511896089626\\
0.859759903961585	220.43677926339\\
0.86015606242497	220.361716913857\\
0.860552220888355	220.286708978012\\
0.860948379351741	220.211755392943\\
0.861344537815126	220.136856095838\\
0.861740696278511	220.062011023983\\
0.862136854741897	219.987220114767\\
0.862533013205282	219.912483305678\\
0.862929171668667	219.837800534303\\
0.863325330132053	219.763171738329\\
0.863721488595438	219.688596855542\\
0.864117647058824	219.614075823828\\
0.864513805522209	219.53960858117\\
0.864909963985594	219.465195065652\\
0.86530612244898	219.390835215455\\
0.865702280912365	219.316528968858\\
0.86609843937575	219.24227626424\\
0.866494597839136	219.168077040076\\
0.866890756302521	219.093931234939\\
0.867286914765906	219.0198387875\\
0.867683073229292	218.945799636526\\
0.868079231692677	218.871813720885\\
0.868475390156062	218.797880979536\\
0.868871548619448	218.724001351539\\
0.869267707082833	218.650174776049\\
0.869663865546219	218.576401192318\\
0.870060024009604	218.502680539693\\
0.870456182472989	218.429012757618\\
0.870852340936375	218.355397785633\\
0.87124849939976	218.281835563371\\
0.871644657863145	218.208326030563\\
0.872040816326531	218.134869127035\\
0.872436974789916	218.061464792707\\
0.872833133253301	217.988112967594\\
0.873229291716687	217.914813591805\\
0.873625450180072	217.841566605544\\
0.874021608643457	217.76837194911\\
0.874417767106843	217.695229562894\\
0.874813925570228	217.622139387383\\
0.875210084033613	217.549101363157\\
0.875606242496999	217.476115430887\\
0.876002400960384	217.403181531342\\
0.87639855942377	217.330299605379\\
0.876794717887155	217.257469593952\\
0.87719087635054	217.184691438104\\
0.877587034813926	217.111965078974\\
0.877983193277311	217.039290457791\\
0.878379351740696	216.966667515878\\
0.878775510204082	216.894096194647\\
0.879171668667467	216.821576435605\\
0.879567827130852	216.749108180349\\
0.879963985594238	216.676691370566\\
0.880360144057623	216.604325948038\\
0.880756302521008	216.532011854634\\
0.881152460984394	216.459749032317\\
0.881548619447779	216.387537423139\\
0.881944777911164	216.315376969242\\
0.88234093637455	216.243267612859\\
0.882737094837935	216.171209296315\\
0.883133253301321	216.099201962022\\
0.883529411764706	216.027245552482\\
0.883925570228091	215.955340010289\\
0.884321728691477	215.883485278125\\
0.884717887154862	215.811681298761\\
0.885114045618247	215.739928015056\\
0.885510204081633	215.668225369961\\
0.885906362545018	215.596573306512\\
0.886302521008403	215.524971767837\\
0.886698679471789	215.453420697149\\
0.887094837935174	215.381920037753\\
0.887490996398559	215.310469733038\\
0.887887154861945	215.239069726484\\
0.88828331332533	215.167719961656\\
0.888679471788716	215.096420382209\\
0.889075630252101	215.025170931884\\
0.889471788715486	214.953971554509\\
0.889867947178872	214.882822193999\\
0.890264105642257	214.811722794357\\
0.890660264105642	214.74067329967\\
0.891056422569028	214.669673654115\\
0.891452581032413	214.598723801952\\
0.891848739495798	214.527823687528\\
0.892244897959184	214.456973255278\\
0.892641056422569	214.386172449721\\
0.893037214885954	214.315421215461\\
0.89343337334934	214.244719497188\\
0.893829531812725	214.174067239678\\
0.89422569027611	214.103464387791\\
0.894621848739496	214.032910886473\\
0.895018007202881	213.962406680755\\
0.895414165666267	213.89195171575\\
0.895810324129652	213.821545936658\\
0.896206482593037	213.751189288763\\
0.896602641056423	213.680881717432\\
0.896998799519808	213.610623168117\\
0.897394957983193	213.540413586353\\
0.897791116446579	213.470252917759\\
0.898187274909964	213.400141108037\\
0.898583433373349	213.330078102973\\
0.898979591836735	213.260063848435\\
0.89937575030012	213.190098290375\\
0.899771908763505	213.120181374828\\
0.900168067226891	213.050313047911\\
0.900564225690276	212.980493255822\\
0.900960384153661	212.910721944845\\
0.901356542617047	212.840999061342\\
0.901752701080432	212.77132455176\\
0.902148859543818	212.701698362626\\
0.902545018007203	212.632120440551\\
0.902941176470588	212.562590732224\\
0.903337334933974	212.493109184418\\
0.903733493397359	212.423675743986\\
0.904129651860744	212.354290357863\\
0.90452581032413	212.284952973065\\
0.904921968787515	212.215663536686\\
0.9053181272509	212.146421995904\\
0.905714285714286	212.077228297976\\
0.906110444177671	212.008082390238\\
0.906506602641056	211.938984220109\\
0.906902761104442	211.869933735085\\
0.907298919567827	211.800930882743\\
0.907695078031213	211.73197561074\\
0.908091236494598	211.663067866812\\
0.908487394957983	211.594207598774\\
0.908883553421369	211.525394754521\\
0.909279711884754	211.456629282026\\
0.909675870348139	211.387911129342\\
0.910072028811525	211.319240244599\\
0.91046818727491	211.250616576008\\
0.910864345738295	211.182040071856\\
0.911260504201681	211.113510680509\\
0.911656662665066	211.045028350412\\
0.912052821128451	210.976593030088\\
0.912448979591837	210.908204668135\\
0.912845138055222	210.839863213232\\
0.913241296518607	210.771568614134\\
0.913637454981993	210.703320819673\\
0.914033613445378	210.63511977876\\
0.914429771908764	210.56696544038\\
0.914825930372149	210.498857753598\\
0.915222088835534	210.430796667554\\
0.91561824729892	210.362782131464\\
0.916014405762305	210.294814094623\\
0.91641056422569	210.2268925064\\
0.916806722689076	210.15901731624\\
0.917202881152461	210.091188473666\\
0.917599039615846	210.023405928276\\
0.917995198079232	209.955669629742\\
0.918391356542617	209.887979527814\\
0.918787515006002	209.820335572317\\
0.919183673469388	209.752737713149\\
0.919579831932773	209.685185900287\\
0.919975990396159	209.61768008378\\
0.920372148859544	209.550220213751\\
0.920768307322929	209.482806240402\\
0.921164465786315	209.415438114006\\
0.9215606242497	209.348115784912\\
0.921956782713085	209.280839203541\\
0.92235294117647	209.213608320391\\
0.922749099639856	209.146423086033\\
0.923145258103241	209.079283451112\\
0.923541416566627	209.012189366345\\
0.923937575030012	208.945140782525\\
0.924333733493397	208.878137650517\\
0.924729891956783	208.81117992126\\
0.925126050420168	208.744267545766\\
0.925522208883553	208.67740047512\\
0.925918367346939	208.610578660478\\
0.926314525810324	208.543802053073\\
0.92671068427371	208.477070604206\\
0.927106842737095	208.410384265254\\
0.92750300120048	208.343742987664\\
0.927899159663865	208.277146722956\\
0.928295318127251	208.210595422723\\
0.928691476590636	208.144089038628\\
0.929087635054022	208.077627522407\\
0.929483793517407	208.011210825868\\
0.929879951980792	207.944838900889\\
0.930276110444178	207.878511699422\\
0.930672268907563	207.812229173486\\
0.931068427370948	207.745991275175\\
0.931464585834334	207.679797956653\\
0.931860744297719	207.613649170153\\
0.932256902761104	207.547544867982\\
0.93265306122449	207.481485002514\\
0.933049219687875	207.415469526196\\
0.933445378151261	207.349498391544\\
0.933841536614646	207.283571551144\\
0.934237695078031	207.217688957654\\
0.934633853541416	207.151850563799\\
0.935030012004802	207.086056322376\\
0.935426170468187	207.020306186252\\
0.935822328931573	206.95460010836\\
0.936218487394958	206.888938041706\\
0.936614645858343	206.823319939363\\
0.937010804321729	206.757745754476\\
0.937406962785114	206.692215440256\\
0.937803121248499	206.626728949984\\
0.938199279711885	206.561286237009\\
0.93859543817527	206.495887254751\\
0.938991596638656	206.430531956695\\
0.939387755102041	206.365220296396\\
0.939783913565426	206.299952227478\\
0.940180072028811	206.234727703633\\
0.940576230492197	206.169546678619\\
0.940972388955582	206.104409106263\\
0.941368547418967	206.039314940461\\
0.941764705882353	205.974264135175\\
0.942160864345738	205.909256644434\\
0.942557022809124	205.844292422335\\
0.942953181272509	205.779371423044\\
0.943349339735894	205.714493600792\\
0.94374549819928	205.649658909876\\
0.944141656662665	205.584867304661\\
0.94453781512605	205.520118739581\\
0.944933973589436	205.455413169133\\
0.945330132052821	205.390750547881\\
0.945726290516206	205.326130830458\\
0.946122448979592	205.261553971561\\
0.946518607442977	205.197019925952\\
0.946914765906362	205.132528648462\\
0.947310924369748	205.068080093985\\
0.947707082833133	205.003674217483\\
0.948103241296519	204.939310973981\\
0.948499399759904	204.874990318573\\
0.948895558223289	204.810712206415\\
0.949291716686675	204.746476592729\\
0.94968787515006	204.682283432804\\
0.950084033613445	204.618132681992\\
0.950480192076831	204.55402429571\\
0.950876350540216	204.489958229441\\
0.951272509003601	204.425934438731\\
0.951668667466987	204.361952879191\\
0.952064825930372	204.298013506498\\
0.952460984393757	204.23411627639\\
0.952857142857143	204.170261144673\\
0.953253301320528	204.106448067214\\
0.953649459783913	204.042676999945\\
0.954045618247299	203.978947898862\\
0.954441776710684	203.915260720024\\
0.95483793517407	203.851615419554\\
0.955234093637455	203.788011953639\\
0.95563025210084	203.724450278528\\
0.956026410564226	203.660930350534\\
0.956422569027611	203.597452126033\\
0.956818727490996	203.534015561465\\
0.957214885954382	203.470620613329\\
0.957611044417767	203.407267238192\\
0.958007202881152	203.34395539268\\
0.958403361344538	203.280685033484\\
0.958799519807923	203.217456117353\\
0.959195678271308	203.154268601104\\
0.959591836734694	203.091122441613\\
0.959987995198079	203.028017595817\\
0.960384153661464	202.964954020718\\
0.96078031212485	202.901931673376\\
0.961176470588235	202.838950510918\\
0.961572629051621	202.776010490527\\
0.961968787515006	202.71311156945\\
0.962364945978391	202.650253704997\\
0.962761104441777	202.587436854536\\
0.963157262905162	202.524660975499\\
0.963553421368547	202.461926025376\\
0.963949579831933	202.39923196172\\
0.964345738295318	202.336578742146\\
0.964741896758703	202.273966324326\\
0.965138055222089	202.211394665997\\
0.965534213685474	202.148863724951\\
0.965930372148859	202.086373459047\\
0.966326530612245	202.023923826198\\
0.96672268907563	201.961514784381\\
0.967118847539016	201.899146291631\\
0.967515006002401	201.836818306046\\
0.967911164465786	201.774530785779\\
0.968307322929172	201.712283689047\\
0.968703481392557	201.650076974125\\
0.969099639855942	201.587910599346\\
0.969495798319328	201.525784523104\\
0.969891956782713	201.463698703854\\
0.970288115246098	201.401653100106\\
0.970684273709484	201.339647670432\\
0.971080432172869	201.277682373463\\
0.971476590636254	201.215757167887\\
0.97187274909964	201.153872012451\\
0.972268907563025	201.092026865963\\
0.97266506602641	201.030221687287\\
0.973061224489796	200.968456435347\\
0.973457382953181	200.906731069124\\
0.973853541416567	200.845045547657\\
0.974249699879952	200.783399830045\\
0.974645858343337	200.721793875444\\
0.975042016806723	200.660227643067\\
0.975438175270108	200.598701092186\\
0.975834333733493	200.53721418213\\
0.976230492196879	200.475766872286\\
0.976626650660264	200.414359122098\\
0.977022809123649	200.352990891068\\
0.977418967587035	200.291662138754\\
0.97781512605042	200.230372824772\\
0.978211284513805	200.169122908796\\
0.978607442977191	200.107912350556\\
0.979003601440576	200.046741109838\\
0.979399759903962	199.985609146485\\
0.979795918367347	199.924516420399\\
0.980192076830732	199.863462891536\\
0.980588235294118	199.802448519909\\
0.980984393757503	199.741473265587\\
0.981380552220888	199.680537088697\\
0.981776710684274	199.61963994942\\
0.982172869147659	199.558781807995\\
0.982569027611044	199.497962624714\\
0.98296518607443	199.437182359928\\
0.983361344537815	199.376440974042\\
0.983757503001201	199.315738427517\\
0.984153661464586	199.25507468087\\
0.984549819927971	199.194449694673\\
0.984945978391356	199.133863429553\\
0.985342136854742	199.073315846192\\
0.985738295318127	199.012806905329\\
0.986134453781513	198.952336567755\\
0.986530612244898	198.89190479432\\
0.986926770708283	198.831511545924\\
0.987322929171669	198.771156783526\\
0.987719087635054	198.710840468138\\
0.988115246098439	198.650562560825\\
0.988511404561825	198.590323022708\\
0.98890756302521	198.530121814964\\
0.989303721488596	198.46995889882\\
0.989699879951981	198.409834235562\\
0.990096038415366	198.349747786526\\
0.990492196878752	198.289699513104\\
0.990888355342137	198.229689376742\\
0.991284513805522	198.169717338939\\
0.991680672268907	198.109783361248\\
0.992076830732293	198.049887405275\\
0.992472989195678	197.990029432681\\
0.992869147659064	197.930209405179\\
0.993265306122449	197.870427284535\\
0.993661464585834	197.81068303257\\
0.99405762304922	197.750976611156\\
0.994453781512605	197.69130798222\\
0.99484993997599	197.63167710774\\
0.995246098439376	197.572083949748\\
0.995642256902761	197.512528470328\\
0.996038415366147	197.453010631619\\
0.996434573829532	197.393530395808\\
0.996830732292917	197.334087725139\\
0.997226890756303	197.274682581906\\
0.997623049219688	197.215314928455\\
0.998019207683073	197.155984727186\\
0.998415366146459	197.096691940549\\
0.998811524609844	197.037436531047\\
0.999207683073229	196.978218461235\\
0.999603841536615	196.91903769372\\
1	196.85989419116\\
1	196.85989419116\\
5.00120048019208	47.4741628634766\\
9.00240096038415	24.8986343796326\\
13.0036014405762	15.985089135877\\
17.0048019207683	11.3576570262754\\
21.0060024009604	8.58991344408891\\
25.0072028811525	6.780758842063\\
29.0084033613445	5.52299216617288\\
33.0096038415366	4.60775009245818\\
37.0108043217287	3.91784921333667\\
41.0120048019208	3.38296349101047\\
45.0132052821128	2.9586047034454\\
49.0144057623049	2.61540833435707\\
53.015606242497	2.33330061847869\\
57.0168067226891	2.09814837633044\\
61.0180072028812	1.89974733319915\\
65.0192076830732	1.73056785783114\\
69.0204081632653	1.58494729140237\\
73.0216086434574	1.45855506909952\\
77.0228091236495	1.34802967785685\\
81.0240096038415	1.25072682273727\\
85.0252100840336	1.1645413111958\\
89.0264105642257	1.08777886095383\\
93.0276110444178	1.01906237457766\\
97.0288115246099	0.957262427374077\\
101.030012004802	0.901445036792607\\
105.031212484994	0.850831945515616\\
109.032412965186	0.804770086705849\\
113.033613445378	0.762707869537182\\
117.03481392557	0.724176588075968\\
121.036014405762	0.688775719186432\\
125.037214885954	0.656161201295673\\
129.038415366146	0.626036018681217\\
133.039615846339	0.598142584072416\\
137.040816326531	0.572256535078297\\
141.042016806723	0.548181650438128\\
145.043217286915	0.525745659438014\\
149.044417767107	0.504796768409692\\
153.045618247299	0.485200766522722\\
157.046818727491	0.466838602308325\\
161.048019207683	0.44960434482603\\
165.049219687875	0.433403460785915\\
169.050420168067	0.41815135250412\\
173.051620648259	0.403772112210607\\
177.052821128451	0.390197456626286\\
181.054021608643	0.377365812392552\\
185.055222088836	0.365221528256223\\
189.056422569028	0.353714194180716\\
193.05762304922	0.342798050995263\\
197.058823529412	0.332431476981354\\
201.060024009604	0.322576540063919\\
205.061224489796	0.31319860612869\\
209.062424969988	0.304265995508769\\
213.06362545018	0.295749680937264\\
217.064825930372	0.287623021300035\\
221.066026410564	0.279861526383722\\
225.067226890756	0.272442648531751\\
229.068427370948	0.26534559772092\\
233.06962785114	0.258551177074382\\
237.070828331333	0.252041636250328\\
241.072028811525	0.245800540503102\\
245.073229291717	0.239812653516118\\
249.074429771909	0.234063832362843\\
253.075630252101	0.228540933170879\\
257.076830732293	0.223231726250874\\
261.078031212485	0.218124819611808\\
265.079231692677	0.213209589921311\\
269.080432172869	0.208476120087593\\
273.081632653061	0.203915142741232\\
277.082833133253	0.199517988982901\\
281.084033613445	0.195276541839188\\
285.085234093637	0.191183193934662\\
289.08643457383	0.187230808945771\\
293.087635054022	0.183412686452149\\
297.088835534214	0.179722529844587\\
301.090036014406	0.176154416987156\\
305.091236494598	0.172702773364405\\
309.09243697479	0.169362347473984\\
313.093637454982	0.166128188250882\\
317.094837935174	0.162995624332244\\
321.096038415366	0.159960244991832\\
325.097238895558	0.157017882590958\\
329.09843937575	0.154164596408454\\
333.099639855942	0.151396657726163\\
337.100840336134	0.148710536058849\\
341.102040816327	0.146102886428421\\
345.103241296519	0.143570537592154\\
349.104441776711	0.141110481143381\\
353.105642256903	0.138719861410884\\
357.106842737095	0.136395966090245\\
361.108043217287	0.134136217546638\\
365.109243697479	0.131938164734161\\
369.110444177671	0.129799475681835\\
373.111644657863	0.127717930500931\\
377.112845138055	0.125691414872328\\
381.114045618247	0.123717913976296\\
385.115246098439	0.121795506830399\\
389.116446578631	0.119922361004173\\
393.117647058824	0.118096727681969\\
397.118847539016	0.116316937047754\\
401.120048019208	0.114581393967907\\
405.1212484994	0.112888573950024\\
409.122448979592	0.111237019357566\\
413.123649459784	0.109625335861871\\
417.124849939976	0.108052189114494\\
421.126050420168	0.106516301624268\\
425.12725090036	0.105016449824675\\
429.128451380552	0.103551461318275\\
433.129651860744	0.102120212285989\\
437.130852340936	0.100721625049941\\
441.132052821128	0.0993546657794728\\
445.133253301321	0.0980183423307042\\
449.134453781513	0.0967117022107546\\
453.135654261705	0.0954338306583988\\
457.136854741897	0.0941838488335438\\
461.138055222089	0.0929609121084682\\
465.139255702281	0.0917642084542847\\
469.140456182473	0.0905929569165539\\
473.141656662665	0.0894464061744162\\
477.142857142857	0.0883238331780074\\
481.144057623049	0.0872245418592906\\
485.145258103241	0.0861478619117792\\
489.146458583433	0.0850931476349396\\
493.147659063625	0.0840597768393487\\
497.148859543818	0.0830471498089521\\
501.15006002401	0.0820546883170144\\
505.151260504202	0.0810818346925824\\
509.152460984394	0.0801280509344919\\
513.153661464586	0.0791928178701492\\
517.154861944778	0.0782756343564937\\
521.15606242497	0.0773760165207228\\
525.157262905162	0.076493497038512\\
529.158463385354	0.0756276244476118\\
533.159663865546	0.0747779624948341\\
537.160864345738	0.0739440895145695\\
541.16206482593	0.073125597837092\\
545.163265306122	0.0723220932250148\\
549.164465786315	0.0715331943363654\\
553.165666266507	0.0707585322128375\\
557.166866746699	0.0699977497918691\\
561.168067226891	0.0692505014412756\\
565.169267707083	0.0685164525152433\\
569.170468187275	0.0677952789305601\\
573.171668667467	0.0670866667620277\\
577.172869147659	0.0663903118560582\\
581.174069627851	0.0657059194615218\\
585.175270108043	0.06503320387696\\
589.176470588235	0.0643718881133356\\
593.177671068427	0.0637217035715331\\
597.178871548619	0.0630823897338718\\
601.180072028812	0.0624536938689336\\
605.181272509004	0.0618353707490454\\
609.182472989196	0.0612271823797968\\
613.183673469388	0.0606288977410027\\
617.18487394958	0.0600402925385582\\
621.186074429772	0.0594611489666589\\
625.187274909964	0.0588912554798909\\
629.188475390156	0.0583304065747217\\
633.189675870348	0.057778402579946\\
637.19087635054	0.0572350494556664\\
641.192076830732	0.0567001586004105\\
645.193277310924	0.0561735466660068\\
649.194477791116	0.0556550353798613\\
653.195678271309	0.0551444513742954\\
657.196878751501	0.0546416260226238\\
661.198079231693	0.0541463952816673\\
665.199279711885	0.05365859954041\\
669.200480192077	0.0531780834745266\\
673.201680672269	0.0527046959065188\\
677.202881152461	0.0522382896712121\\
681.204081632653	0.0517787214863778\\
685.205282112845	0.0513258518282565\\
689.206482593037	0.050879544811769\\
693.207683073229	0.050439668075214\\
697.208883553421	0.0500060926692574\\
701.210084033614	0.049578692950032\\
705.211284513805	0.0491573464761723\\
709.212484993998	0.0487419339096172\\
713.21368547419	0.048332338920025\\
717.214885954382	0.0479284480926464\\
721.216086434574	0.0475301508395147\\
725.217286914766	0.047137339313815\\
729.218487394958	0.0467499083273017\\
733.21968787515	0.0463677552706406\\
737.220888355342	0.0459907800365554\\
741.222088835534	0.0456188849456668\\
745.223289315726	0.0452519746749152\\
749.224489795918	0.044889956188463\\
753.22569027611	0.04453273867098\\
757.226890756303	0.0441802334632151\\
761.228091236495	0.0438323539997658\\
765.229291716687	0.0434890157489598\\
769.230492196879	0.0431501361547649\\
773.231692677071	0.0428156345806503\\
777.232893157263	0.0424854322553229\\
781.234093637455	0.042159452220267\\
785.235294117647	0.0418376192790195\\
789.236494597839	0.0415198599481128\\
793.237695078031	0.0412061024096247\\
797.238895558223	0.0408962764652729\\
801.240096038415	0.0405903134919976\\
805.241296518607	0.0402881463989766\\
809.2424969988	0.0399897095860194\\
813.243697478992	0.0396949389032897\\
817.244897959184	0.0394037716123087\\
821.246098439376	0.0391161463481896\\
825.247298919568	0.0388320030830626\\
829.24849939976	0.0385512830906441\\
833.249699879952	0.0382739289119097\\
837.250900360144	0.0379998843218334\\
841.252100840336	0.0377290942971522\\
845.253301320528	0.0374615049851215\\
849.25450180072	0.0371970636732266\\
853.255702280912	0.0369357187598148\\
857.256902761104	0.0366774197256187\\
861.258103241297	0.0364221171061367\\
865.259303721489	0.0361697624648435\\
869.260504201681	0.0359203083672008\\
873.261704681873	0.0356737083554405\\
877.262905162065	0.0354299169240953\\
881.264105642257	0.0351888894962492\\
885.265306122449	0.034950582400486\\
889.266506602641	0.0347149528485104\\
893.267707082833	0.0344819589134197\\
897.268907563025	0.0342515595086049\\
901.270108043217	0.0340237143672592\\
905.271308523409	0.0337983840224753\\
909.272509003601	0.0335755297879101\\
913.273709483793	0.0333551137390003\\
917.274909963986	0.0331370986947093\\
921.276110444178	0.0329214481997892\\
925.27731092437	0.0327081265075397\\
929.278511404562	0.0324970985630504\\
933.279711884754	0.0322883299869084\\
937.280912364946	0.0320817870593575\\
941.282112845138	0.0318774367048951\\
945.28331332533	0.0316752464772918\\
949.284513805522	0.0314751845450211\\
953.285714285714	0.0312772196770862\\
957.286914765906	0.031081321229231\\
961.288115246098	0.0308874591305244\\
965.289315726291	0.0306956038703049\\
969.290516206483	0.0305057264854755\\
973.291716686675	0.0303177985481374\\
977.292917166867	0.0301317921535527\\
981.294117647059	0.0299476799084256\\
985.295318127251	0.0297654349194923\\
989.296518607443	0.0295850307824111\\
993.297719087635	0.0294064415709423\\
997.298919567827	0.0292296418264109\\
1001.30012004802	0.0290546065474413\\
1005.30132052821	0.0288813111799589\\
1009.3025210084	0.0287097316074474\\
1013.3037214886	0.0285398441414575\\
1017.30492196879	0.0283716255123563\\
1021.30612244898	0.0282050528603136\\
1025.30732292917	0.0280401037265153\\
1029.30852340936	0.0278767560445991\\
1033.30972388956	0.0277149881323059\\
1037.31092436975	0.0275547786833393\\
1041.31212484994	0.0273961067594294\\
1045.31332533013	0.0272389517825925\\
1049.31452581032	0.0270832935275836\\
1053.31572629052	0.0269291121145347\\
1057.31692677071	0.0267763880017733\\
1061.3181272509	0.0266251019788177\\
1065.31932773109	0.0264752351595426\\
1069.32052821128	0.0263267689755108\\
1073.32172869148	0.0261796851694663\\
1077.32292917167	0.0260339657889843\\
1081.32412965186	0.025889593180274\\
1085.32533013205	0.0257465499821294\\
1089.32653061224	0.0256048191200244\\
1093.32773109244	0.0254643838003487\\
1097.32893157263	0.0253252275047794\\
1101.33013205282	0.0251873339847864\\
1105.33133253301	0.0250506872562655\\
1109.33253301321	0.0249152715942987\\
1113.3337334934	0.0247810715280361\\
1117.33493397359	0.0246480718356962\\
1121.33613445378	0.0245162575396834\\
1125.33733493397	0.0243856139018171\\
1129.33853541417	0.0242561264186711\\
1133.33973589436	0.0241277808170194\\
1137.34093637455	0.0240005630493867\\
1141.34213685474	0.0238744592896998\\
1145.34333733493	0.0237494559290372\\
1149.34453781513	0.0236255395714759\\
1153.34573829532	0.0235026970300307\\
1157.34693877551	0.0233809153226851\\
1161.3481392557	0.023260181668511\\
1165.34933973589	0.0231404834838752\\
1169.35054021609	0.0230218083787294\\
1173.35174069628	0.022904144152983\\
1177.35294117647	0.022787478792956\\
1181.35414165666	0.0226718004679095\\
1185.35534213685	0.0225570975266524\\
1189.35654261705	0.0224433584942223\\
1193.35774309724	0.0223305720686389\\
1197.35894357743	0.0222187271177274\\
1201.36014405762	0.0221078126760105\\
1205.36134453782	0.0219978179416684\\
1209.36254501801	0.0218887322735628\\
1213.3637454982	0.0217805451883256\\
1217.36494597839	0.0216732463575092\\
1221.36614645858	0.0215668256047978\\
1225.36734693878	0.0214612729032773\\
1229.36854741897	0.0213565783727638\\
1233.36974789916	0.0212527322771874\\
1237.37094837935	0.0211497250220314\\
1241.37214885954	0.0210475471518253\\
1245.37334933974	0.0209461893476898\\
1249.37454981993	0.0208456424249328\\
1253.37575030012	0.0207458973306955\\
1257.37695078031	0.0206469451416472\\
1261.3781512605	0.020548777061727\\
1265.3793517407	0.0204513844199327\\
1269.38055222089	0.0203547586681545\\
1273.38175270108	0.0202588913790525\\
1277.38295318127	0.0201637742439786\\
1281.38415366146	0.020069399070939\\
1285.38535414166	0.0199757577825994\\
1289.38655462185	0.0198828424143291\\
1293.38775510204	0.0197906451122853\\
1297.38895558223	0.0196991581315352\\
1301.39015606242	0.019608373834216\\
1305.39135654262	0.0195182846877315\\
1309.39255702281	0.0194288832629841\\
1313.393757503	0.0193401622326419\\
1317.39495798319	0.0192521143694407\\
1321.39615846339	0.0191647325445183\\
1325.39735894358	0.0190780097257819\\
1329.39855942377	0.018991938976308\\
1333.39975990396	0.0189065134527726\\
1337.40096038415	0.0188217264039123\\
1341.40216086435	0.0187375711690151\\
1345.40336134454	0.0186540411764407\\
1349.40456182473	0.0185711299421688\\
1353.40576230492	0.0184888310683755\\
1357.40696278511	0.0184071382420372\\
1361.40816326531	0.0183260452335606\\
1365.4093637455	0.0182455458954393\\
1369.41056422569	0.0181656341609353\\
1373.41176470588	0.0180863040427865\\
1377.41296518607	0.0180075496319372\\
1381.41416566627	0.0179293650962936\\
1385.41536614646	0.0178517446795021\\
1389.41656662665	0.0177746826997506\\
1393.41776710684	0.0176981735485922\\
1397.41896758703	0.0176222116897906\\
1401.42016806723	0.0175467916581871\\
1405.42136854742	0.0174719080585882\\
1409.42256902761	0.0173975555646747\\
1413.4237695078	0.0173237289179294\\
1417.424969988	0.0172504229265854\\
1421.42617046819	0.0171776324645936\\
1425.42737094838	0.017105352470609\\
1429.42857142857	0.0170335779469946\\
1433.42977190876	0.0169623039588449\\
1437.43097238896	0.0168915256330256\\
1441.43217286915	0.0168212381572314\\
1445.43337334934	0.0167514367790607\\
1449.43457382953	0.0166821168051066\\
1453.43577430972	0.016613273600064\\
1457.43697478992	0.0165449025858537\\
1461.43817527011	0.0164769992407604\\
1465.4393757503	0.0164095590985877\\
1469.44057623049	0.0163425777478265\\
1473.44177671068	0.0162760508308393\\
1477.44297719088	0.0162099740430583\\
1481.44417767107	0.0161443431321971\\
1485.44537815126	0.0160791538974776\\
1489.44657863145	0.0160144021888686\\
1493.44777911164	0.0159500839063391\\
1497.44897959184	0.0158861949991237\\
1501.45018007203	0.0158227314650015\\
1505.45138055222	0.0157596893495862\\
1509.45258103241	0.0156970647456299\\
1513.45378151261	0.0156348537923377\\
1517.4549819928	0.0155730526746948\\
1521.45618247299	0.0155116576228045\\
1525.45738295318	0.015450664911238\\
1529.45858343337	0.0153900708583947\\
1533.45978391357	0.0153298718258741\\
1537.46098439376	0.0152700642178574\\
1541.46218487395	0.0152106444805002\\
1545.46338535414	0.0151516091013354\\
1549.46458583433	0.0150929546086856\\
1553.46578631453	0.0150346775710857\\
1557.46698679472	0.0149767745967154\\
1561.46818727491	0.0149192423328402\\
1565.4693877551	0.0148620774652633\\
1569.47058823529	0.0148052767177844\\
1573.47178871549	0.0147488368516696\\
1577.47298919568	0.0146927546651286\\
1581.47418967587	0.0146370269928011\\
1585.47539015606	0.0145816507052514\\
1589.47659063625	0.0145266227084712\\
1593.47779111645	0.0144719399433912\\
1597.47899159664	0.0144175993853993\\
1601.48019207683	0.014363598043868\\
1605.48139255702	0.0143099329616889\\
1609.48259303721	0.0142566012148141\\
1613.48379351741	0.0142035999118061\\
1617.4849939976	0.0141509261933943\\
1621.48619447779	0.0140985772320387\\
1625.48739495798	0.0140465502315003\\
1629.48859543818	0.0139948424264191\\
1633.48979591837	0.013943451081898\\
1637.49099639856	0.0138923734930939\\
1641.49219687875	0.013841606984815\\
1645.49339735894	0.0137911489111241\\
1649.49459783914	0.0137409966549491\\
1653.49579831933	0.0136911476276989\\
1657.49699879952	0.0136415992688852\\
1661.49819927971	0.013592349045751\\
1665.4993997599	0.0135433944529038\\
1669.5006002401	0.0134947330119557\\
1673.50180072029	0.0134463622711682\\
1677.50300120048	0.0133982798051027\\
1681.50420168067	0.013350483214277\\
1685.50540216086	0.0133029701248259\\
1689.50660264106	0.0132557381881684\\
1693.50780312125	0.0132087850806789\\
1697.50900360144	0.013162108503364\\
1701.51020408163	0.0131157061815441\\
1705.51140456182	0.0130695758645401\\
1709.51260504202	0.0130237153253641\\
1713.51380552221	0.0129781223604157\\
1717.5150060024	0.0129327947891827\\
1721.51620648259	0.0128877304539455\\
1725.51740696279	0.0128429272194872\\
1729.51860744298	0.0127983829728069\\
1733.51980792317	0.0127540956228385\\
1737.52100840336	0.0127100631001724\\
1741.52220888355	0.0126662833567824\\
1745.52340936375	0.0126227543657561\\
1749.52460984394	0.0125794741210293\\
1753.52581032413	0.0125364406371252\\
1757.52701080432	0.0124936519488957\\
1761.52821128451	0.0124511061112687\\
1765.52941176471	0.0124088011989971\\
1769.5306122449	0.0123667353064128\\
1773.53181272509	0.012324906547184\\
1777.53301320528	0.0122833130540758\\
1781.53421368547	0.0122419529787142\\
1785.53541416567	0.0122008244913542\\
1789.53661464586	0.0121599257806506\\
1793.53781512605	0.0121192550534324\\
1797.53901560624	0.0120788105344804\\
1801.54021608643	0.0120385904663082\\
1805.54141656663	0.0119985931089459\\
1809.54261704682	0.0119588167397274\\
1813.54381752701	0.0119192596530804\\
1817.5450180072	0.0118799201603198\\
1821.5462184874	0.0118407965894436\\
1825.54741896759	0.0118018872849317\\
1829.54861944778	0.0117631906075483\\
1833.54981992797	0.0117247049341459\\
1837.55102040816	0.0116864286574735\\
1841.55222088836	0.011648360185986\\
1845.55342136855	0.0116104979436573\\
1849.55462184874	0.011572840369796\\
1853.55582232893	0.0115353859188631\\
1857.55702280912	0.0114981330602928\\
1861.55822328932	0.0114610802783153\\
1865.55942376951	0.0114242260717826\\
1869.5606242497	0.0113875689539967\\
1873.56182472989	0.0113511074525394\\
1877.56302521008	0.0113148401091054\\
1881.56422569028	0.0112787654793373\\
1885.56542617047	0.0112428821326626\\
1889.56662665066	0.0112071886521338\\
1893.56782713085	0.0111716836342697\\
1897.56902761104	0.0111363656888992\\
1901.57022809124	0.0111012334390079\\
1905.57142857143	0.0110662855205857\\
1909.57262905162	0.0110315205824773\\
1913.57382953181	0.0109969372862342\\
1917.575030012	0.0109625343059694\\
1921.5762304922	0.0109283103282132\\
1925.57743097239	0.0108942640517713\\
1929.57863145258	0.0108603941875853\\
1933.57983193277	0.0108266994585943\\
1937.58103241297	0.0107931785995988\\
1941.58223289316	0.0107598303571265\\
1945.58343337335	0.0107266534892994\\
1949.58463385354	0.0106936467657032\\
1953.58583433373	0.0106608089672582\\
1957.58703481393	0.0106281388860918\\
1961.58823529412	0.0105956353254133\\
1965.58943577431	0.0105632970993891\\
1969.5906362545	0.0105311230330212\\
1973.59183673469	0.0104991119620258\\
1977.59303721489	0.0104672627327143\\
1981.59423769508	0.0104355742018761\\
1985.59543817527	0.0104040452366618\\
1989.59663865546	0.0103726747144692\\
1993.59783913565	0.0103414615228302\\
1997.59903961585	0.0103104045592988\\
2001.60024009604	0.0102795027313414\\
2005.60144057623	0.0102487549562282\\
2009.60264105642	0.0102181601609252\\
2013.60384153661	0.0101877172819889\\
2017.60504201681	0.0101574252654618\\
2021.606242497	0.0101272830667687\\
2025.60744297719	0.0100972896506152\\
2029.60864345738	0.0100674439908868\\
2033.60984393758	0.0100377450705501\\
2037.61104441777	0.0100081918815543\\
2041.61224489796	0.00997878342473467\\
2045.61344537815	0.00994951870971688\\
2049.61464585834	0.00992039675482275\\
2053.61584633854	0.00989141658697706\\
2057.61704681873	0.00986257724161561\\
2061.61824729892	0.00983387776259438\\
2065.61944777911	0.00980531720209991\\
2069.6206482593	0.00977689462056071\\
2073.6218487395	0.00974860908655987\\
2077.62304921969	0.00972045967674874\\
2081.62424969988	0.0096924454757616\\
2085.62545018007	0.00966456557613157\\
2089.62665066026	0.00963681907820737\\
2093.62785114046	0.00960920509007129\\
2097.62905162065	0.009581722727458\\
2101.63025210084	0.00955437111367455\\
2105.63145258103	0.00952714937952121\\
2109.63265306122	0.00950005666321337\\
2113.63385354142	0.00947309211030434\\
2117.63505402161	0.00944625487360916\\
2121.6362545018	0.00941954411312929\\
2125.63745498199	0.00939295899597827\\
2129.63865546218	0.00936649869630823\\
2133.63985594238	0.00934016239523735\\
2137.64105642257	0.00931394928077816\\
2141.64225690276	0.00928785854776677\\
2145.64345738295	0.00926188939779284\\
2149.64465786315	0.00923604103913056\\
2153.64585834334	0.00921031268667033\\
2157.64705882353	0.00918470356185128\\
2161.64825930372	0.00915921289259472\\
2165.64945978391	0.00913383991323821\\
2169.65066026411	0.00910858386447056\\
2173.6518607443	0.00908344399326754\\
2177.65306122449	0.00905841955282839\\
2181.65426170468	0.00903350980251304\\
2185.65546218487	0.00900871400778015\\
2189.65666266507	0.00898403144012581\\
2193.65786314526	0.00895946137702301\\
2197.65906362545	0.00893500310186182\\
2201.66026410564	0.00891065590389028\\
2205.66146458583	0.00888641907815595\\
2209.66266506603	0.00886229192544822\\
2213.66386554622	0.00883827375224121\\
2217.66506602641	0.00881436387063741\\
2221.6662665066	0.00879056159831194\\
2225.66746698679	0.00876686625845749\\
2229.66866746699	0.00874327717972988\\
2233.66986794718	0.00871979369619425\\
2237.67106842737	0.00869641514727189\\
2241.67226890756	0.00867314087768767\\
2245.67346938776	0.00864997023741813\\
2249.67466986795	0.00862690258164008\\
2253.67587034814	0.00860393727067986\\
2257.67707082833	0.00858107366996319\\
2261.67827130852	0.00855831114996555\\
2265.67947178872	0.00853564908616314\\
2269.68067226891	0.00851308685898446\\
2273.6818727491	0.00849062385376233\\
2277.68307322929	0.00846825946068661\\
2281.68427370948	0.0084459930747573\\
2285.68547418968	0.00842382409573827\\
2289.68667466987	0.00840175192811153\\
2293.68787515006	0.00837977598103193\\
2297.68907563025	0.00835789566828245\\
2301.69027611044	0.00833611040822997\\
2305.69147659064	0.00831441962378151\\
2309.69267707083	0.00829282274234104\\
2313.69387755102	0.00827131919576668\\
2317.69507803121	0.00824990842032844\\
2321.6962785114	0.00822858985666643\\
2325.6974789916	0.00820736294974952\\
2329.69867947179	0.00818622714883446\\
2333.69987995198	0.00816518190742545\\
2337.70108043217	0.00814422668323421\\
2341.70228091237	0.00812336093814042\\
2345.70348139256	0.00810258413815266\\
2349.70468187275	0.00808189575336974\\
2353.70588235294	0.00806129525794246\\
2357.70708283313	0.00804078213003583\\
2361.70828331333	0.0080203558517917\\
2365.70948379352	0.00800001590929171\\
2369.71068427371	0.00797976179252078\\
2373.7118847539	0.00795959299533092\\
2377.71308523409	0.00793950901540545\\
2381.71428571429	0.00791950935422361\\
2385.71548619448	0.00789959351702557\\
2389.71668667467	0.00787976101277781\\
2393.71788715486	0.00786001135413886\\
2397.71908763505	0.00784034405742543\\
2401.72028811525	0.00782075864257894\\
2405.72148859544	0.00780125463313233\\
2409.72268907563	0.00778183155617729\\
2413.72388955582	0.00776248894233184\\
2417.72509003601	0.00774322632570823\\
2421.72629051621	0.00772404324388119\\
2425.7274909964	0.00770493923785658\\
2429.72869147659	0.00768591385204026\\
2433.72989195678	0.00766696663420739\\
2437.73109243697	0.00764809713547204\\
2441.73229291717	0.00762930491025709\\
2445.73349339736	0.00761058951626446\\
2449.73469387755	0.0075919505144457\\
2453.73589435774	0.00757338746897282\\
2457.73709483794	0.00755489994720948\\
2461.73829531813	0.00753648751968249\\
2465.73949579832	0.00751814976005353\\
2469.74069627851	0.0074998862450913\\
2473.7418967587	0.00748169655464386\\
2477.7430972389	0.00746358027161128\\
2481.74429771909	0.00744553698191858\\
2485.74549819928	0.007427566274489\\
2489.74669867947	0.00740966774121748\\
2493.74789915966	0.00739184097694444\\
2497.74909963986	0.00737408557942986\\
2501.75030012005	0.0073564011493276\\
2505.75150060024	0.00733878729015998\\
2509.75270108043	0.00732124360829266\\
2513.75390156062	0.00730376971290975\\
2517.75510204082	0.00728636521598917\\
2521.75630252101	0.0072690297322783\\
2525.7575030012	0.00725176287926985\\
2529.75870348139	0.00723456427717798\\
2533.75990396158	0.00721743354891468\\
2537.76110444178	0.0072003703200664\\
2541.76230492197	0.00718337421887087\\
2545.76350540216	0.00716644487619421\\
2549.76470588235	0.00714958192550824\\
2553.76590636255	0.00713278500286807\\
2557.76710684274	0.00711605374688983\\
2561.76830732293	0.00709938779872873\\
2565.76950780312	0.00708278680205723\\
2569.77070828331	0.00706625040304356\\
2573.77190876351	0.00704977825033034\\
2577.7731092437	0.00703336999501348\\
2581.77430972389	0.00701702529062127\\
2585.77551020408	0.00700074379309369\\
2589.77671068427	0.00698452516076195\\
2593.77791116447	0.00696836905432815\\
2597.77911164466	0.00695227513684525\\
2601.78031212485	0.00693624307369719\\
2605.78151260504	0.0069202725325792\\
2609.78271308523	0.00690436318347831\\
2613.78391356543	0.00688851469865408\\
2617.78511404562	0.00687272675261949\\
2621.78631452581	0.00685699902212204\\
2625.787515006	0.00684133118612501\\
2629.78871548619	0.00682572292578896\\
2633.78991596639	0.00681017392445336\\
2637.79111644658	0.00679468386761839\\
2641.79231692677	0.006779252442927\\
2645.79351740696	0.00676387934014704\\
2649.79471788716	0.00674856425115369\\
2653.79591836735	0.0067333068699119\\
2657.79711884754	0.00671810689245916\\
2661.79831932773	0.00670296401688835\\
2665.79951980792	0.0066878779433308\\
2669.80072028812	0.00667284837393943\\
2673.80192076831	0.00665787501287221\\
2677.8031212485	0.0066429575662756\\
2681.80432172869	0.00662809574226829\\
2685.80552220888	0.00661328925092505\\
2689.80672268908	0.00659853780426068\\
2693.80792316927	0.00658384111621424\\
2697.80912364946	0.0065691989026333\\
2701.81032412965	0.00655461088125843\\
2705.81152460984	0.00654007677170778\\
2709.81272509004	0.0065255962954619\\
2713.81392557023	0.00651116917584856\\
2717.81512605042	0.00649679513802786\\
2721.81632653061	0.00648247390897739\\
2725.8175270108	0.00646820521747759\\
2729.818727491	0.00645398879409719\\
2733.81992797119	0.00643982437117882\\
2737.82112845138	0.00642571168282479\\
2741.82232893157	0.00641165046488295\\
2745.82352941176	0.00639764045493268\\
2749.82472989196	0.00638368139227109\\
2753.82593037215	0.00636977301789926\\
2757.82713085234	0.00635591507450865\\
2761.82833133253	0.00634210730646767\\
2765.82953181273	0.00632834945980827\\
2769.83073229292	0.00631464128221283\\
2773.83193277311	0.00630098252300096\\
2777.8331332533	0.00628737293311663\\
2781.83433373349	0.00627381226511525\\
2785.83553421369	0.00626030027315099\\
2789.83673469388	0.00624683671296417\\
2793.83793517407	0.00623342134186874\\
2797.83913565426	0.00622005391873994\\
2801.84033613445	0.00620673420400201\\
2805.84153661465	0.0061934619596161\\
2809.84273709484	0.00618023694906817\\
2813.84393757503	0.0061670589373571\\
2817.84513805522	0.00615392769098288\\
2821.84633853541	0.00614084297793492\\
2825.84753901561	0.00612780456768042\\
2829.8487394958	0.00611481223115291\\
2833.84993997599	0.00610186574074086\\
2837.85114045618	0.0060889648702764\\
2841.85234093637	0.00607610939502413\\
2845.85354141657	0.00606329909167007\\
2849.85474189676	0.00605053373831069\\
2853.85594237695	0.00603781311444201\\
2857.85714285714	0.00602513700094883\\
2861.85834333734	0.0060125051800941\\
2865.85954381753	0.00599991743550828\\
2869.86074429772	0.00598737355217888\\
2873.86194477791	0.0059748733164401\\
2877.8631452581	0.0059624165159625\\
2881.8643457383	0.00595000293974279\\
2885.86554621849	0.00593763237809378\\
2889.86674669868	0.00592530462263428\\
2893.86794717887	0.00591301946627924\\
2897.86914765906	0.00590077670322989\\
2901.87034813926	0.00588857612896394\\
2905.87154861945	0.00587641754022601\\
2909.87274909964	0.00586430073501796\\
2913.87394957983	0.00585222551258945\\
2917.87515006002	0.00584019167342854\\
2921.87635054022	0.00582819901925232\\
2925.87755102041	0.0058162473529977\\
2929.8787515006	0.00580433647881226\\
2933.87995198079	0.00579246620204513\\
2937.88115246098	0.00578063632923805\\
2941.88235294118	0.0057688466681164\\
2945.88355342137	0.00575709702758039\\
2949.88475390156	0.0057453872176963\\
2953.88595438175	0.00573371704968776\\
2957.88715486195	0.0057220863359272\\
2961.88835534214	0.00571049488992727\\
2965.88955582233	0.00569894252633238\\
2969.89075630252	0.00568742906091038\\
2973.89195678271	0.00567595431054415\\
2977.89315726291	0.00566451809322346\\
2981.8943577431	0.00565312022803673\\
2985.89555822329	0.00564176053516298\\
2989.89675870348	0.00563043883586379\\
2993.89795918367	0.00561915495247534\\
2997.89915966387	0.00560790870840053\\
3001.90036014406	0.00559669992810116\\
3005.90156062425	0.0055855284370902\\
3009.90276110444	0.00557439406192406\\
3013.90396158463	0.00556329663019503\\
3017.90516206483	0.00555223597052369\\
3021.90636254502	0.00554121191255142\\
3025.90756302521	0.00553022428693303\\
3029.9087635054	0.00551927292532932\\
3033.90996398559	0.00550835766039987\\
3037.91116446579	0.00549747832579575\\
3041.91236494598	0.00548663475615236\\
3045.91356542617	0.00547582678708235\\
3049.91476590636	0.00546505425516852\\
3053.91596638655	0.00545431699795688\\
3057.91716686675	0.0054436148539497\\
3061.91836734694	0.00543294766259862\\
3065.91956782713	0.00542231526429787\\
3069.92076830732	0.00541171750037752\\
3073.92196878752	0.00540115421309672\\
3077.92316926771	0.00539062524563714\\
3081.9243697479	0.00538013044209633\\
3085.92557022809	0.00536966964748123\\
3089.92677070828	0.00535924270770164\\
3093.92797118848	0.00534884946956387\\
3097.92917166867	0.00533848978076431\\
3101.93037214886	0.00532816348988316\\
3105.93157262905	0.00531787044637816\\
3109.93277310924	0.00530761050057834\\
3113.93397358944	0.00529738350367795\\
3117.93517406963	0.00528718930773027\\
3121.93637454982	0.00527702776564162\\
3125.93757503001	0.0052668987311653\\
3129.9387755102	0.0052568020588957\\
3133.9399759904	0.00524673760426232\\
3137.94117647059	0.00523670522352399\\
3141.94237695078	0.00522670477376303\\
3145.94357743097	0.00521673611287945\\
3149.94477791116	0.00520679909958533\\
3153.94597839136	0.00519689359339908\\
3157.94717887155	0.00518701945463984\\
3161.94837935174	0.00517717654442194\\
3165.94957983193	0.00516736472464932\\
3169.95078031213	0.00515758385801012\\
3173.95198079232	0.00514783380797117\\
3177.95318127251	0.00513811443877266\\
3181.9543817527	0.00512842561542274\\
3185.95558223289	0.00511876720369226\\
3189.95678271309	0.0051091390701095\\
3193.95798319328	0.00509954108195494\\
3197.95918367347	0.00508997310725609\\
3201.96038415366	0.00508043501478237\\
3205.96158463385	0.00507092667403999\\
3209.96278511405	0.00506144795526693\\
3213.96398559424	0.00505199872942792\\
3217.96518607443	0.00504257886820947\\
3221.96638655462	0.00503318824401494\\
3225.96758703481	0.00502382672995966\\
3229.96878751501	0.00501449419986607\\
3233.9699879952	0.0050051905282589\\
3237.97118847539	0.00499591559036044\\
3241.97238895558	0.00498666926208577\\
3245.97358943577	0.00497745142003804\\
3249.97478991597	0.0049682619415039\\
3253.97599039616	0.00495910070444878\\
3257.97719087635	0.00494996758751238\\
3261.97839135654	0.00494086247000407\\
3265.97959183673	0.00493178523189842\\
3269.98079231693	0.00492273575383069\\
3273.98199279712	0.00491371391709242\\
3277.98319327731	0.004904719603627\\
3281.9843937575	0.00489575269602533\\
3285.9855942377	0.00488681307752145\\
3289.98679471789	0.00487790063198826\\
3293.98799519808	0.00486901524393326\\
3297.98919567827	0.0048601567984943\\
3301.99039615846	0.00485132518143537\\
3305.99159663866	0.00484252027914249\\
3309.99279711885	0.00483374197861949\\
3313.99399759904	0.00482499016748398\\
3317.99519807923	0.00481626473396325\\
3321.99639855942	0.00480756556689024\\
3325.99759903962	0.00479889255569951\\
3329.99879951981	0.00479024559042331\\
3334	0.0047816245616876\\
3338.00120048019	0.00477302936070816\\
3342.00240096038	0.00476445987928667\\
3346.00360144058	0.0047559160098069\\
3350.00480192077	0.00474739764523088\\
3354.00600240096	0.00473890467909508\\
3358.00720288115	0.00473043700550667\\
3362.00840336134	0.00472199451913977\\
3366.00960384154	0.00471357711523178\\
3370.01080432173	0.00470518468957966\\
3374.01200480192	0.00469681713853628\\
3378.01320528211	0.00468847435900684\\
3382.0144057623	0.00468015624844525\\
3386.0156062425	0.00467186270485057\\
3390.01680672269	0.00466359362676346\\
3394.01800720288	0.0046553489132627\\
3398.01920768307	0.00464712846396166\\
3402.02040816327	0.00463893217900489\\
3406.02160864346	0.00463075995906466\\
3410.02280912365	0.00462261170533757\\
3414.02400960384	0.00461448731954115\\
3418.02521008403	0.00460638670391053\\
3422.02641056423	0.00459830976119511\\
3426.02761104442	0.00459025639465524\\
3430.02881152461	0.00458222650805896\\
3434.0300120048	0.00457422000567873\\
3438.03121248499	0.00456623679228824\\
3442.03241296519	0.00455827677315915\\
3446.03361344538	0.00455033985405796\\
3450.03481392557	0.00454242594124281\\
3454.03601440576	0.00453453494146042\\
3458.03721488595	0.00452666676194289\\
3462.03841536615	0.00451882131040467\\
3466.03961584634	0.0045109984950395\\
3470.04081632653	0.00450319822451736\\
3474.04201680672	0.00449542040798145\\
3478.04321728691	0.00448766495504519\\
3482.04441776711	0.00447993177578927\\
3486.0456182473	0.0044722207807587\\
3490.04681872749	0.00446453188095984\\
3494.04801920768	0.00445686498785758\\
3498.04921968787	0.00444922001337235\\
3502.05042016807	0.00444159686987735\\
3506.05162064826	0.00443399547019566\\
3510.05282112845	0.00442641572759744\\
3514.05402160864	0.00441885755579709\\
3518.05522208884	0.00441132086895056\\
3522.05642256903	0.00440380558165248\\
3526.05762304922	0.00439631160893352\\
3530.05882352941	0.00438883886625761\\
3534.0600240096	0.00438138726951927\\
3538.0612244898	0.00437395673504094\\
3542.06242496999	0.00436654717957028\\
3546.06362545018	0.00435915852027759\\
3550.06482593037	0.00435179067475315\\
3554.06602641056	0.00434444356100462\\
3558.06722689076	0.0043371170974545\\
3562.06842737095	0.0043298112029375\\
3566.06962785114	0.00432252579669808\\
3570.07082833133	0.00431526079838784\\
3574.07202881152	0.00430801612806307\\
3578.07322929172	0.00430079170618226\\
3582.07442977191	0.00429358745360359\\
3586.0756302521	0.00428640329158253\\
3590.07683073229	0.00427923914176935\\
3594.07803121248	0.00427209492620676\\
3598.07923169268	0.00426497056732747\\
3602.08043217287	0.00425786598795181\\
3606.08163265306	0.00425078111128538\\
3610.08283313325	0.00424371586091672\\
3614.08403361345	0.00423667016081493\\
3618.08523409364	0.0042296439353274\\
3622.08643457383	0.00422263710917747\\
3626.08763505402	0.0042156496074622\\
3630.08883553421	0.00420868135565006\\
3634.09003601441	0.00420173227957871\\
3638.0912364946	0.00419480230545276\\
3642.09243697479	0.00418789135984154\\
3646.09363745498	0.00418099936967692\\
3650.09483793517	0.0041741262622511\\
3654.09603841537	0.00416727196521447\\
3658.09723889556	0.00416043640657344\\
3662.09843937575	0.00415361951468828\\
3666.09963985594	0.00414682121827105\\
3670.10084033613	0.00414004144638345\\
3674.10204081633	0.00413328012843471\\
3678.10324129652	0.00412653719417957\\
3682.10444177671	0.00411981257371616\\
3686.1056422569	0.004113106197484\\
3690.10684273709	0.0041064179962619\\
3694.10804321729	0.00409974790116601\\
3698.10924369748	0.00409309584364776\\
3702.11044417767	0.0040864617554919\\
3706.11164465786	0.00407984556881449\\
3710.11284513806	0.00407324721606098\\
3714.11404561825	0.00406666663000421\\
3718.11524609844	0.0040601037437425\\
3722.11644657863	0.00405355849069774\\
3726.11764705882	0.00404703080461343\\
3730.11884753902	0.00404052061955284\\
3734.12004801921	0.00403402786989709\\
3738.1212484994	0.00402755249034329\\
3742.12244897959	0.00402109441590267\\
3746.12364945978	0.00401465358189878\\
3750.12484993998	0.00400822992396558\\
3754.12605042017	0.0040018233780457\\
3758.12725090036	0.00399543388038857\\
3762.12845138055	0.00398906136754869\\
3766.12965186074	0.00398270577638378\\
3770.13085234094	0.00397636704405305\\
3774.13205282113	0.00397004510801544\\
3778.13325330132	0.00396373990602786\\
3782.13445378151	0.00395745137614349\\
3786.1356542617	0.00395117945671\\
3790.1368547419	0.0039449240863679\\
3794.13805522209	0.00393868520404881\\
3798.13925570228	0.00393246274897377\\
3802.14045618247	0.00392625666065159\\
3806.14165666266	0.00392006687887716\\
3810.14285714286	0.00391389334372983\\
3814.14405762305	0.00390773599557173\\
3818.14525810324	0.00390159477504616\\
3822.14645858343	0.00389546962307598\\
3826.14765906363	0.003889360480862\\
3830.14885954382	0.00388326728988136\\
3834.15006002401	0.00387718999188598\\
3838.1512605042	0.00387112852890094\\
3842.15246098439	0.00386508284322298\\
3846.15366146459	0.00385905287741888\\
3850.15486194478	0.00385303857432394\\
3854.15606242497	0.00384703987704049\\
3858.15726290516	0.00384105672893628\\
3862.15846338535	0.00383508907364307\\
3866.15966386555	0.00382913685505502\\
3870.16086434574	0.0038232000173273\\
3874.16206482593	0.00381727850487453\\
3878.16326530612	0.00381137226236937\\
3882.16446578631	0.00380548123474099\\
3886.16566626651	0.00379960536717369\\
3890.1668667467	0.00379374460510539\\
3894.16806722689	0.00378789889422626\\
3898.16926770708	0.00378206818047724\\
3902.17046818727	0.00377625241004866\\
3906.17166866747	0.00377045152937884\\
3910.17286914766	0.00376466548515265\\
3914.17406962785	0.00375889422430019\\
3918.17527010804	0.00375313769399536\\
3922.17647058824	0.0037473958416545\\
3926.17767106843	0.00374166861493508\\
3930.17887154862	0.00373595596173426\\
3934.18007202881	0.00373025783018766\\
3938.181272509	0.00372457416866791\\
3942.1824729892	0.00371890492578344\\
3946.18367346939	0.00371325005037707\\
3950.18487394958	0.00370760949152478\\
3954.18607442977	0.00370198319853436\\
3958.18727490996	0.00369637112094414\\
3962.18847539016	0.00369077320852173\\
3966.18967587035	0.0036851894112627\\
3970.19087635054	0.00367961967938937\\
3974.19207683073	0.00367406396334952\\
3978.19327731092	0.00366852221381515\\
3982.19447779112	0.00366299438168126\\
3986.19567827131	0.00365748041806459\\
3990.1968787515	0.00365198027430241\\
3994.19807923169	0.00364649390195131\\
3998.19927971188	0.00364102125278599\\
4002.20048019208	0.00363556227879806\\
4006.20168067227	0.00363011693219486\\
4010.20288115246	0.00362468516539824\\
4014.20408163265	0.00361926693104344\\
4018.20528211285	0.00361386218197786\\
4022.20648259304	0.00360847087125996\\
4026.20768307323	0.00360309295215806\\
4030.20888355342	0.00359772837814922\\
4034.21008403361	0.00359237710291809\\
4038.21128451381	0.00358703908035577\\
4042.212484994	0.00358171426455874\\
4046.21368547419	0.00357640260982765\\
4050.21488595438	0.00357110407066629\\
4054.21608643457	0.00356581860178046\\
4058.21728691477	0.00356054615807688\\
4062.21848739496	0.00355528669466208\\
4066.21968787515	0.00355004016684135\\
4070.22088835534	0.00354480653011764\\
4074.22208883553	0.00353958574019052\\
4078.22328931573	0.00353437775295509\\
4082.22448979592	0.00352918252450095\\
4086.22569027611	0.00352400001111113\\
4090.2268907563	0.00351883016926108\\
4094.22809123649	0.00351367295561759\\
4098.22929171669	0.00350852832703781\\
4102.23049219688	0.00350339624056821\\
4106.23169267707	0.00349827665344357\\
4110.23289315726	0.00349316952308595\\
4114.23409363745	0.00348807480710371\\
4118.23529411765	0.00348299246329052\\
4122.23649459784	0.00347792244962436\\
4126.23769507803	0.00347286472426654\\
4130.23889555822	0.0034678192455607\\
4134.24009603842	0.00346278597203189\\
4138.24129651861	0.00345776486238556\\
4142.2424969988	0.00345275587550663\\
4146.24369747899	0.00344775897045851\\
4150.24489795918	0.0034427741064822\\
4154.24609843938	0.00343780124299529\\
4158.24729891957	0.00343284033959106\\
4162.24849939976	0.00342789135603757\\
4166.24969987995	0.00342295425227669\\
4170.25090036014	0.00341802898842321\\
4174.25210084034	0.00341311552476393\\
4178.25330132053	0.00340821382175675\\
4182.25450180072	0.00340332384002977\\
4186.25570228091	0.00339844554038039\\
4190.2569027611	0.00339357888377444\\
4194.2581032413	0.00338872383134525\\
4198.25930372149	0.00338388034439285\\
4202.26050420168	0.00337904838438301\\
4206.26170468187	0.00337422791294644\\
4210.26290516207	0.0033694188918779\\
4214.26410564226	0.00336462128313534\\
4218.26530612245	0.00335983504883905\\
4222.26650660264	0.00335506015127083\\
4226.26770708283	0.00335029655287313\\
4230.26890756303	0.00334554421624825\\
4234.27010804322	0.00334080310415744\\
4238.27130852341	0.00333607317952014\\
4242.2725090036	0.00333135440541315\\
4246.27370948379	0.00332664674506977\\
4250.27490996399	0.00332195016187904\\
4254.27611044418	0.00331726461938492\\
4258.27731092437	0.00331259008128547\\
4262.27851140456	0.00330792651143209\\
4266.27971188475	0.00330327387382869\\
4270.28091236495	0.00329863213263093\\
4274.28211284514	0.00329400125214545\\
4278.28331332533	0.00328938119682903\\
4282.28451380552	0.00328477193128792\\
4286.28571428571	0.00328017342027698\\
4290.28691476591	0.00327558562869897\\
4294.2881152461	0.00327100852160377\\
4298.28931572629	0.00326644206418764\\
4302.29051620648	0.00326188622179246\\
4306.29171668667	0.003257340959905\\
4310.29291716687	0.00325280624415617\\
4314.29411764706	0.00324828204032027\\
4318.29531812725	0.0032437683143143\\
4322.29651860744	0.00323926503219719\\
4326.29771908764	0.00323477216016912\\
4330.29891956783	0.00323028966457077\\
4334.30012004802	0.00322581751188261\\
4338.30132052821	0.0032213556687242\\
4342.3025210084	0.0032169041018535\\
4346.3037214886	0.00321246277816613\\
4350.30492196879	0.00320803166469471\\
4354.30612244898	0.00320361072860816\\
4358.30732292917	0.00319919993721098\\
4362.30852340936	0.00319479925794263\\
4366.30972388956	0.00319040865837679\\
4370.31092436975	0.0031860281062207\\
4374.31212484994	0.00318165756931451\\
4378.31332533013	0.0031772970156306\\
4382.31452581032	0.0031729464132729\\
4386.31572629052	0.00316860573047624\\
4390.31692677071	0.00316427493560572\\
4394.3181272509	0.00315995399715602\\
4398.31932773109	0.00315564288375076\\
4402.32052821128	0.00315134156414187\\
4406.32172869148	0.00314705000720895\\
4410.32292917167	0.00314276818195862\\
4414.32412965186	0.00313849605752387\\
4418.32533013205	0.0031342336031635\\
4422.32653061224	0.0031299807882614\\
4426.32773109244	0.00312573758232599\\
4430.32893157263	0.0031215039549896\\
4434.33013205282	0.00311727987600782\\
4438.33133253301	0.00311306531525893\\
4442.33253301321	0.00310886024274326\\
4446.3337334934	0.0031046646285826\\
4450.33493397359	0.00310047844301961\\
4454.33613445378	0.0030963016564172\\
4458.33733493397	0.00309213423925794\\
4462.33853541417	0.0030879761621435\\
4466.33973589436	0.00308382739579403\\
4470.34093637455	0.00307968791104757\\
4474.34213685474	0.00307555767885953\\
4478.34333733493	0.00307143667030203\\
4482.34453781513	0.00306732485656339\\
4486.34573829532	0.00306322220894754\\
4490.34693877551	0.00305912869887344\\
4494.3481392557	0.00305504429787455\\
4498.34933973589	0.00305096897759822\\
4502.35054021609	0.00304690270980519\\
4506.35174069628	0.003042845466369\\
4510.35294117647	0.00303879721927544\\
4514.35414165666	0.00303475794062201\\
4518.35534213686	0.00303072760261741\\
4522.35654261705	0.00302670617758093\\
4526.35774309724	0.00302269363794196\\
4530.35894357743	0.00301868995623945\\
4534.36014405762	0.00301469510512138\\
4538.36134453782	0.00301070905734419\\
4542.36254501801	0.00300673178577234\\
4546.3637454982	0.00300276326337769\\
4550.36494597839	0.00299880346323906\\
4554.36614645858	0.00299485235854165\\
4558.36734693878	0.00299090992257659\\
4562.36854741897	0.00298697612874037\\
4566.36974789916	0.00298305095053437\\
4570.37094837935	0.00297913436156434\\
4574.37214885954	0.0029752263355399\\
4578.37334933974	0.00297132684627404\\
4582.37454981993	0.00296743586768264\\
4586.37575030012	0.00296355337378394\\
4590.37695078031	0.00295967933869809\\
4594.3781512605	0.00295581373664664\\
4598.3793517407	0.00295195654195204\\
4602.38055222089	0.0029481077290372\\
4606.38175270108	0.00294426727242498\\
4610.38295318127	0.00294043514673771\\
4614.38415366146	0.00293661132669673\\
4618.38535414166	0.00293279578712191\\
4622.38655462185	0.0029289885029312\\
4626.38775510204	0.00292518944914012\\
4630.38895558223	0.00292139860086136\\
4634.39015606243	0.00291761593330424\\
4638.39135654262	0.00291384142177432\\
4642.39255702281	0.00291007504167293\\
4646.393757503	0.00290631676849667\\
4650.39495798319	0.00290256657783703\\
4654.39615846339	0.00289882444537988\\
4658.39735894358	0.00289509034690505\\
4662.39855942377	0.00289136425828591\\
4666.39975990396	0.00288764615548888\\
4670.40096038415	0.00288393601457303\\
4674.40216086435	0.00288023381168962\\
4678.40336134454	0.00287653952308169\\
4682.40456182473	0.00287285312508362\\
4686.40576230492	0.00286917459412067\\
4690.40696278511	0.0028655039067086\\
4694.40816326531	0.00286184103945323\\
4698.4093637455	0.00285818596905001\\
4702.41056422569	0.0028545386722836\\
4706.41176470588	0.00285089912602746\\
4710.41296518607	0.00284726730724345\\
4714.41416566627	0.00284364319298137\\
4718.41536614646	0.00284002676037861\\
4722.41656662665	0.0028364179866597\\
4726.41776710684	0.00283281684913592\\
4730.41896758703	0.00282922332520488\\
4734.42016806723	0.00282563739235016\\
4738.42136854742	0.00282205902814086\\
4742.42256902761	0.00281848821023124\\
4746.4237695078	0.0028149249163603\\
4750.424969988	0.00281136912435141\\
4754.42617046819	0.0028078208121119\\
4758.42737094838	0.00280427995763268\\
4762.42857142857	0.00280074653898788\\
4766.42977190876	0.0027972205343344\\
4770.43097238896	0.0027937019219116\\
4774.43217286915	0.00279019068004087\\
4778.43337334934	0.00278668678712529\\
4782.43457382953	0.0027831902216492\\
4786.43577430972	0.00277970096217791\\
4790.43697478992	0.00277621898735724\\
4794.43817527011	0.0027727442759132\\
4798.4393757503	0.00276927680665163\\
4802.44057623049	0.00276581655845779\\
4806.44177671068	0.00276236351029605\\
4810.44297719088	0.00275891764120948\\
4814.44417767107	0.00275547893031952\\
4818.44537815126	0.00275204735682561\\
4822.44657863145	0.00274862290000485\\
4826.44777911165	0.00274520553921161\\
4830.44897959184	0.00274179525387723\\
4834.45018007203	0.00273839202350963\\
4838.45138055222	0.00273499582769296\\
4842.45258103241	0.00273160664608728\\
4846.45378151261	0.00272822445842821\\
4850.4549819928	0.00272484924452657\\
4854.45618247299	0.00272148098426804\\
4858.45738295318	0.00271811965761285\\
4862.45858343337	0.0027147652445954\\
4866.45978391357	0.00271141772532396\\
4870.46098439376	0.00270807707998033\\
4874.46218487395	0.0027047432888195\\
4878.46338535414	0.00270141633216929\\
4882.46458583433	0.0026980961904301\\
4886.46578631453	0.00269478284407451\\
4890.46698679472	0.00269147627364697\\
4894.46818727491	0.00268817645976353\\
4898.4693877551	0.00268488338311143\\
4902.47058823529	0.00268159702444886\\
4906.47178871549	0.00267831736460459\\
4910.47298919568	0.00267504438447769\\
4914.47418967587	0.00267177806503718\\
4918.47539015606	0.00266851838732175\\
4922.47659063625	0.00266526533243942\\
4926.47779111645	0.00266201888156727\\
4930.47899159664	0.00265877901595108\\
4934.48019207683	0.00265554571690505\\
4938.48139255702	0.00265231896581151\\
4942.48259303722	0.0026490987441206\\
4946.48379351741	0.00264588503334996\\
4950.4849939976	0.00264267781508445\\
4954.48619447779	0.00263947707097582\\
4958.48739495798	0.00263628278274247\\
4962.48859543818	0.0026330949321691\\
4966.48979591837	0.00262991350110642\\
4970.49099639856	0.0026267384714709\\
4974.49219687875	0.00262356982524444\\
4978.49339735894	0.0026204075444741\\
4982.49459783914	0.00261725161127179\\
4986.49579831933	0.00261410200781402\\
4990.49699879952	0.00261095871634157\\
4994.49819927971	0.00260782171915925\\
4998.4993997599	0.00260469099863561\\
5002.5006002401	0.00260156653720261\\
5006.50180072029	0.00259844831735542\\
5010.50300120048	0.00259533632165209\\
5014.50420168067	0.00259223053271327\\
5018.50540216086	0.00258913093322197\\
5022.50660264106	0.00258603750592328\\
5026.50780312125	0.00258295023362405\\
5030.50900360144	0.00257986909919271\\
5034.51020408163	0.0025767940855589\\
5038.51140456182	0.00257372517571327\\
5042.51260504202	0.00257066235270721\\
5046.51380552221	0.00256760559965254\\
5050.5150060024	0.0025645548997213\\
5054.51620648259	0.00256151023614546\\
5058.51740696278	0.00255847159221665\\
5062.51860744298	0.00255543895128594\\
5066.51980792317	0.00255241229676353\\
5070.52100840336	0.00254939161211854\\
5074.52220888355	0.00254637688087872\\
5078.52340936375	0.00254336808663022\\
5082.52460984394	0.00254036521301732\\
5086.52581032413	0.00253736824374219\\
5090.52701080432	0.00253437716256465\\
5094.52821128451	0.00253139195330189\\
5098.52941176471	0.00252841259982823\\
5102.5306122449	0.0025254390860749\\
5106.53181272509	0.00252247139602979\\
5110.53301320528	0.00251950951373715\\
5114.53421368547	0.00251655342329745\\
5118.53541416567	0.00251360310886702\\
5122.53661464586	0.00251065855465792\\
5126.53781512605	0.00250771974493763\\
5130.53901560624	0.00250478666402881\\
5134.54021608643	0.00250185929630913\\
5138.54141656663	0.00249893762621096\\
5142.54261704682	0.00249602163822118\\
5146.54381752701	0.00249311131688093\\
5150.5450180072	0.00249020664678539\\
5154.5462184874	0.00248730761258352\\
5158.54741896759	0.00248441419897788\\
5162.54861944778	0.00248152639072434\\
5166.54981992797	0.00247864417263193\\
5170.55102040816	0.00247576752956253\\
5174.55222088835	0.00247289644643072\\
5178.55342136855	0.00247003090820348\\
5182.55462184874	0.00246717089990005\\
5186.55582232893	0.00246431640659164\\
5190.55702280912	0.00246146741340126\\
5194.55822328932	0.00245862390550346\\
5198.55942376951	0.00245578586812413\\
5202.5606242497	0.00245295328654029\\
5206.56182472989	0.00245012614607987\\
5210.56302521008	0.00244730443212148\\
5214.56422569028	0.00244448813009422\\
5218.56542617047	0.00244167722547744\\
5222.56662665066	0.00243887170380056\\
5226.56782713085	0.00243607155064283\\
5230.56902761104	0.00243327675163313\\
5234.57022809124	0.00243048729244978\\
5238.57142857143	0.00242770315882029\\
5242.57262905162	0.00242492433652122\\
5246.57382953181	0.0024221508113779\\
5250.575030012	0.00241938256926427\\
5254.5762304922	0.00241661959610266\\
5258.57743097239	0.00241386187786361\\
5262.57863145258	0.00241110940056565\\
5266.57983193277	0.00240836215027507\\
5270.58103241297	0.00240562011310581\\
5274.58223289316	0.00240288327521915\\
5278.58343337335	0.00240015162282358\\
5282.58463385354	0.00239742514217462\\
5286.58583433373	0.00239470381957456\\
5290.58703481393	0.00239198764137231\\
5294.58823529412	0.00238927659396319\\
5298.58943577431	0.00238657066378876\\
5302.5906362545	0.00238386983733659\\
5306.59183673469	0.00238117410114011\\
5310.59303721489	0.00237848344177838\\
5314.59423769508	0.00237579784587594\\
5318.59543817527	0.00237311730010259\\
5322.59663865546	0.00237044179117323\\
5326.59783913565	0.00236777130584765\\
5330.59903961585	0.00236510583093034\\
5334.60024009604	0.00236244535327036\\
5338.60144057623	0.00235978985976107\\
5342.60264105642	0.00235713933734003\\
5346.60384153661	0.00235449377298877\\
5350.60504201681	0.00235185315373261\\
5354.606242497	0.00234921746664051\\
5358.60744297719	0.00234658669882486\\
5362.60864345738	0.00234396083744132\\
5366.60984393757	0.00234133986968863\\
5370.61104441777	0.00233872378280845\\
5374.61224489796	0.00233611256408516\\
5378.61344537815	0.00233350620084572\\
5382.61464585834	0.00233090468045946\\
5386.61584633854	0.00232830799033792\\
5390.61704681873	0.0023257161179347\\
5394.61824729892	0.00232312905074524\\
5398.61944777911	0.0023205467763067\\
5402.6206482593	0.00231796928219777\\
5406.6218487395	0.00231539655603849\\
5410.62304921969	0.00231282858549008\\
5414.62424969988	0.0023102653582548\\
5418.62545018007	0.00230770686207577\\
5422.62665066026	0.00230515308473679\\
5426.62785114046	0.00230260401406219\\
5430.62905162065	0.00230005963791666\\
5434.63025210084	0.0022975199442051\\
5438.63145258103	0.00229498492087243\\
5442.63265306122	0.00229245455590346\\
5446.63385354142	0.0022899288373227\\
5450.63505402161	0.00228740775319422\\
5454.6362545018	0.00228489129162148\\
5458.63745498199	0.00228237944074717\\
5462.63865546218	0.00227987218875307\\
5466.63985594238	0.00227736952385987\\
5470.64105642257	0.00227487143432702\\
5474.64225690276	0.00227237790845259\\
5478.64345738295	0.00226988893457308\\
5482.64465786314	0.00226740450106331\\
5486.64585834334	0.00226492459633624\\
5490.64705882353	0.0022624492088428\\
5494.64825930372	0.00225997832707178\\
5498.64945978391	0.00225751193954966\\
5502.65066026411	0.00225505003484043\\
5506.6518607443	0.00225259260154551\\
5510.65306122449	0.00225013962830351\\
5514.65426170468	0.00224769110379017\\
5518.65546218487	0.00224524701671814\\
5522.65666266507	0.00224280735583688\\
5526.65786314526	0.0022403721099325\\
5530.65906362545	0.00223794126782761\\
5534.66026410564	0.00223551481838116\\
5538.66146458583	0.00223309275048833\\
5542.66266506603	0.00223067505308036\\
5546.66386554622	0.00222826171512443\\
5550.66506602641	0.00222585272562347\\
5554.6662665066	0.0022234480736161\\
5558.66746698679	0.00222104774817639\\
5562.66866746699	0.0022186517384138\\
5566.66986794718	0.002216260033473\\
5570.67106842737	0.00221387262253375\\
5574.67226890756	0.00221148949481074\\
5578.67346938776	0.00220911063955346\\
5582.67466986795	0.00220673604604609\\
5586.67587034814	0.00220436570360733\\
5590.67707082833	0.00220199960159026\\
5594.67827130852	0.00219963772938223\\
5598.67947178872	0.00219728007640473\\
5602.68067226891	0.00219492663211322\\
5606.6818727491	0.00219257738599703\\
5610.68307322929	0.0021902323275792\\
5614.68427370948	0.00218789144641639\\
5618.68547418968	0.00218555473209868\\
5622.68667466987	0.00218322217424953\\
5626.68787515006	0.00218089376252555\\
5630.68907563025	0.00217856948661646\\
5634.69027611044	0.00217624933624489\\
5638.69147659064	0.00217393330116631\\
5642.69267707083	0.00217162137116886\\
5646.69387755102	0.00216931353607323\\
5650.69507803121	0.00216700978573255\\
5654.6962785114	0.00216471011003226\\
5658.6974789916	0.00216241449888997\\
5662.69867947179	0.00216012294225533\\
5666.69987995198	0.00215783543010995\\
5670.70108043217	0.00215555195246722\\
5674.70228091236	0.00215327249937223\\
5678.70348139256	0.0021509970609016\\
5682.70468187275	0.00214872562716343\\
5686.70588235294	0.00214645818829709\\
5690.70708283313	0.00214419473447318\\
5694.70828331333	0.00214193525589335\\
5698.70948379352	0.00213967974279022\\
5702.71068427371	0.00213742818542723\\
5706.7118847539	0.00213518057409855\\
5710.71308523409	0.00213293689912892\\
5714.71428571429	0.00213069715087361\\
5718.71548619448	0.00212846131971819\\
5722.71668667467	0.00212622939607852\\
5726.71788715486	0.00212400137040057\\
5730.71908763505	0.00212177723316032\\
5734.72028811525	0.00211955697486367\\
5738.72148859544	0.00211734058604628\\
5742.72268907563	0.00211512805727349\\
5746.72388955582	0.00211291937914019\\
5750.72509003601	0.00211071454227073\\
5754.72629051621	0.00210851353731875\\
5758.7274909964	0.00210631635496716\\
5762.72869147659	0.00210412298592794\\
5766.72989195678	0.00210193342094207\\
5770.73109243697	0.00209974765077944\\
5774.73229291717	0.00209756566623867\\
5778.73349339736	0.00209538745814709\\
5782.73469387755	0.00209321301736057\\
5786.73589435774	0.00209104233476341\\
5790.73709483793	0.00208887540126827\\
5794.73829531813	0.00208671220781604\\
5798.73949579832	0.00208455274537572\\
5802.74069627851	0.00208239700494435\\
5806.7418967587	0.00208024497754687\\
5810.7430972389	0.00207809665423602\\
5814.74429771909	0.00207595202609226\\
5818.74549819928	0.00207381108422362\\
5822.74669867947	0.00207167381976564\\
5826.74789915966	0.00206954022388126\\
5830.74909963986	0.00206741028776066\\
5834.75030012005	0.00206528400262126\\
5838.75150060024	0.00206316135970751\\
5842.75270108043	0.00206104235029086\\
5846.75390156062	0.00205892696566965\\
5850.75510204082	0.00205681519716897\\
5854.75630252101	0.00205470703614059\\
5858.7575030012	0.00205260247396285\\
5862.75870348139	0.00205050150204059\\
5866.75990396158	0.002048404111805\\
5870.76110444178	0.00204631029471355\\
5874.76230492197	0.00204422004224988\\
5878.76350540216	0.00204213334592374\\
5882.76470588235	0.00204005019727084\\
5886.76590636255	0.00203797058785276\\
5890.76710684274	0.00203589450925691\\
5894.76830732293	0.00203382195309636\\
5898.76950780312	0.00203175291100977\\
5902.77070828331	0.00202968737466134\\
5906.77190876351	0.00202762533574064\\
5910.7731092437	0.00202556678596256\\
5914.77430972389	0.00202351171706722\\
5918.77551020408	0.00202146012081985\\
5922.77671068427	0.00201941198901071\\
5926.77791116447	0.00201736731345501\\
5930.77911164466	0.00201532608599279\\
5934.78031212485	0.00201328829848884\\
5938.78151260504	0.00201125394283264\\
5942.78271308523	0.00200922301093821\\
5946.78391356543	0.00200719549474406\\
5950.78511404562	0.00200517138621308\\
5954.78631452581	0.00200315067733247\\
5958.787515006	0.00200113336011363\\
5962.78871548619	0.00199911942659208\\
5966.78991596639	0.00199710886882737\\
5970.79111644658	0.001995101678903\\
5974.79231692677	0.00199309784892632\\
5978.79351740696	0.00199109737102843\\
5982.79471788715	0.00198910023736413\\
5986.79591836735	0.00198710644011179\\
5990.79711884754	0.00198511597147331\\
5994.79831932773	0.00198312882367398\\
5998.79951980792	0.00198114498896245\\
6002.80072028812	0.00197916445961059\\
};
\addplot [color=black,solid,line width=1.4pt,forget plot]
  table[row sep=crcr]{%
6002.80072028812	0.00197916445961059\\
6006.80192076831	0.00197718722791345\\
6010.8031212485	0.00197521328618915\\
6014.80432172869	0.00197324262677881\\
6018.80552220888	0.00197127524204645\\
6022.80672268908	0.00196931112437892\\
6026.80792316927	0.0019673502661858\\
6030.80912364946	0.00196539265989933\\
6034.81032412965	0.00196343829797435\\
6038.81152460984	0.00196148717288817\\
6042.81272509004	0.00195953927714052\\
6046.81392557023	0.00195759460325345\\
6050.81512605042	0.00195565314377127\\
6054.81632653061	0.00195371489126045\\
6058.8175270108	0.00195177983830956\\
6062.818727491	0.00194984797752916\\
6066.81992797119	0.00194791930155176\\
6070.82112845138	0.0019459938030317\\
6074.82232893157	0.00194407147464508\\
6078.82352941176	0.00194215230908972\\
6082.82472989196	0.00194023629908503\\
6086.82593037215	0.00193832343737195\\
6090.82713085234	0.0019364137167129\\
6094.82833133253	0.00193450712989166\\
6098.82953181272	0.0019326036697133\\
6102.83073229292	0.00193070332900415\\
6106.83193277311	0.00192880610061165\\
6110.8331332533	0.00192691197740433\\
6114.83433373349	0.00192502095227171\\
6118.83553421369	0.00192313301812425\\
6122.83673469388	0.00192124816789322\\
6126.83793517407	0.0019193663945307\\
6130.83913565426	0.00191748769100943\\
6134.84033613445	0.00191561205032279\\
6138.84153661465	0.00191373946548471\\
6142.84273709484	0.00191186992952959\\
6146.84393757503	0.00191000343551222\\
6150.84513805522	0.00190813997650773\\
6154.84633853541	0.0019062795456115\\
6158.84753901561	0.00190442213593911\\
6162.8487394958	0.00190256774062623\\
6166.84993997599	0.00190071635282856\\
6170.85114045618	0.0018988679657218\\
6174.85234093637	0.00189702257250152\\
6178.85354141657	0.00189518016638313\\
6182.85474189676	0.00189334074060179\\
6186.85594237695	0.00189150428841233\\
6190.85714285714	0.00188967080308922\\
6194.85834333734	0.00188784027792647\\
6198.85954381753	0.00188601270623754\\
6202.86074429772	0.00188418808135533\\
6206.86194477791	0.00188236639663207\\
6210.8631452581	0.00188054764543923\\
6214.8643457383	0.00187873182116753\\
6218.86554621849	0.00187691891722678\\
6222.86674669868	0.00187510892704587\\
6226.86794717887	0.0018733018440727\\
6230.86914765906	0.00187149766177409\\
6234.87034813926	0.00186969637363573\\
6238.87154861945	0.00186789797316209\\
6242.87274909964	0.0018661024538764\\
6246.87394957983	0.00186430980932053\\
6250.87515006002	0.00186252003305496\\
6254.87635054022	0.00186073311865871\\
6258.87755102041	0.00185894905972926\\
6262.8787515006	0.0018571678498825\\
6266.87995198079	0.00185538948275266\\
6270.88115246098	0.00185361395199224\\
6274.88235294118	0.00185184125127194\\
6278.88355342137	0.00185007137428064\\
6282.88475390156	0.00184830431472528\\
6286.88595438175	0.00184654006633083\\
6290.88715486194	0.0018447786228402\\
6294.88835534214	0.00184301997801421\\
6298.88955582233	0.00184126412563152\\
6302.89075630252	0.00183951105948855\\
6306.89195678271	0.00183776077339941\\
6310.89315726291	0.0018360132611959\\
6314.8943577431	0.00183426851672736\\
6318.89555822329	0.00183252653386067\\
6322.89675870348	0.00183078730648019\\
6326.89795918367	0.00182905082848766\\
6330.89915966387	0.00182731709380217\\
6334.90036014406	0.00182558609636009\\
6338.90156062425	0.001823857830115\\
6342.90276110444	0.00182213228903766\\
6346.90396158463	0.00182040946711592\\
6350.90516206483	0.00181868935835468\\
6354.90636254502	0.00181697195677582\\
6358.90756302521	0.00181525725641814\\
6362.9087635054	0.00181354525133731\\
6366.90996398559	0.00181183593560582\\
6370.91116446579	0.00181012930331288\\
6374.91236494598	0.00180842534856443\\
6378.91356542617	0.00180672406548302\\
6382.91476590636	0.00180502544820777\\
6386.91596638655	0.00180332949089435\\
6390.91716686675	0.00180163618771487\\
6394.91836734694	0.00179994553285785\\
6398.91956782713	0.00179825752052817\\
6402.92076830732	0.00179657214494698\\
6406.92196878751	0.00179488940035171\\
6410.92316926771	0.00179320928099593\\
6414.9243697479	0.00179153178114935\\
6418.92557022809	0.00178985689509777\\
6422.92677070828	0.00178818461714299\\
6426.92797118848	0.00178651494160276\\
6430.92917166867	0.00178484786281077\\
6434.93037214886	0.00178318337511653\\
6438.93157262905	0.00178152147288536\\
6442.93277310924	0.00177986215049834\\
6446.93397358944	0.00177820540235221\\
6450.93517406963	0.00177655122285938\\
6454.93637454982	0.00177489960644781\\
6458.93757503001	0.00177325054756103\\
6462.9387755102	0.00177160404065802\\
6466.9399759904	0.00176996008021318\\
6470.94117647059	0.00176831866071631\\
6474.94237695078	0.00176667977667251\\
6478.94357743097	0.00176504342260215\\
6482.94477791116	0.00176340959304083\\
6486.94597839136	0.00176177828253931\\
6490.94717887155	0.00176014948566344\\
6494.94837935174	0.00175852319699417\\
6498.94957983193	0.00175689941112743\\
6502.95078031213	0.00175527812267414\\
6506.95198079232	0.00175365932626011\\
6510.95318127251	0.00175204301652601\\
6514.9543817527	0.00175042918812733\\
6518.95558223289	0.00174881783573431\\
6522.95678271309	0.0017472089540319\\
6526.95798319328	0.00174560253771972\\
6530.95918367347	0.001743998581512\\
6534.96038415366	0.00174239708013751\\
6538.96158463385	0.00174079802833957\\
6542.96278511405	0.00173920142087592\\
6546.96398559424	0.00173760725251876\\
6550.96518607443	0.00173601551805462\\
6554.96638655462	0.00173442621228436\\
6558.96758703481	0.00173283933002313\\
6562.96878751501	0.00173125486610028\\
6566.9699879952	0.00172967281535933\\
6570.97118847539	0.00172809317265795\\
6574.97238895558	0.00172651593286788\\
6578.97358943577	0.00172494109087489\\
6582.97478991597	0.00172336864157875\\
6586.97599039616	0.00172179857989313\\
6590.97719087635	0.00172023090074565\\
6594.97839135654	0.00171866559907774\\
6598.97959183673	0.00171710266984463\\
6602.98079231693	0.00171554210801532\\
6606.98199279712	0.00171398390857251\\
6610.98319327731	0.00171242806651257\\
6614.9843937575	0.00171087457684547\\
6618.9855942377	0.00170932343459478\\
6622.98679471789	0.00170777463479757\\
6626.98799519808	0.00170622817250442\\
6630.98919567827	0.00170468404277932\\
6634.99039615846	0.00170314224069966\\
6638.99159663866	0.0017016027613562\\
6642.99279711885	0.00170006559985298\\
6646.99399759904	0.0016985307513073\\
6650.99519807923	0.0016969982108497\\
6654.99639855942	0.00169546797362388\\
6658.99759903962	0.00169394003478665\\
6662.99879951981	0.00169241438950793\\
6667	0.00169089103297069\\
6671.00120048019	0.00168936996037086\\
6675.00240096038	0.00168785116691738\\
6679.00360144058	0.00168633464783206\\
6683.00480192077	0.00168482039834961\\
6687.00600240096	0.00168330841371755\\
6691.00720288115	0.00168179868919619\\
6695.00840336134	0.0016802912200586\\
6699.00960384154	0.00167878600159055\\
6703.01080432173	0.00167728302909045\\
6707.01200480192	0.00167578229786935\\
6711.01320528211	0.00167428380325087\\
6715.0144057623	0.00167278754057118\\
6719.0156062425	0.00167129350517894\\
6723.01680672269	0.00166980169243525\\
6727.01800720288	0.00166831209771364\\
6731.01920768307	0.00166682471640002\\
6735.02040816327	0.00166533954389263\\
6739.02160864346	0.00166385657560198\\
6743.02280912365	0.00166237580695087\\
6747.02400960384	0.00166089723337429\\
6751.02521008403	0.0016594208503194\\
6755.02641056423	0.00165794665324552\\
6759.02761104442	0.00165647463762403\\
6763.02881152461	0.0016550047989384\\
6767.0300120048	0.00165353713268409\\
6771.03121248499	0.00165207163436854\\
6775.03241296519	0.00165060829951114\\
6779.03361344538	0.00164914712364317\\
6783.03481392557	0.00164768810230778\\
6787.03601440576	0.00164623123105993\\
6791.03721488595	0.00164477650546636\\
6795.03841536615	0.00164332392110557\\
6799.03961584634	0.00164187347356777\\
6803.04081632653	0.00164042515845483\\
6807.04201680672	0.00163897897138024\\
6811.04321728692	0.00163753490796912\\
6815.04441776711	0.00163609296385811\\
6819.0456182473	0.00163465313469538\\
6823.04681872749	0.00163321541614061\\
6827.04801920768	0.00163177980386488\\
6831.04921968787	0.00163034629355072\\
6835.05042016807	0.001628914880892\\
6839.05162064826	0.00162748556159395\\
6843.05282112845	0.00162605833137308\\
6847.05402160864	0.00162463318595717\\
6851.05522208884	0.00162321012108524\\
6855.05642256903	0.00162178913250748\\
6859.05762304922	0.00162037021598524\\
6863.05882352941	0.00161895336729099\\
6867.0600240096	0.00161753858220829\\
6871.0612244898	0.00161612585653174\\
6875.06242496999	0.00161471518606695\\
6879.06362545018	0.00161330656663052\\
6883.06482593037	0.00161189999404998\\
6887.06602641056	0.00161049546416377\\
6891.06722689076	0.00160909297282119\\
6895.06842737095	0.00160769251588241\\
6899.06962785114	0.00160629408921837\\
6903.07082833133	0.0016048976887108\\
6907.07202881152	0.00160350331025215\\
6911.07322929172	0.00160211094974557\\
6915.07442977191	0.0016007206031049\\
6919.0756302521	0.00159933226625457\\
6923.07683073229	0.00159794593512965\\
6927.07803121249	0.00159656160567574\\
6931.07923169268	0.00159517927384901\\
6935.08043217287	0.00159379893561608\\
6939.08163265306	0.00159242058695408\\
6943.08283313325	0.00159104422385054\\
6947.08403361345	0.0015896698423034\\
6951.08523409364	0.00158829743832096\\
6955.08643457383	0.00158692700792187\\
6959.08763505402	0.00158555854713505\\
6963.08883553421	0.00158419205199971\\
6967.09003601441	0.0015828275185653\\
6971.0912364946	0.00158146494289144\\
6975.09243697479	0.00158010432104795\\
6979.09363745498	0.00157874564911478\\
6983.09483793517	0.00157738892318199\\
6987.09603841537	0.0015760341393497\\
6991.09723889556	0.00157468129372809\\
6995.09843937575	0.00157333038243734\\
6999.09963985594	0.00157198140160762\\
7003.10084033613	0.00157063434737903\\
7007.10204081633	0.00156928921590161\\
7011.10324129652	0.00156794600333527\\
7015.10444177671	0.00156660470584979\\
7019.1056422569	0.00156526531962474\\
7023.10684273709	0.00156392784084953\\
7027.10804321729	0.0015625922657233\\
7031.10924369748	0.00156125859045494\\
7035.11044417767	0.00155992681126302\\
7039.11164465786	0.0015585969243758\\
7043.11284513806	0.00155726892603119\\
7047.11404561825	0.00155594281247667\\
7051.11524609844	0.00155461857996936\\
7055.11644657863	0.00155329622477587\\
7059.11764705882	0.00155197574317238\\
7063.11884753902	0.00155065713144453\\
7067.12004801921	0.00154934038588744\\
7071.1212484994	0.00154802550280566\\
7075.12244897959	0.00154671247851313\\
7079.12364945978	0.00154540130933319\\
7083.12484993998	0.00154409199159849\\
7087.12605042017	0.00154278452165103\\
7091.12725090036	0.00154147889584208\\
7095.12845138055	0.00154017511053217\\
7099.12965186074	0.00153887316209106\\
7103.13085234094	0.0015375730468977\\
7107.13205282113	0.00153627476134024\\
7111.13325330132	0.00153497830181596\\
7115.13445378151	0.00153368366473123\\
7119.13565426171	0.00153239084650156\\
7123.1368547419	0.00153109984355147\\
7127.13805522209	0.00152981065231455\\
7131.13925570228	0.00152852326923336\\
7135.14045618247	0.00152723769075946\\
7139.14165666266	0.00152595391335335\\
7143.14285714286	0.00152467193348445\\
7147.14405762305	0.00152339174763108\\
7151.14525810324	0.00152211335228041\\
7155.14645858343	0.00152083674392847\\
7159.14765906363	0.00151956191908009\\
7163.14885954382	0.00151828887424888\\
7167.15006002401	0.00151701760595721\\
7171.1512605042	0.0015157481107362\\
7175.15246098439	0.00151448038512564\\
7179.15366146459	0.00151321442567401\\
7183.15486194478	0.00151195022893847\\
7187.15606242497	0.00151068779148475\\
7191.15726290516	0.00150942710988723\\
7195.15846338535	0.00150816818072882\\
7199.15966386555	0.00150691100060101\\
7203.16086434574	0.00150565556610377\\
7207.16206482593	0.00150440187384561\\
7211.16326530612	0.00150314992044347\\
7215.16446578631	0.00150189970252274\\
7219.16566626651	0.00150065121671724\\
7223.1668667467	0.00149940445966916\\
7227.16806722689	0.00149815942802907\\
7231.16926770708	0.00149691611845587\\
7235.17046818728	0.00149567452761678\\
7239.17166866747	0.0014944346521873\\
7243.17286914766	0.0014931964888512\\
7247.17406962785	0.00149196003430048\\
7251.17527010804	0.00149072528523537\\
7255.17647058824	0.00148949223836428\\
7259.17767106843	0.00148826089040376\\
7263.17887154862	0.00148703123807854\\
7267.18007202881	0.00148580327812144\\
7271.181272509	0.00148457700727336\\
7275.1824729892	0.00148335242228329\\
7279.18367346939	0.00148212951990825\\
7283.18487394958	0.00148090829691327\\
7287.18607442977	0.00147968875007137\\
7291.18727490996	0.00147847087616355\\
7295.18847539016	0.00147725467197875\\
7299.18967587035	0.00147604013431383\\
7303.19087635054	0.00147482725997355\\
7307.19207683073	0.00147361604577052\\
7311.19327731092	0.00147240648852523\\
7315.19447779112	0.00147119858506597\\
7319.19567827131	0.00146999233222886\\
7323.1968787515	0.00146878772685776\\
7327.19807923169	0.00146758476580431\\
7331.19927971188	0.00146638344592789\\
7335.20048019208	0.00146518376409556\\
7339.20168067227	0.00146398571718209\\
7343.20288115246	0.00146278930206989\\
7347.20408163265	0.00146159451564903\\
7351.20528211285	0.00146040135481719\\
7355.20648259304	0.00145920981647963\\
7359.20768307323	0.00145801989754921\\
7363.20888355342	0.0014568315949463\\
7367.21008403361	0.00145564490559882\\
7371.21128451381	0.0014544598264422\\
7375.212484994	0.00145327635441933\\
7379.21368547419	0.00145209448648057\\
7383.21488595438	0.00145091421958372\\
7387.21608643457	0.00144973555069398\\
7391.21728691477	0.00144855847678395\\
7395.21848739496	0.00144738299483361\\
7399.21968787515	0.00144620910183027\\
7403.22088835534	0.00144503679476858\\
7407.22208883553	0.00144386607065049\\
7411.22328931573	0.00144269692648522\\
7415.22448979592	0.00144152935928928\\
7419.22569027611	0.0014403633660864\\
7423.2268907563	0.00143919894390753\\
7427.22809123649	0.00143803608979081\\
7431.22929171669	0.00143687480078157\\
7435.23049219688	0.00143571507393229\\
7439.23169267707	0.00143455690630258\\
7443.23289315726	0.00143340029495916\\
7447.23409363745	0.00143224523697585\\
7451.23529411765	0.00143109172943353\\
7455.23649459784	0.00142993976942013\\
7459.23769507803	0.00142878935403062\\
7463.23889555822	0.00142764048036697\\
7467.24009603842	0.00142649314553814\\
7471.24129651861	0.00142534734666004\\
7475.2424969988	0.00142420308085554\\
7479.24369747899	0.00142306034525445\\
7483.24489795918	0.00142191913699346\\
7487.24609843938	0.00142077945321615\\
7491.24729891957	0.00141964129107298\\
7495.24849939976	0.00141850464772124\\
7499.24969987995	0.00141736952032505\\
7503.25090036014	0.00141623590605533\\
7507.25210084034	0.00141510380208979\\
7511.25330132053	0.00141397320561289\\
7515.25450180072	0.00141284411381587\\
7519.25570228091	0.00141171652389665\\
7523.2569027611	0.00141059043305989\\
7527.2581032413	0.00140946583851692\\
7531.25930372149	0.00140834273748574\\
7535.26050420168	0.00140722112719099\\
7539.26170468187	0.00140610100486394\\
7543.26290516207	0.00140498236774249\\
7547.26410564226	0.0014038652130711\\
7551.26530612245	0.0014027495381008\\
7555.26650660264	0.00140163534008919\\
7559.26770708283	0.0014005226163004\\
7563.26890756303	0.00139941136400505\\
7567.27010804322	0.00139830158048027\\
7571.27130852341	0.00139719326300966\\
7575.2725090036	0.00139608640888328\\
7579.27370948379	0.00139498101539764\\
7583.27490996399	0.00139387707985563\\
7587.27611044418	0.00139277459956657\\
7591.27731092437	0.00139167357184616\\
7595.27851140456	0.00139057399401645\\
7599.27971188475	0.00138947586340584\\
7603.28091236495	0.00138837917734907\\
7607.28211284514	0.00138728393318716\\
7611.28331332533	0.00138619012826744\\
7615.28451380552	0.00138509775994351\\
7619.28571428571	0.0013840068255752\\
7623.28691476591	0.00138291732252861\\
7627.2881152461	0.00138182924817604\\
7631.28931572629	0.00138074259989598\\
7635.29051620648	0.00137965737507312\\
7639.29171668667	0.0013785735710983\\
7643.29291716687	0.0013774911853685\\
7647.29411764706	0.00137641021528686\\
7651.29531812725	0.00137533065826258\\
7655.29651860744	0.00137425251171101\\
7659.29771908764	0.00137317577305353\\
7663.29891956783	0.00137210043971759\\
7667.30012004802	0.0013710265091367\\
7671.30132052821	0.00136995397875037\\
7675.3025210084	0.00136888284600412\\
7679.3037214886	0.00136781310834947\\
7683.30492196879	0.0013667447632439\\
7687.30612244898	0.00136567780815085\\
7691.30732292917	0.00136461224053969\\
7695.30852340936	0.00136354805788573\\
7699.30972388956	0.00136248525767016\\
7703.31092436975	0.00136142383738007\\
7707.31212484994	0.00136036379450841\\
7711.31332533013	0.001359305126554\\
7715.31452581032	0.00135824783102148\\
7719.31572629052	0.00135719190542132\\
7723.31692677071	0.00135613734726979\\
7727.3181272509	0.00135508415408893\\
7731.31932773109	0.00135403232340659\\
7735.32052821128	0.00135298185275633\\
7739.32172869148	0.00135193273967748\\
7743.32292917167	0.00135088498171507\\
7747.32412965186	0.00134983857641985\\
7751.32533013205	0.00134879352134825\\
7755.32653061224	0.00134774981406237\\
7759.32773109244	0.00134670745212997\\
7763.32893157263	0.00134566643312446\\
7767.33013205282	0.00134462675462487\\
7771.33133253301	0.00134358841421581\\
7775.33253301321	0.00134255140948753\\
7779.3337334934	0.00134151573803583\\
7783.33493397359	0.00134048139746207\\
7787.33613445378	0.00133944838537316\\
7791.33733493397	0.00133841669938154\\
7795.33853541417	0.00133738633710517\\
7799.33973589436	0.00133635729616749\\
7803.34093637455	0.00133532957419745\\
7807.34213685474	0.00133430316882944\\
7811.34333733493	0.00133327807770333\\
7815.34453781513	0.0013322542984644\\
7819.34573829532	0.00133123182876337\\
7823.34693877551	0.00133021066625636\\
7827.3481392557	0.00132919080860487\\
7831.34933973589	0.0013281722534758\\
7835.35054021609	0.00132715499854139\\
7839.35174069628	0.00132613904147924\\
7843.35294117647	0.00132512437997227\\
7847.35414165666	0.00132411101170872\\
7851.35534213686	0.00132309893438213\\
7855.35654261705	0.00132208814569133\\
7859.35774309724	0.00132107864334042\\
7863.35894357743	0.00132007042503875\\
7867.36014405762	0.00131906348850092\\
7871.36134453782	0.00131805783144675\\
7875.36254501801	0.00131705345160128\\
7879.3637454982	0.00131605034669474\\
7883.36494597839	0.00131504851446256\\
7887.36614645858	0.00131404795264531\\
7891.36734693878	0.00131304865898875\\
7895.36854741897	0.00131205063124375\\
7899.36974789916	0.00131105386716632\\
7903.37094837935	0.00131005836451757\\
7907.37214885954	0.00130906412106374\\
7911.37334933974	0.00130807113457611\\
7915.37454981993	0.00130707940283106\\
7919.37575030012	0.00130608892361001\\
7923.37695078031	0.00130509969469943\\
7927.3781512605	0.00130411171389081\\
7931.3793517407	0.00130312497898065\\
7935.38055222089	0.00130213948777046\\
7939.38175270108	0.00130115523806674\\
7943.38295318127	0.00130017222768093\\
7947.38415366146	0.00129919045442947\\
7951.38535414166	0.00129820991613371\\
7955.38655462185	0.00129723061061995\\
7959.38775510204	0.0012962525357194\\
7963.38895558223	0.00129527568926817\\
7967.39015606243	0.00129430006910727\\
7971.39135654262	0.00129332567308257\\
7975.39255702281	0.00129235249904482\\
7979.393757503	0.00129138054484961\\
7983.39495798319	0.00129040980835737\\
7987.39615846339	0.00128944028743334\\
7991.39735894358	0.00128847197994759\\
7995.39855942377	0.00128750488377497\\
7999.39975990396	0.00128653899679511\\
8003.40096038415	0.00128557431689243\\
8007.40216086435	0.00128461084195609\\
8011.40336134454	0.00128364856988\\
8015.40456182473	0.00128268749856279\\
8019.40576230492	0.00128172762590783\\
8023.40696278511	0.00128076894982316\\
8027.40816326531	0.00127981146822155\\
8031.4093637455	0.00127885517902043\\
8035.41056422569	0.00127790008014188\\
8039.41176470588	0.00127694616951267\\
8043.41296518607	0.00127599344506417\\
8047.41416566627	0.00127504190473242\\
8051.41536614646	0.00127409154645804\\
8055.41656662665	0.00127314236818627\\
8059.41776710684	0.00127219436786694\\
8063.41896758703	0.00127124754345445\\
8067.42016806723	0.00127030189290777\\
8071.42136854742	0.00126935741419041\\
8075.42256902761	0.00126841410527045\\
8079.4237695078	0.00126747196412047\\
8083.424969988	0.00126653098871757\\
8087.42617046819	0.00126559117704337\\
8091.42737094838	0.00126465252708396\\
8095.42857142857	0.00126371503682992\\
8099.42977190876	0.00126277870427629\\
8103.43097238896	0.00126184352742257\\
8107.43217286915	0.00126090950427271\\
8111.43337334934	0.00125997663283507\\
8115.43457382953	0.00125904491112244\\
8119.43577430972	0.00125811433715202\\
8123.43697478992	0.0012571849089454\\
8127.43817527011	0.00125625662452855\\
8131.4393757503	0.00125532948193182\\
8135.44057623049	0.0012544034791899\\
8139.44177671068	0.00125347861434185\\
8143.44297719088	0.00125255488543105\\
8147.44417767107	0.0012516322905052\\
8151.44537815126	0.00125071082761633\\
8155.44657863145	0.00124979049482076\\
8159.44777911165	0.00124887129017909\\
8163.44897959184	0.00124795321175621\\
8167.45018007203	0.00124703625762127\\
8171.45138055222	0.00124612042584768\\
8175.45258103241	0.00124520571451309\\
8179.45378151261	0.00124429212169936\\
8183.4549819928	0.00124337964549261\\
8187.45618247299	0.00124246828398314\\
8191.45738295318	0.00124155803526546\\
8195.45858343337	0.00124064889743824\\
8199.45978391357	0.00123974086860437\\
8203.46098439376	0.00123883394687087\\
8207.46218487395	0.00123792813034891\\
8211.46338535414	0.00123702341715383\\
8215.46458583433	0.00123611980540507\\
8219.46578631453	0.00123521729322621\\
8223.46698679472	0.00123431587874492\\
8227.46818727491	0.00123341556009298\\
8231.4693877551	0.00123251633540625\\
8235.47058823529	0.00123161820282468\\
8239.47178871549	0.00123072116049226\\
8243.47298919568	0.00122982520655704\\
8247.47418967587	0.00122893033917114\\
8251.47539015606	0.00122803655649067\\
8255.47659063625	0.00122714385667578\\
8259.47779111645	0.00122625223789065\\
8263.47899159664	0.00122536169830342\\
8267.48019207683	0.00122447223608626\\
8271.48139255702	0.00122358384941528\\
8275.48259303722	0.00122269653647059\\
8279.48379351741	0.00122181029543624\\
8283.4849939976	0.00122092512450022\\
8287.48619447779	0.00122004102185448\\
8291.48739495798	0.00121915798569489\\
8295.48859543818	0.0012182760142212\\
8299.48979591837	0.00121739510563712\\
8303.49099639856	0.00121651525815022\\
8307.49219687875	0.00121563646997196\\
8311.49339735894	0.00121475873931767\\
8315.49459783914	0.00121388206440657\\
8319.49579831933	0.0012130064434617\\
8323.49699879952	0.00121213187470996\\
8327.49819927971	0.00121125835638208\\
8331.4993997599	0.00121038588671262\\
8335.5006002401	0.00120951446393994\\
8339.50180072029	0.00120864408630621\\
8343.50300120048	0.0012077747520574\\
8347.50420168067	0.00120690645944326\\
8351.50540216086	0.0012060392067173\\
8355.50660264106	0.0012051729921368\\
8359.50780312125	0.00120430781396281\\
8363.50900360144	0.0012034436704601\\
8367.51020408163	0.00120258055989718\\
8371.51140456182	0.0012017184805463\\
8375.51260504202	0.00120085743068339\\
8379.51380552221	0.00119999740858812\\
8383.5150060024	0.00119913841254384\\
8387.51620648259	0.00119828044083757\\
8391.51740696278	0.00119742349176003\\
8395.51860744298	0.00119656756360559\\
8399.51980792317	0.00119571265467229\\
8403.52100840336	0.00119485876326179\\
8407.52220888355	0.0011940058876794\\
8411.52340936375	0.00119315402623408\\
8415.52460984394	0.00119230317723838\\
8419.52581032413	0.00119145333900846\\
8423.52701080432	0.0011906045098641\\
8427.52821128451	0.00118975668812863\\
8431.52941176471	0.00118890987212901\\
8435.5306122449	0.00118806406019572\\
8439.53181272509	0.00118721925066285\\
8443.53301320528	0.00118637544186801\\
8447.53421368547	0.00118553263215236\\
8451.53541416567	0.0011846908198606\\
8455.53661464586	0.00118385000334094\\
8459.53781512605	0.00118301018094512\\
8463.53901560624	0.00118217135102839\\
8467.54021608643	0.00118133351194948\\
8471.54141656663	0.00118049666207062\\
8475.54261704682	0.00117966079975751\\
8479.54381752701	0.00117882592337932\\
8483.5450180072	0.00117799203130871\\
8487.54621848739	0.00117715912192174\\
8491.54741896759	0.00117632719359796\\
8495.54861944778	0.00117549624472033\\
8499.54981992797	0.00117466627367524\\
8503.55102040816	0.00117383727885249\\
8507.55222088836	0.00117300925864531\\
8511.55342136855	0.00117218221145029\\
8515.55462184874	0.00117135613566745\\
8519.55582232893	0.00117053102970017\\
8523.55702280912	0.00116970689195521\\
8527.55822328932	0.00116888372084267\\
8531.55942376951	0.00116806151477605\\
8535.5606242497	0.00116724027217216\\
8539.56182472989	0.00116641999145115\\
8543.56302521008	0.00116560067103653\\
8547.56422569028	0.00116478230935508\\
8551.56542617047	0.00116396490483695\\
8555.56662665066	0.00116314845591554\\
8559.56782713085	0.00116233296102759\\
8563.56902761104	0.00116151841861309\\
8567.57022809124	0.00116070482711533\\
8571.57142857143	0.00115989218498086\\
8575.57262905162	0.0011590804906595\\
8579.57382953181	0.00115826974260431\\
8583.57503001201	0.00115745993927161\\
8587.5762304922	0.00115665107912095\\
8591.57743097239	0.0011558431606151\\
8595.57863145258	0.00115503618222005\\
8599.57983193277	0.00115423014240503\\
8603.58103241297	0.00115342503964243\\
8607.58223289316	0.00115262087240787\\
8611.58343337335	0.00115181763918014\\
8615.58463385354	0.0011510153384412\\
8619.58583433373	0.00115021396867621\\
8623.58703481393	0.00114941352837346\\
8627.58823529412	0.00114861401602442\\
8631.58943577431	0.0011478154301237\\
8635.5906362545	0.00114701776916903\\
8639.59183673469	0.0011462210316613\\
8643.59303721489	0.0011454252161045\\
8647.59423769508	0.00114463032100574\\
8651.59543817527	0.00114383634487524\\
8655.59663865546	0.00114304328622632\\
8659.59783913565	0.00114225114357539\\
8663.59903961585	0.00114145991544194\\
8667.60024009604	0.00114066960034854\\
8671.60144057623	0.00113988019682082\\
8675.60264105642	0.00113909170338747\\
8679.60384153661	0.00113830411858024\\
8683.60504201681	0.00113751744093392\\
8687.606242497	0.00113673166898634\\
8691.60744297719	0.00113594680127834\\
8695.60864345738	0.00113516283635381\\
8699.60984393757	0.00113437977275964\\
8703.61104441777	0.00113359760904571\\
8707.61224489796	0.00113281634376493\\
8711.61344537815	0.00113203597547317\\
8715.61464585834	0.00113125650272931\\
8719.61584633853	0.00113047792409518\\
8723.61704681873	0.0011297002381356\\
8727.61824729892	0.00112892344341834\\
8731.61944777911	0.00112814753851411\\
8735.6206482593	0.0011273725219966\\
8739.6218487395	0.00112659839244241\\
8743.62304921969	0.00112582514843107\\
8747.62424969988	0.00112505278854506\\
8751.62545018007	0.00112428131136974\\
8755.62665066026	0.00112351071549341\\
8759.62785114046	0.00112274099950725\\
8763.62905162065	0.00112197216200535\\
8767.63025210084	0.00112120420158467\\
8771.63145258103	0.00112043711684507\\
8775.63265306122	0.00111967090638927\\
8779.63385354142	0.00111890556882286\\
8783.63505402161	0.00111814110275427\\
8787.6362545018	0.00111737750679481\\
8791.63745498199	0.00111661477955862\\
8795.63865546218	0.00111585291966267\\
8799.63985594238	0.00111509192572678\\
8803.64105642257	0.00111433179637356\\
8807.64225690276	0.00111357253022846\\
8811.64345738295	0.00111281412591974\\
8815.64465786315	0.00111205658207845\\
8819.64585834334	0.00111129989733844\\
8823.64705882353	0.00111054407033634\\
8827.64825930372	0.00110978909971158\\
8831.64945978391	0.00110903498410634\\
8835.65066026411	0.00110828172216557\\
8839.6518607443	0.001107529312537\\
8843.65306122449	0.0011067777538711\\
8847.65426170468	0.00110602704482108\\
8851.65546218487	0.00110527718404289\\
8855.65666266507	0.00110452817019522\\
8859.65786314526	0.0011037800019395\\
8863.65906362545	0.00110303267793984\\
8867.66026410564	0.00110228619686309\\
8871.66146458583	0.00110154055737881\\
8875.66266506603	0.00110079575815923\\
8879.66386554622	0.00110005179787931\\
8883.66506602641	0.00109930867521666\\
8887.6662665066	0.00109856638885159\\
8891.6674669868	0.00109782493746709\\
8895.66866746699	0.00109708431974878\\
8899.66986794718	0.00109634453438497\\
8903.67106842737	0.00109560558006662\\
8907.67226890756	0.00109486745548731\\
8911.67346938776	0.0010941301593433\\
8915.67466986795	0.00109339369033345\\
8919.67587034814	0.00109265804715926\\
8923.67707082833	0.00109192322852485\\
8927.67827130852	0.00109118923313694\\
8931.67947178872	0.00109045605970488\\
8935.68067226891	0.00108972370694059\\
8939.6818727491	0.00108899217355862\\
8943.68307322929	0.00108826145827608\\
8947.68427370948	0.00108753155981267\\
8951.68547418968	0.00108680247689066\\
8955.68667466987	0.00108607420823491\\
8959.68787515006	0.00108534675257282\\
8963.68907563025	0.00108462010863434\\
8967.69027611044	0.001083894275152\\
8971.69147659064	0.00108316925086084\\
8975.69267707083	0.00108244503449847\\
8979.69387755102	0.001081721624805\\
8983.69507803121	0.00108099902052308\\
8987.6962785114	0.00108027722039788\\
8991.6974789916	0.00107955622317708\\
8995.69867947179	0.00107883602761086\\
8999.69987995198	0.00107811663245192\\
9003.70108043217	0.00107739803645543\\
9007.70228091236	0.00107668023837906\\
9011.70348139256	0.00107596323698297\\
9015.70468187275	0.00107524703102977\\
9019.70588235294	0.00107453161928456\\
9023.70708283313	0.00107381700051491\\
9027.70828331332	0.00107310317349083\\
9031.70948379352	0.0010723901369848\\
9035.71068427371	0.00107167788977172\\
9039.7118847539	0.00107096643062895\\
9043.71308523409	0.00107025575833629\\
9047.71428571429	0.00106954587167594\\
9051.71548619448	0.00106883676943255\\
9055.71668667467	0.00106812845039318\\
9059.71788715486	0.00106742091334729\\
9063.71908763505	0.00106671415708675\\
9067.72028811525	0.00106600818040583\\
9071.72148859544	0.0010653029821012\\
9075.72268907563	0.00106459856097189\\
9079.72388955582	0.00106389491581935\\
9083.72509003601	0.00106319204544738\\
9087.72629051621	0.00106248994866216\\
9091.7274909964	0.00106178862427222\\
9095.72869147659	0.00106108807108846\\
9099.72989195678	0.00106038828792414\\
9103.73109243697	0.00105968927359486\\
9107.73229291717	0.00105899102691854\\
9111.73349339736	0.00105829354671547\\
9115.73469387755	0.00105759683180826\\
9119.73589435774	0.00105690088102182\\
9123.73709483794	0.00105620569318341\\
9127.73829531813	0.00105551126712259\\
9131.73949579832	0.00105481760167123\\
9135.74069627851	0.0010541246956635\\
9139.7418967587	0.00105343254793587\\
9143.7430972389	0.00105274115732708\\
9147.74429771909	0.0010520505226782\\
9151.74549819928	0.00105136064283253\\
9155.74669867947	0.00105067151663569\\
9159.74789915966	0.00104998314293553\\
9163.74909963986	0.00104929552058218\\
9167.75030012005	0.00104860864842804\\
9171.75150060024	0.00104792252532775\\
9175.75270108043	0.00104723715013819\\
9179.75390156062	0.00104655252171849\\
9183.75510204082	0.00104586863893003\\
9187.75630252101	0.00104518550063639\\
9191.7575030012	0.00104450310570341\\
9195.75870348139	0.00104382145299913\\
9199.75990396159	0.00104314054139381\\
9203.76110444178	0.00104246036975991\\
9207.76230492197	0.00104178093697212\\
9211.76350540216	0.0010411022419073\\
9215.76470588235	0.00104042428344453\\
9219.76590636255	0.00103974706046506\\
9223.76710684274	0.00103907057185233\\
9227.76830732293	0.00103839481649195\\
9231.76950780312	0.00103771979327173\\
9235.77070828331	0.00103704550108162\\
9239.77190876351	0.00103637193881375\\
9243.7731092437	0.00103569910536239\\
9247.77430972389	0.00103502699962398\\
9251.77551020408	0.0010343556204971\\
9255.77671068427	0.00103368496688247\\
9259.77791116447	0.00103301503768295\\
9263.77911164466	0.00103234583180355\\
9267.78031212485	0.00103167734815137\\
9271.78151260504	0.00103100958563565\\
9275.78271308523	0.00103034254316777\\
9279.78391356543	0.00102967621966119\\
9283.78511404562	0.00102901061403148\\
9287.78631452581	0.00102834572519633\\
9291.787515006	0.00102768155207552\\
9295.78871548619	0.00102701809359091\\
9299.78991596639	0.00102635534866647\\
9303.79111644658	0.00102569331622823\\
9307.79231692677	0.0010250319952043\\
9311.79351740696	0.00102437138452489\\
9315.79471788715	0.00102371148312223\\
9319.79591836735	0.00102305228993066\\
9323.79711884754	0.00102239380388655\\
9327.79831932773	0.00102173602392832\\
9331.79951980792	0.00102107894899647\\
9335.80072028811	0.0010204225780335\\
9339.80192076831	0.00101976690998398\\
9343.8031212485	0.00101911194379449\\
9347.80432172869	0.00101845767841367\\
9351.80552220888	0.00101780411279216\\
9355.80672268908	0.00101715124588262\\
9359.80792316927	0.00101649907663974\\
9363.80912364946	0.0010158476040202\\
9367.81032412965	0.00101519682698269\\
9371.81152460984	0.00101454674448792\\
9375.81272509004	0.00101389735549857\\
9379.81392557023	0.00101324865897933\\
9383.81512605042	0.00101260065389687\\
9387.81632653061	0.00101195333921982\\
9391.8175270108	0.00101130671391883\\
9395.818727491	0.00101066077696649\\
9399.81992797119	0.00101001552733737\\
9403.82112845138	0.00100937096400799\\
9407.82232893157	0.00100872708595686\\
9411.82352941176	0.0010080838921644\\
9415.82472989196	0.00100744138161302\\
9419.82593037215	0.00100679955328705\\
9423.82713085234	0.00100615840617277\\
9427.82833133253	0.00100551793925839\\
9431.82953181273	0.00100487815153405\\
9435.83073229292	0.00100423904199184\\
9439.83193277311	0.00100360060962574\\
9443.8331332533	0.00100296285343167\\
9447.83433373349	0.00100232577240744\\
9451.83553421369	0.00100168936555281\\
9455.83673469388	0.00100105363186939\\
9459.83793517407	0.00100041857036074\\
9463.83913565426	0.000999784180032293\\
9467.84033613445	0.000999150459891364\\
9471.84153661465	0.000998517408947172\\
9475.84273709484	0.000997885026210811\\
9479.84393757503	0.000997253310695254\\
9483.84513805522	0.000996622261415349\\
9487.84633853541	0.000995991877387811\\
9491.84753901561	0.000995362157631223\\
9495.8487394958	0.000994733101166025\\
9499.84993997599	0.000994104707014517\\
9503.85114045618	0.000993476974200849\\
9507.85234093638	0.000992849901751019\\
9511.85354141657	0.000992223488692871\\
9515.85474189676	0.000991597734056084\\
9519.85594237695	0.000990972636872175\\
9523.85714285714	0.000990348196174492\\
9527.85834333734	0.000989724410998208\\
9531.85954381753	0.00098910128038032\\
9535.86074429772	0.00098847880335964\\
9539.86194477791	0.000987856978976799\\
9543.8631452581	0.000987235806274234\\
9547.8643457383	0.000986615284296187\\
9551.86554621849	0.000985995412088706\\
9555.86674669868	0.00098537618869963\\
9559.86794717887	0.000984757613178597\\
9563.86914765906	0.000984139684577032\\
9567.87034813925	0.000983522401948142\\
9571.87154861945	0.000982905764346919\\
9575.87274909964	0.00098228977083013\\
9579.87394957983	0.000981674420456314\\
9583.87515006002	0.00098105971228578\\
9587.87635054022	0.0009804456453806\\
9591.87755102041	0.000979832218804608\\
9595.8787515006	0.000979219431623394\\
9599.87995198079	0.000978607282904299\\
9603.88115246098	0.000977995771716413\\
9607.88235294118	0.000977384897130573\\
9611.88355342137	0.000976774658219353\\
9615.88475390156	0.000976165054057064\\
9619.88595438175	0.000975556083719752\\
9623.88715486194	0.000974947746285188\\
9627.88835534214	0.000974340040832869\\
9631.88955582233	0.000973732966444015\\
9635.89075630252	0.000973126522201557\\
9639.89195678271	0.000972520707190145\\
9643.8931572629	0.000971915520496133\\
9647.8943577431	0.000971310961207584\\
9651.89555822329	0.000970707028414257\\
9655.89675870348	0.000970103721207614\\
9659.89795918367	0.000969501038680806\\
9663.89915966387	0.000968898979928674\\
9667.90036014406	0.000968297544047748\\
9671.90156062425	0.000967696730136235\\
9675.90276110444	0.000967096537294024\\
9679.90396158463	0.000966496964622674\\
9683.90516206483	0.000965898011225418\\
9687.90636254502	0.000965299676207155\\
9691.90756302521	0.000964701958674445\\
9695.9087635054	0.000964104857735507\\
9699.90996398559	0.000963508372500218\\
9703.91116446579	0.000962912502080103\\
9707.91236494598	0.000962317245588337\\
9711.91356542617	0.000961722602139739\\
9715.91476590636	0.000961128570850766\\
9719.91596638655	0.000960535150839515\\
9723.91716686675	0.000959942341225712\\
9727.91836734694	0.000959350141130715\\
9731.91956782713	0.000958758549677506\\
9735.92076830732	0.00095816756599069\\
9739.92196878752	0.00095757718919649\\
9743.92316926771	0.000956987418422741\\
9747.9243697479	0.000956398252798893\\
9751.92557022809	0.000955809691456\\
9755.92677070828	0.000955221733526719\\
9759.92797118848	0.000954634378145311\\
9763.92917166867	0.00095404762444763\\
9767.93037214886	0.000953461471571123\\
9771.93157262905	0.000952875918654827\\
9775.93277310924	0.000952290964839364\\
9779.93397358944	0.00095170660926694\\
9783.93517406963	0.000951122851081336\\
9787.93637454982	0.00095053968942791\\
9791.93757503001	0.000949957123453592\\
9795.9387755102	0.000949375152306878\\
9799.9399759904	0.00094879377513783\\
9803.94117647059	0.00094821299109807\\
9807.94237695078	0.000947632799340777\\
9811.94357743097	0.000947053199020684\\
9815.94477791117	0.000946474189294076\\
9819.94597839136	0.000945895769318782\\
9823.94717887155	0.000945317938254176\\
9827.94837935174	0.000944740695261172\\
9831.94957983193	0.000944164039502221\\
9835.95078031213	0.000943587970141305\\
9839.95198079232	0.000943012486343938\\
9843.95318127251	0.000942437587277158\\
9847.9543817527	0.000941863272109529\\
9851.95558223289	0.00094128954001113\\
9855.95678271309	0.000940716390153559\\
9859.95798319328	0.000940143821709928\\
9863.95918367347	0.000939571833854853\\
9867.96038415366	0.000939000425764462\\
9871.96158463385	0.000938429596616381\\
9875.96278511404	0.000937859345589738\\
9879.96398559424	0.000937289671865154\\
9883.96518607443	0.000936720574624745\\
9887.96638655462	0.000936152053052116\\
9891.96758703481	0.000935584106332356\\
9895.96878751501	0.00093501673365204\\
9899.9699879952	0.000934449934199218\\
9903.97118847539	0.000933883707163418\\
9907.97238895558	0.000933318051735642\\
9911.97358943577	0.00093275296710836\\
9915.97478991597	0.000932188452475508\\
9919.97599039616	0.000931624507032485\\
9923.97719087635	0.000931061129976152\\
9927.97839135654	0.000930498320504822\\
9931.97959183673	0.000929936077818265\\
9935.98079231693	0.000929374401117699\\
9939.98199279712	0.00092881328960579\\
9943.98319327731	0.000928252742486647\\
9947.9843937575	0.00092769275896582\\
9951.98559423769	0.000927133338250296\\
9955.98679471789	0.000926574479548494\\
9959.98799519808	0.000926016182070267\\
9963.98919567827	0.000925458445026893\\
9967.99039615846	0.000924901267631078\\
9971.99159663866	0.000924344649096946\\
9975.99279711885	0.000923788588640041\\
9979.99399759904	0.000923233085477321\\
9983.99519807923	0.000922678138827158\\
9987.99639855942	0.000922123747909331\\
9991.99759903962	0.000921569911945025\\
9995.99879951981	0.000921016630156829\\
10000	0.000920463901768731\\
};
\end{axis}
\end{tikzpicture}%